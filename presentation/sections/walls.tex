
\subsection{Några definitioner}

\frame{
        \begin{align}
        Q &= U\Delta T \\
        Q &= h(T-T_\infty)
        \end{align}
}

\subsection{Parametrar för huset}

\frame{
        \begin{table}[hbtp]
        \centering
        \caption{Areor och U-värden för fastighetens klimatsköld.}
        \label{tbl:uvalue}

        \begin{tabular}{|l|r|r|}
        \hline
        \textbf{Del} & \textbf{Area~$[\unit{m^2}]$} &\textbf{U-värde~$[\unit{W~m^{-2}~K^{-1}}]$} \\
        \hline
        Söderväggen &  151 & 1,186 \\ 
        Västerväggen & 61 & 1,186 \\
        Norrväggen & 290 & 0,279 \\
        Burspråket & 47 & 0,393 \\
        Taket & 257 & 0,171 \\
        \hline
        Fönster, söder & 109 & 1,0 \\
        Fönster, norr & 89 & 1,0 \\
        Fönster, tak & 8 & 1,0 \\
        \hline
        \textbf{Totalt} & \textbf{1012} & \textbf{0,6}\\
        \hline
        \end{tabular}
        \end{table}
        \emph{\color{red} h-värdes kurvan kanske ska in hit också?}
}

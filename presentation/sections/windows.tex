\subsection{Fönster}

\subsubsection{Solens position och intensitet}
\frame{
  Bild på solens position en solig decemberdag
}

\subsubsection{Direkt strålning, konduktion, konvektion och IR}
\frame{
  \begin{align*}
    g\left( \theta \right) & = g_0 \cdot p\left( \theta \right)\\[10pt]
    p\left( \theta \right) & \propto \text{Antal rutor \& beläggningar}\\[10pt]
    Q_{sol} \,\,\,\, & = g\left( \theta \right) \cdot I_0 \cos{\left( \theta \right)}\\[10pt]
    Q_{kond} & = \frac{1}{\frac{1}{U}+\frac{1}{h}} \left( T_{inne} - T_{ute}\right)\\[10pt]
    Q_{konv} & = 0,75 \cdot \sigma \left( T_{inne}^4 - T_{ute}^4\right)
  \end{align*}
}

% Se Karlsson, J. och Roos, A. (2000) för g-värden
% Förklaring av hur vi räknat med S-B:s lag
% Diffus strålning och omedelbar reflektion borträknat

\subsubsection{Totalt genom fönster}
\frame{
  Bild på alla flöden som ges av föregående slide
}

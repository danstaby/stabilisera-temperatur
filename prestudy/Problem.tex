Många moderna värmesystem är programmerade enligt vissa konstanta egenskaper hos huset, som byggnadsmaterial, storlek, isolering samt i vissa fall hur mycket värme som tillförs genom antalet boende och värmeproducerande apparatur. Den enda variabeln som tas hänsyn till är därefter utetemperaturen. Uppvärmningen regleras alltså varken efter husets eller värmesystemets tröghet, eller någon annan väderfaktor.

Ett första steg i vårt arbete blir att ta fram ett energiflödesschema för fastigheten där alla konstanta flöden kan identifieras. Sådana kan till exempel vara cirkulerande varmvatten för hushållsbruk, värmeproducerande vitvaror samt kroppsvärme från människor. Denna energi kan inte påverkas på samma sätt som radiatorvärmen och bör kunna sättas konstant. Genom att identifiera alla dess källor kan vi också finna ett värde för den.

Nästa steg blir att utöka energiflödesschemat med de väderparametrar\footnote{Temperatur, nedebörd, vindriktning, vindfart, luftfuktighet, lufttryck, solintensitet.} som kan mätas med väderstationen som finns monterad på fastighetens tak. Dessa ger eller tar energi från fastigheten och kan inte heller påverkas. De är dock variabla och kräver därför att man närmare studerar hur de påverkar fastighetens temperatur. Två begrepp som vi då har till vår hjälp är oreglerad (free-running) temperatur – temperatur i lägenheten utan radiatoruppvärmning samt ekvivalent temperatur – alla väderparametrar sammanlagda.

Genom att veta hur mycket energi som går in och ut ur huset i kombination med väderdata och kunskap om trögheten i väggarna kan vi beräkna hur mycket energi som behöver tillföras för att en behaglig boendetemperatur ska kunna uppnås. På grund av just fördöjningen i väggar och tak vet vi helt säkert vilket väder vi ska räkna med, för när vi väl ska reglera för värdet har det redan hänt. Vi söker en oreglerad (free-running) temperatur med vilken vi kan bestämma hur stort energiinflöde som behövs i varje ögonblick. Detta ger oss möjlighet att ta fram en modell för styrsystemet. Vi kommer här att se på fastigheten som en enhet med en temperatur.

Vi hoppas att vi kan få data från fastigheten för att även kunna göra statistiska modeller och jämföra dem med våra teoretiska resultat. Finns inte den möjligheten vill vi ändå ta fram förslag på hur statistisk data kan behandlas när det väl finns tillgång till den. Utifrån dessa modeller bör det även gå att beräkna en approximativ siffra på energibesparingen vid infört system.

Som en sista del avser vi att undersöka hur reglersystemet ska byggas upp då man har termometrar i varje rum och hur systemet skulle kunna göras självförbättrande.
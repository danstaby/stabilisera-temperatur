En detaljerad specifikation av värmeanläggningen och instruktioner för hur den ska drivas för att optimera energiförbrukningen kommer inte att ges i denna rapport, eftersom detta för sig självt skulle kunna omfattas av ett eget kandidatarbete. Baserat på resultatet av detta arbete ska man däremot kunna bygga vidare mot en sådan tillämpning, med exempelvis dimensionering av ett reglersystem som följd.

Om jämförelser med empiriska resultat kommer vara möjliga är i nuläget osäkert. Det beror på i vilken grad vi får tillgång till data från väderstationen, men också på om vi kommer kunna använda oss av data från utomstående part. Det skulle vara möjligt att jämföra data från väderstationen med väderprognoser från SMHI eller motsvarande. Utifrån detta kommer vi att analysera fördelarna med att använda prognosstyrning.

Vi kommer att bortse från snabba temperaturfluktuationer orsakade av exempelvis öppna fönster eller värmeeffekten från onormalt många människor i lägenheterna. Dessa är svåra, om inte omöjliga, att förutse.
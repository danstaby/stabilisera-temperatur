\documentclass[11pt,a4paper]{article}
\usepackage[utf8]{inputenc}
\usepackage[T1]{fontenc}
\usepackage[swedish]{babel}
\usepackage{amsmath}
\usepackage{ae}
\usepackage{units}
\usepackage{icomma}
\usepackage{color}
\usepackage{graphicx}
\usepackage{bbm}
\usepackage{textcomp}
\usepackage{url}
\usepackage{verbatim}
\usepackage{subfig}
\usepackage{marvosym}
\usepackage{eso-pic}
\usepackage{fancyhdr}
\usepackage{amssymb}
\usepackage[margin=1in]{geometry}
\usepackage{cite}
\usepackage{multicol}
\usepackage{wrapfig}
\usepackage{calc}
\usepackage{minibox}
\usepackage[table,dvipsnames]{xcolor}
\usepackage{multirow}

\begin{document}

\pagenumbering{gobble}
\pagestyle{fancy}
\lhead{\sc\footnotesize Pitch till CTK Bachelor's Challenge}
\rhead{\sc\footnotesize \today}
\mbox{}
%\vspace{2cm}

\begin{center}
\textcolor{ForestGreen}{\textbf{\Large För ett bekvämt och grönt boende}}
\end{center}

\mbox{}
\vspace{-5mm}

\setlength{\columnsep}{5mm}
%\setlength{\columnseprule}{0.5mm}
\begin{multicols}{2}
\normalsize

\begin{comment}
Vårt kandidatarbete, ''Stabilisera temperaturen i en fastighet med avseende på väderförändringar'', genomförs vid institutionen för Teknisk Fysik. Arbetsgruppen består av Mats Lindström, Dan Ståby, Ylva Dahl och Erik Ahlqvist och i vår mening passar projektet utmärkt under kategorin ''Green Initiative \& Sustainable Development''.
\end{comment}

% Ingress
\textbf{I dagens samhälle blir kontrollen över byggnaders energikonsumtion allt viktigare, en naturlig följd av både miljömedvetenhet och skenande elpriser. En stor del av en byggnads energianvändning går åt till uppvärmning, vilket föranleder frågeställningen: kan man minimera denna förbrukning genom att ta hänsyn till vädrets makter?}

En av vår tids stora utmaningar är att drastiskt minska vår påverkan på miljö och klimat utan att sänka människors levnadsstandard. Bland de stora energibovarna återfinns uppvärmningen av äldre fastigheter som inte byggdes med hänsyn tagen till energiförbrukning.

Vår uppdragsgivare håller för närvarande på att renovera värme- och ventilationsanläggningen i ett flerbostadshus i Göteborg, med siktet inställt på energibesparingar och behagligare inomhusklimat. I dagsläget styrs radiatorernas  temperatur endast av den momentana utomhustemperaturen i en referenspunkt på husets norrsida. Eftersom värmeförluster genom väggar och tak även beror på parametrar såsom solintensitet, luftfuktighet och vindhastighet är detta system alltför primitivt för att ge någon större noggrannhet, vilket orsakar stora fluktuationer i inomhustemperaturen. 

% Något om tidsfördröjning?
Målet med detta projekt är att utarbeta förslag till ett system som förutser energibehovet i byggnaden baserat på den sammanvägda effekten av olika väderparametrar och att undersöka hur mycket energi som kan sparas vid implementation av en sådan metod. Detta för 
\renewcommand{\arraystretch}{1.2}
\noindent
\resizebox{8cm}{!} {
\begin{tabular}{l r}
\hline \hline
\multicolumn{2}{||c||}{\cellcolor{ForestGreen} \textbf{Projektnamn:}}\\
\multicolumn{2}{||c||}{\cellcolor{ForestGreen} \textit{Stabilisera temperaturen i en fastighet}}\\
\multicolumn{2}{||c||}{\cellcolor{ForestGreen} \textit{med avseende på väderförändringar}} \\
\hline
\multicolumn{2}{||c||}{\cellcolor{ForestGreen} \textbf{Gruppmedlemmar:}} \\
\multicolumn{1}{||l}{\cellcolor{ForestGreen} Erik Ahlqvist (E)} & \multicolumn{1}{r||}{\cellcolor{ForestGreen} Ylva Dahl (F)}\\
\multicolumn{1}{||l}{\cellcolor{ForestGreen} Mats Lindström (F)} & \multicolumn{1}{r||}{\cellcolor{ForestGreen} Dan Ståby (F)}\\ \hline
\multicolumn{2}{||c||}{\cellcolor{ForestGreen} Institutionen för Teknisk Fysik} \\
\hline
\multicolumn{2}{||c||}{\cellcolor{ForestGreen} \textbf{Kategori:}} \\
\multicolumn{2}{||c||}{\cellcolor{ForestGreen} Green Initiative \&}\\
\multicolumn{2}{||c||}{\cellcolor{ForestGreen} Sustainable Development}\\
\hline \hline
& \\
\end{tabular}
}
att fastigheten i högre grad skall följa naturens variationer istället för att motarbeta dem och slösa energi.

Detta tillvägagångssätt kommer även jämföras med diverse andra tekniska och byggnadsmässiga lösningar i syfte att finna de metoder som lättast, billigast och grönast kan implementeras för att spara än mer energi i uppvärmningsprocessen.

För lösning av problemet används en mängd olika metoder. Vår repertoar består exempelvis av simuleringar med finita elementmetoden (FEM), analytiska beräkningar av differentialekvationer och djuplodade litteraturstudier över tidigare forskning.

Resultatet kommer kunna generaliseras för framtida användning i andra byggnader. Särskilt givande blir systemet då på äldre fastigheter som inte byggdes med miljöaspekten i åtanke och idag slösar mängder av energi.

\end{multicols}
\end{document}

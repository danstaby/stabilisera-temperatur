\documentclass[11pt,a4paper]{article}
\usepackage[utf8]{inputenc}
\usepackage[T1]{fontenc}
\usepackage[swedish]{babel}
\usepackage{amsmath}
\usepackage{ae}
\usepackage{units}
\usepackage{icomma}
\usepackage{color}
\usepackage{graphicx}
\usepackage{bbm}
\usepackage{textcomp}
\usepackage{url}
\usepackage{verbatim}
\usepackage{subfig}
\usepackage{marvosym}
\usepackage{eso-pic}
\usepackage{fancyhdr}
\usepackage{amssymb}
\usepackage[margin=1in]{geometry}
\usepackage{cite}
\usepackage{multicol}
\usepackage{wrapfig}
\usepackage{calc}
\usepackage{minibox}
\usepackage[table,dvipsnames]{xcolor}
\usepackage{multirow}

\begin{document}

\addtolength{\parindent}{-0.6 cm}
\pagenumbering{gobble}
\pagestyle{fancy}
\lhead{\sc\footnotesize Executive Summary till CTK Bachelor's Challenge}
\rhead{\sc\footnotesize \today}
\mbox{}
%\vspace{4mm}

\begin{center}
\textcolor{ForestGreen}{\textbf{\Huge För ett behagligt och grönt boende}}
\end{center}

\mbox{}
%\vspace{4mm}

\setlength{\columnsep}{5mm}
%\setlength{\columnseprule}{0.5mm}
\begin{multicols}{2}
\addtolength{\parskip}{1.2ex}
\linespread{1.02}
\normalsize



% Ingress
\textbf{Vi har undersökt hur vädret påverkar energiflödena genom en fastighet. Genom att ta hänsyn till fler väderparametrar än utomhustemperaturen kan energikonsumtionen minskas vilket i förlängningen bevarar miljön. Dessutom fås ett jämnare, och där med trivsammare, inomhusklimat.}

%Argument 1: Spara energi
Genom att beräkna de olika energiflödena – ledning, konvektion och strålning – genom en fastighets olika gränsytor – väggar, fönster, tak och grund – har det totala energiflödet vid olika väder kunnat beräknats. Detta har sedan jämförts med det energiflöde man tror sig få vid reglering efter enbart utomhustemperaturen. 
Det har visat sig att både sol och vind kan
ha markant effekt på inomhusklimatet och således energiförbrukningen. Det 
förstnämnda leder till ett nettoinflöde av energi vilket gör att 17 \%
kan sparas genom att ta hänsyn till solinstrålningen en solig dag. Det motsvara ca 5 \% av den totala energikostnaden. Vinden
ger däremot ett nettoutflöde av energi, så där går det bara att spara energi
om fastigheten har för hög inomhus\-temperaturer då det inte blåser.

%Argument 2: Stabilisera temperatur
Varierande väder resulterar även i fluktuerande inomhustemperatur. För maximal komfort är detta något som helst skall undvikas. Genom att ta hänsyn till väderparametrar vid uppvärmning av fastigheter kan temperaturen stabilseras. Detta bidrar till att en uppvärmningsanläggning som tar hänsyn till väderparametrar är en intressant produkt.

%Argument 3: Många fastigheter - Stor marknad samt många bäckar...                     
I Sverige regleras värmen i dagsläget endast med hänsyn till den momentana utomhustemperaturen i så gott som alla fastigheter. Vid en kommersialisering av våra
resultat finns således många potentiella kunder. Detta leder till att även vid en liten
energibesparing blir mängden sparad energi för de boende, och för samhället,
stor i slutändan.

\renewcommand{\arraystretch}{1.2}
\noindent
\resizebox{8cm}{!} {
\begin{tabular}{l r}
\hline
\multicolumn{2}{|c|}{\cellcolor{YellowGreen} \textbf{Projektnamn:}}\\[3pt]
\multicolumn{2}{|c|}{\cellcolor{YellowGreen} \textit{Väderparametrars inverkan på}}\\
\multicolumn{2}{|c|}{\cellcolor{YellowGreen} \textit{energiförluster i en fastighet -}}\\
\multicolumn{2}{|c|}{\cellcolor{YellowGreen} \textit{en studie av värmeflöden}}\\
\multicolumn{2}{|c|}{\cellcolor{YellowGreen} \textbf{Gruppmedlemmar:}} \\[3pt]
\multicolumn{1}{|l}{\cellcolor{YellowGreen} Erik Ahlqvist (E)} & \multicolumn{1}{r|}{\cellcolor{YellowGreen} Ylva Dahl (F)}\\
\multicolumn{1}{|l}{\cellcolor{YellowGreen} Mats Lindström (F)} & \multicolumn{1}{r|}{\cellcolor{YellowGreen} Dan Ståby (F)}\\
\multicolumn{2}{|c|}{\cellcolor{YellowGreen} Institutionen för Teknisk Fysik} \\
\multicolumn{2}{|c|}{\cellcolor{YellowGreen} \textbf{Kategori:}} \\[3pt]
\multicolumn{2}{|c|}{\cellcolor{YellowGreen} Green Initiative \&}\\
\multicolumn{2}{|c|}{\cellcolor{YellowGreen} Sustainable Development}\\
\hline
& \\
\end{tabular}
}

%Avslutning - Billigt, nära kommersiell produkt, Återknytning till 1,2,3
Vår modell kan enkelt implementeras i befintliga värmeanläggningar då den omvandlar informationen från väderstationen till korrigerad utomhustemperatur. I och med en låg kostnad för implementering och i kombination med en stor ackumulerad energivinst ligger det nära till hands att snart ha en färdig produkt ute på marknaden. 
En produkt baserad på denna princip bör inte få problem att bli kommersiellt gångbar även utan subvensioner. Väderstyrning av värmeanläggningar är en lättillgängligt åtgärd som kan nyttjas i många fastigheter och fler kan vara med och spara våra gemensamma resurser. 

Resultaten i vårt kandidatarbete talar för sig själva. Förbättringarna som fås vid en implementering är främst energibesparande. Mycket energi används för uppvärmning av fastigheter, och i kyliga klimat är det ypperligt att effektivisera uppvärmningsprocesser och minimera energiförluster. Det spar energi och det ger ett behagligare inomhusklimat, till en väldigt låg kostnad. Att minska energiförbrukningen genom effektivare processer är centralt i både dagens och framtidens teknologiska utveckling. 

För att inte förstöra vår planet i jakten på energi behöver alla dra sitt strå till stacken, och att implementera våra resultat i befintliga fastigheter, kommer definitivt att bidra till hållbar framtid.

\end{multicols}
\end{document}

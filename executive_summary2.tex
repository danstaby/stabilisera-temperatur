\documentclass[11pt,a4paper]{article}
\usepackage[utf8]{inputenc}
\usepackage[T1]{fontenc}
\usepackage[swedish]{babel}
\usepackage{amsmath}
\usepackage{ae}
\usepackage{units}
\usepackage{icomma}
\usepackage{color}
\usepackage{graphicx}
\usepackage{bbm}
\usepackage{textcomp}
\usepackage{url}
\usepackage{verbatim}
\usepackage{subfig}
\usepackage{marvosym}
\usepackage{eso-pic}
\usepackage{fancyhdr}
\usepackage{amssymb}
\usepackage[margin=1in]{geometry}
\usepackage{cite}
\usepackage{multicol}
\usepackage{wrapfig}
\usepackage{calc}
\usepackage{minibox}
\usepackage[table,dvipsnames]{xcolor}
\usepackage{multirow}

\begin{document}

\addtolength{\parindent}{-0.6 cm}
\pagenumbering{gobble}
\pagestyle{fancy}
\lhead{\sc\footnotesize Executive Summary till CTK Bachelor's Challenge}
\rhead{\sc\footnotesize \today}
\mbox{}
%\vspace{4mm}

\begin{center}
\textcolor{ForestGreen}{\textbf{\Huge För ett behagligt och grönt boende}}
\end{center}

\mbox{}
%\vspace{4mm}

\setlength{\columnsep}{5mm}
%\setlength{\columnseprule}{0.5mm}
\begin{multicols}{2}
\addtolength{\parskip}{1.3ex}
\linespread{1.02}
\normalsize



% Ingress
\textbf{Vi har undersökt hur vädret påverkar energiflödena genom en fastighet. Genom att ta hänsyn till fler väderparametrar än utomhustemperaturen kan energikonsumtionen minskas vilket i förlängningen bevarar miljön. Dessutom fås ett jämnare, och där med trivsammare, inomhusklimat vilket bara är en positiv bieffekt i vår förhoppning att kunna minska energiförbrukingen.}

%Argument 1: Spara energi
Genom att beräkna de olika energiflödena – ledning, konvektion och strålning – genom en fastighets olika gränsytor – väggar, fönster, tak och grund – har det totala energiflödet vid olika väder kunnat beräknats. Detta har sedan jämfört med det energiflöde man tror sig få vid reglering efter enbart utomhustemperaturen. 
Det har visat sig att både sol och vind kan
ha markant effekt på inomhusklimatet och således energiförbrukningen. Det 
förstnämnda leder till ett netto inflöde av energi vilket gör att en ganska
betydande del kan sparas (\emph{\color{red} Hur mycket?}). Vinden
ger däremot ett netto utflöde av energi och där går det inte att spara energi
om inte fastigheten har för höga temperaturer inomhus då det inte blåser utomhus.

%Argument 2: Stabilisera temperatur
Varienrande väder resulterar även i fluktuerande inomhustemperatur. För maximal komfort är detta något som helst skall undvikas. Genom att ta hänsyn till väderparametrar vid uppvärmning av fastigheter så kan temperaturen stabliseras. Detta bidrar till att en uppvärmningsanläggning som tar hänsyn till väderparametrar är en intressant produkt.

%Argument 3: Många fastigheter - Stor marknad samt många bäckar...                     
På platser med kyligare klimat regleras värmen i dagsläget med hänsyn till endast utomhustemperatur,
vilket sker i nästan alla fastigheter. Vid en kommersialisering av våra
resultat finns således många potentiella kunder. Detta leder till att även vid en liten
energibesparing så blir mängden sparad energi för samhället och de boende, väldigt
stor i slutändan.


\renewcommand{\arraystretch}{1.2}
\noindent
\resizebox{8cm}{!} {
\begin{tabular}{l r}
\hline
\multicolumn{2}{|c|}{\cellcolor{YellowGreen} \textbf{Projektnamn:}}\\[3pt]
\multicolumn{2}{|c|}{\cellcolor{YellowGreen} \textit{Väder\!parametrars inverkan på}}\\
\multicolumn{2}{|c|}{\cellcolor{YellowGreen} \textit{energi\!f\,örluster i en fastighe\!t -}}\\
\multicolumn{2}{|c|}{\cellcolor{YellowGreen} \textit{en s\!tu\!die av värme\!fl\,öden}}\\
\multicolumn{2}{|c|}{\cellcolor{YellowGreen} \textbf{Gruppmedlemmar:}} \\[3pt]
\multicolumn{1}{|l}{\cellcolor{YellowGreen} Erik Ahlqvist (E)} & \multicolumn{1}{r|}{\cellcolor{YellowGreen} Ylva Dahl (F)}\\
\multicolumn{1}{|l}{\cellcolor{YellowGreen} Mats Lindström (F)} & \multicolumn{1}{r|}{\cellcolor{YellowGreen} Dan Ståby (F)}\\
\multicolumn{2}{|c|}{\cellcolor{YellowGreen} Institutionen för Teknisk Fysik} \\
\multicolumn{2}{|c|}{\cellcolor{YellowGreen} \textbf{Kategori:}} \\[3pt]
\multicolumn{2}{|c|}{\cellcolor{YellowGreen} Green Initiative \&}\\
\multicolumn{2}{|c|}{\cellcolor{YellowGreen} Sustainable Development}\\
\hline
& \\
\end{tabular}
}


%Avslutning - Billigt, nära kommersiell produkt, Återknytning till 1,2,3
För att implementera vår modell kan nuvarande värmeanläggningar med lätthet användas då utomhustemperaturen korrigeras med avseende på andra väderparametrar.
I och med att resultaten är billiga att implementera och kan ge en stor ackumulerad energivinst ligger det nära till hands att snart ha en färdig produkt ute på marknaden. En lättillgängligt åtgärd kan nyttjas i många fastigheter och fler kan vara med och spara in på våra gemensamma resurser. 

En essentiell del av dagens och morgondagens teknologiska utveckling måste vara
att minska energiförbrukningen genom effektivare processer. Då mycket energi
används för uppvärmning av fastigheter, speciellt i kyliga klimat
är det ypperligt att effektivisera uppvärmningsprocesser och
minimera energiflörluster. Just den typen av förbättring som vårt kandidatarbete
har studerat är väldigt bra på att uppfylla detta då det dels sparar mycket energi
och dels är billigt att implementera. Detta innebär att en produkt baserad på
denna princip även blir kommersiellt gångbar utan subventioner. Ett problem
som många andra miljöförbättringar har.


\end{multicols}
\end{document}

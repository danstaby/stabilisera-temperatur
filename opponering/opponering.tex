Förslag till opponeringspunkter

Upplägg och struktur:
	+ Tydligt, systematiskt, lätttillgängligt, intresseväckande.

Problemdef:
	+ Väl presenterat och väl besvarat

Avgränsningar:
	+ Tydligt presenterade och väl motiverade
	- Varför har ni valt bort deplacerad ventilation?

Teori:
	- Vilka fysikaliska metoder har använts för att att lösa problemen?
	- Hur många partiklar innehöll tilluften? Hur renades den?

Metod/Genomförande: 
	- Val av metod? För- och nackdelar? Alternativa metoder?
	- Hur fungerar dessa Comsol-moduler?
	- Vilka randvillkor och fysikaliska antaganden har gjorts?
	- Rummets uppvärmning?
	- Partiklarnas form och påverkan på transport? Antal partiklar?

Resultat och analys:
	+ Väl underbyggda slutsater.
	+ Direkt applicerbara åtgärder.
	+ Stark koppling mellan frågeställning och resultat.

Diskussion:
	- Vilka felkällor kan tänkas ha introducerats i och med de begränsningar som 	satts upp?

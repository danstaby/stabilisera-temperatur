\chapter{Inledning}

Tidigare har det gjorts tre arbeten angående samma fastighet.  Men endast två av dem har använts, då det tredje enligt rekommendation var för gammalt och inte i samma riktning som det här arbetet var tänkt att gå.
Det första av dem Energiflödet genom ett hus, Teknisk fysik 2008 behandlar energiflödet genom ett hus, och hur väl det stämmer överens med ett befintligt beräkningsprogram. I arbetet läggs stor vikt vid jämförelse samt utvärdering av befintligt program för beräkning av energiflöden genom hus.
Det andra, Optimal energihushållning i en fastighet har i högra grad samma infallsvinklar som detta arbetet, men större och överskådligare. De använder sig av samt jämför tre olika simuleringsprogram, och arbetet spänner över en stor vidd, men på ett väldigt övergripligt sätt.

\subsection{Bakgrund}

Energi flödar hela tiden in och ut ur fastigheter, bland annat genom människors kroppsvärme, VVA (värme, ventilation och avlopp) och vädret ute. Dessa energiflöden kan delas in i konstanta energiflöden samt variabla energiflöden. Den främsta variabla energikällan är troligen vädret. Vädret kan, genom sina skiftningar, både ge och ta energi från byggnaden. För att bibehålla en jämn inomhustemperatur i fastigheten kan inte en konstant mängd energi tillföras av värmesystemet, utan energitillförseln måste hela tiden regleras utefter både konstanta samt variabla energiflöden.

I dagsläget regleras de flesta energisystem i fastigheter endast med tanke på utomhustemperaturen i varje ögonblick och på så sätt blir det alltid en fördröjning i uppvärmningen vilket i vårt fall leder till ojämn inomhustemperatur och eventuellt också till onödig energiåtgång.

Vår uppdragsgivare sköter utrustning för uppvärmning av fastigheten. Han har ett pågående projekt med syfte att minska energiförbrukningen i fastigheten samtidigt som ett behagligt inomhusklimat bibehålls. Inom ramen för detta så har en väderstation installerats på taket till fastigheten och sensorer av diverse slag har anslutits på strategiska platser.  Dessa enheter tillåter uppvärmningssystemet att anpassa energianvändningen efter väderlek.

Denna typ av effektivisering av energianvändningen i en fastighet är idag högaktuell på grund av höga energipriser och ökad förståelse för hur vår energianvändning kan påverka planeten negativt.


\section{Syfte}
Detta arbete syftar till att undersöka vilka energiförluster en fastighet har och hur dessa påverkas av olika väderrelaterade parametrar, som solinstrålning, utomhustemperatur och vind. Målet är att finna en kvanitativ beskrivning av hur energiflödena in och ut ur fastigheten påverkas av några olika väderparametrar. Primärt kommer vi att undersöka en fastighet på Walleriusgatan i Göteborg, men många av resultaten kommer att kunna appliceras även på andra byggnader.

Beskrivningen av hur vädret påverkar energiflödena skall sedan kunna leda till en modell för hur fastighetens reglerbara energiflöden ska kunna anpassas efter vädret så att en önskad inomhustemperatur bibehålls. I fastigheten som är associerad med projektet finns en väderstation monterad och det är ifrån den väderdata är tänkt att hämtas. Detta ska inte bara ge en trivsammare inomhusmiljö för de boende, utan även en energibesparing för fastigheten. Vi hoppas också kunna ge några byggnadstekniska förslag på åtgärder och kvantifiera hur stora besparingar detta skulle kunna ge – både energimässiga och ekonomiska.

Vi avser att främst bygga vår modell på att luften närmast fastigheten värms upp och bildar ett isolerande lager. Vi kommer också att ta hänsyn till både den fördröjning som sker i fastighetens väggar och den direkta uppvärmning och avkylning som sker med solinstrålning genom fönster respektive ofrivillig ventilation i form av drag.

På sikt hoppas vi att vårt arbete ska leda fram till ett modell som ger ett värde på hur mycket energi som behöver tillföras fastigheten i varje givet ögonblick, beroende på vilka värden väderstationen tidigare har tagit emot. I förlängningen ska detta kunna leda till energibesparande åtgärder för både den här och äldre fastigheter.

\section{Avgränsningar}

En detaljerad specifikation av värmeanläggningen och instruktioner för hur den ska drivas för att optimera energiförbrukningen kommer inte att ges i denna rapport. Troligen kan detta omfattas av ett eget kandidatarbete. Baserat på resultatet av detta arbete ska man däremot kunna bygga vidare mot en sådan tillämpning, med exempelvis dimensionering av ett reglersystem som följd.

Jämförelser med empiriska resultat har tyvärr inte varit möjligt då vi inte haft tillgång till nödvändig data, varken från väderstationen eller någon utomstående part.

Vi kommer också att bortse från oförutsägbara temperaturfluktuationer orsakade av exempelvis öppna fönster eller värmeeffekten från onormalt många människor i lägenheterna.

\subsection{Johanneberg 7:8 – en beskrivning av fastigheten på Walleniusgatan}

\subsubsection{Huset som bostad}
% Det är så här stort och har så här många rum, så här högt i tak o.s.v. 

% De olika gränsytornas material och uppbyggnad.
\subsubsection{Väggarna}

\subsubsection{Taket}

\subsubsection{Grunden}


\subsubsection{Huset byggandstil och histora}
% Hur tänkte de när de byggde och renoverade huset?
% Vad ville de uppnå och vilka regler och normer hade man att hålla sig till?

\subsubsection{Byggnadsfysikens betydelse}
% Att huset är byggt så här vad betyder det för hur huset påverkas och hur huset är att bo i?

\subsubsection{Uppvärmning}
% Kortfattat vad finns det för värmeförsörjningssystem idag?
% Hur fungerar det? Varför valde man det?

\subsubsection{Ventilation}
% Hur fungerar ventilationen?
\subsection{Dokumentets disposition}

Hur dokumentet är uppbyggt.


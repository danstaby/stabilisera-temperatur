\chapter{Inledning}

Energi flödar hela tiden in och ut ur fastigheter, bland annat genom människors aktivitet, ventilation och vädret ute. Dessa energiflöden kan delas in i konstanta och variabla där den främsta variabla energikällan är vädret. Vädret kan, genom sina skiftningar, både ge och ta energi från byggnaden. För att bibehålla en jämn inomhustemperatur i fastigheten kan inte en konstant mängd energi tillföras av värmesystemet, utan energitillförseln måste hela tiden regleras efter alla de opåverkbara energiflödena, konstanta så väl som variabla.

I dagsläget regleras de flesta energisystem i fastigheter endast med tanke på utomhustemperaturen i varje ögonblick och på så sätt blir det alltid en fördröjning i uppvärmningen vilket i vårt fall leder till ojämn inomhustemperatur och eventuellt också till onödig energiåtgång.

Projektets uppdragsgivare, Peter Särneö, sköter utrustning för uppvärmning av en äldre fastighet på Walleniusngatan i Göteborg. Det pågår ett projekt med syfte att minska energiförbrukningen i fastigheten samtidigt som ett behagligt inomhusklimat bibehålls. Inom ramen för detta så har en väderstation installerats på taket till fastigheten och sensorer av diverse slag har anslutits på strategiska platser.
%Vad har hänt och vad ska hända? Finns det sensorer utplacerade?
 Dessa enheter tillåter uppvärmningssystemet att anpassa energianvändningen efter väderlek. % Det är väl det vi ska tillse?

Denna typ av effektivisering av energianvändningen i en fastighet är idag högaktuell på grund av höga energipriser och ökad förståelse för hur vår energianvändning kan påverka klimatet negativt.

% Tidigare kandidatarbeten.
Det har tidigare gjorts tre kandidatarbeten som har undersökt fastigheten på Walleriusgatan. Vi har primärt byggt vårt arbete på de två senare, då det första arbetet mer tittade på fastighetens energisystem, vilket inte var relevant för oss. År 2008 respektive 2010 gjordes det dock kandidatarbeten inom området energieffektivisering av fastigheten på Walleniusgatan. 

Det första, \textit{Energiflödet genom ett hus}\cite{kandidatarbete2008}, behandlade mer allmänt energiflödet genom ett hus och hur väl det stämmer överens med en simulering ett befintligt beräkningsprogram. Arbetet syftar till att, på ett mer fysikaliskt sätt tolka termerna som används för energieffektivitet. Resultaten verifieras med ett välkänt simuleringsprogram, IDA Indoor Climate and Energy, och slutsatsen är att simuleringsprogram fungerar bra. En ytterligare slutsats är att solinstrålningen bidrar till en stor del av uppvärmningen, och att man för att utnyttja det skulle kunna koppla ett regulatorsystem till en väderstation.

Det andra, \textit{Optimal energihushållning i en fastighet}\cite{kandidatarbete2010}, går ytterligare ett steg vidare och tar fram underlag för att automatisera och optimera fastighetens värmeförsörjning i den befintlig fastigheten med hjälp av flera väderparametrar, fastighetens specifikationer samt den inre verksamheten. Det har i högre grad samma infallsvinklar som detta arbete, men ur ett mer övergripande perspektiv.

Sedan de tidigare projekten avslutats har fastighetens energiförsörjningssytem uppdateras och detta arbete blir en vidareutveckling av de modeller som kan användas för att modullera hur värme flödar in och ut ur byggnaden. Vi upplever att det inte är helt utrett hur stor påverkan de olika väderaspekterna har. Detta är essentiellt för att förstå var de energibesparande åtgärdena ska sättas in för att bli så effektiva som möjligt och i förlängningen eventuellt kunna dimesionera ett specialanpassat reglersystem.


\section{Syfte}
Detta arbete syftar till att undersöka vilka energiförluster en fastighet har och hur dessa påverkas av olika väderrelaterade parametrar, som solinstrålning, utomhustemperatur och vind. Målet är att finna en kvanitativ beskrivning av hur energiflödena in och ut ur fastigheten påverkas av några olika väderparametrar. Primärt kommer vi att undersöka en fastighet på Walleriusgatan i Göteborg, men många av resultaten kommer att kunna appliceras även på andra byggnader.

Beskrivningen av hur vädret påverkar energiflödena skall sedan kunna leda till en modell för hur fastighetens reglerbara energiflöden ska kunna anpassas efter vädret så att en önskad inomhustemperatur bibehålls. I fastigheten som är associerad med projektet finns en väderstation monterad och det är ifrån den väderdata är tänkt att hämtas. Detta ska inte bara ge en trivsammare inomhusmiljö för de boende, utan även en energibesparing för fastigheten. Vi hoppas också kunna ge några byggnadstekniska förslag på åtgärder och kvantifiera hur stora besparingar detta skulle kunna ge – både energimässiga och ekonomiska.

Vi avser att främst bygga vår modell på att luften närmast fastigheten värms upp och bildar ett isolerande lager. Vi kommer också att ta hänsyn till både den fördröjning som sker i fastighetens väggar och den direkta uppvärmning och avkylning som sker med solinstrålning genom fönster respektive ofrivillig ventilation i form av drag.

På sikt hoppas vi att vårt arbete ska leda fram till ett modell som ger ett värde på hur mycket energi som behöver tillföras fastigheten i varje givet ögonblick, beroende på vilka värden väderstationen tidigare har tagit emot. I förlängningen ska detta kunna leda till energibesparande åtgärder för både den här och äldre fastigheter.

\subsection{Dokumentets disposition}

Hur dokumentet är uppbyggt.

\subsection{Johanneberg 7:8 – en beskrivning av fastigheten på Walleniusgatan}

\subsubsection{Huset som bostad}
% Det är så här stort och har så här många rum, så här högt i tak o.s.v. 

% De olika gränsytornas material och uppbyggnad.
\subsubsection{Väggarna}

\subsubsection{Taket}

\subsubsection{Grunden}


\subsubsection{Huset byggandstil och histora}
% Hur tänkte de när de byggde och renoverade huset?
% Vad ville de uppnå och vilka regler och normer hade man att hålla sig till?

\subsubsection{Byggnadsfysikens betydelse}
% Att huset är byggt så här vad betyder det för hur huset påverkas och hur huset är att bo i?

\subsubsection{Uppvärmning}
% Kortfattat vad finns det för värmeförsörjningssystem idag?
% Hur fungerar det? Varför valde man det?

\subsubsection{Ventilation}
% Hur fungerar ventilationen?


\chapter{Inledning}

Energi flödar hela tiden in och ut ur fastigheter, bland annat genom människors aktivitet, ventilation och olika väderparametrar. Dessa energiflöden kan delas in i konstanta och variabla. Vädret är främsta variabla energikällan och vädret kan, genom sina skiftningar, både ge och ta energi från byggnaden. För att bibehålla en jämn inomhustemperatur i fastigheten kan inte en konstant mängd energi tillföras av värmesystemet, utan energitillförseln måste hela tiden regleras efter de opåverkbara energiflödena, där vädret är en stor del.

I dagsläget regleras många fastigheters energisystem endast med avseende på utomhustemperaturen i varje enskilt ögonblick och på så sätt blir det alltid en fördröjning i uppvärmningen. Det leder till ojämn inomhustemperatur och eventuellt också till onödig energiåtgång vilket även gäller fastigheten i det här projektet.

Projektets uppdragsgivare Peter Särneö sköter utrustning för uppvärmning av en äldre fastighet på Walleriusgatan i Göteborg. Han driver ett projekt med syfte att minska energiförbrukningen i fastigheten samtidigt som ett behagligt inomhusklimat bibehålls. Inom ramen för detta har en väderstation och en solintensitetsmätare installerats på taket till fastigheten. Dessa enheter ger en möjlighet att anpassa uppvärmningssystemet till aktuellt  väder. Denna typ av effektivisering av energianvändningen i en fastighet är idag högaktuell på grund av höga energipriser och ökad förståelse för hur vår energianvändning kan påverka klimatet negativt.

% Tidigare kandidatarbeten.
Det har tidigare gjorts tre kandidatarbeten som har undersökt fastigheten på Walleriusgatan. Detta projekt bygger primärt på de två senare, då det första arbetet studerade fastighetens energisystem, vilket ligger utanför ramarna för denna studie. De andra två gjordes åren 2008 och 2010 och då inom området energieffektivisering av fastigheten på Walleriusgatan. 

Det första kandidatarbetet, \textit{Energiflödet genom ett hus}\cite{kandidatarbete2008},
behandlade mer allmänt energiflödet genom ett hus och hur väl det stämmer
överens med en simulering gjorde med ett befintligt beräkningsprogram. Arbetet syftade
till att på ett mer fysikaliskt sätt tolka termerna som används för
energieffektivitet. För att kvantifiera energiflödena utfördes mätningar i
fastigheten. Resultaten verifieras med ett välkänt simuleringsprogram,
IDA Indoor Climate and Energy, och slutsatsen är att simuleringsprogramet fungerar bra.
De konstaterar även att för att mäta på fastigheten så är det många parametrar som
måste isoleras och de rekommenderar utökade mätningar.
 En ytterligare slutsats är att solinstrålningen bidrar till en stor del av
uppvärmningen, och att man för att utnyttja det skulle kunna koppla ett
regulatorsystem till en väderstation.

Det andra kandidatarbetet, \textit{Optimal energihushållning i en fastighet}\cite{kandidatarbete2010},
går ytterligare ett steg och tar fram underlag för att automatisera och
optimera den befintliga fastighetens värmeförsörjning utifrån
flera väderparametrar, fastighetens specifikationer och de boendes verksamhet. Det har i högre grad samma infallsvinklar som detta arbete, men ur ett mer övergripande perspektiv. En stor del av arbetet använder dessutom de avancerade energimodelleringsverktyg ''Revit Architecture 2010'' och 
''EcoTect Analysis 2010'' där en 3D-modell av fastigheten används och även studerar vädrets effekt med en väderfil över Göteborg.

Sedan de tidigare projekten avslutades har fastighetens energiförsörjningssytem uppdateras och detta arbete blir en vidareutveckling av de modeller som kan användas för att modellera hur värme flödar in i och ut ur byggnaden. Vi upplever att det inte är helt utrett hur stor påverkan de olika väderaspekterna, såsom vind, sol och regn, har. Detta är essentiellt för att förstå var de energibesparande åtgärdena ska sättas in för att bli så effektiva som möjligt och för att i förlängningen eventuellt ska kunna dimensionera ett specialanpassat reglersystem.

Ett närliggande område är prognosstyrning av fastigheters energisystem som har både likheter och olikheter med reglering utifrån det momentana och gångna vädret. Båda systemen styr värmesystem i byggnader med hjälp av olika väderparametrar samt byggnadsspecifika parametrar såsom material och täthet. Skillnaden är varifrån den hämtar väderinformationen. Prognosstyrning bygger på prognoser, vilket innebär vissa osäkerheter. När man styr med avseende på det momentana eller gångna vädret är nackdelen att parametrarna redan påverkat inomhustemperaturen innan reglersystemet kan korrigera för dem.

SMHI har under en tid arbetat med prognosstyrning av fastigheter. Deras projekt inleddes med en avhandling inom området av Roger Taesler. Den behandlar främst området ekvivalent temperatur, och från den synvinkeln har SMHI tagit fram ett system som anpassas specifikt till varje fastighet. De har konstruerat ett antal modeller över byggnader där de även tagit hänsyn till byggnadernas läge och när systemet installeras i en fastighet väljs den modell som stämmer bäst överens med det faktiska byggnaden. Systemet tar sedan emot prognoser från SMHI och reglerar radiatorsystemet efter dessa. Systemet är kommersiellt och SMHI förbättrar systemet löpande.

\section{Syfte}
Detta arbete syftar till att undersöka vilka energiförluster en fastighet har och hur dessa påverkas av olika väderrelaterade parametrar, som solinstrålning, utomhustemperatur och vind. Målet är att finna en kvanitativ beskrivning av hur energiflödena in och ut ur fastigheten påverkas av några olika väderparametrar. Primärt kommer vi att undersöka en fastighet på Walleriusgatan i Göteborg, men många av resultaten kommer att kunna appliceras även på andra byggnader.

Beskrivningen av hur vädret påverkar energiflödena skall sedan kunna leda till en modell för hur fastighetens reglerbara energiflöden ska kunna anpassas efter vädret så att en önskad inomhustemperatur bibehålls. I fastigheten som är associerad med projektet finns en väderstation monterad och det är ifrån den väderdata är tänkt att hämtas. Detta ska inte bara ge en trivsammare inomhusmiljö för de boende, utan även en energibesparing för fastigheten. Vi hoppas också kunna ge några byggnadstekniska förslag på åtgärder och kvantifiera hur stora besparingar detta skulle kunna ge – både energimässiga och ekonomiska.

Vi avser att främst bygga vår modell på att luften närmast fastigheten värms upp och bildar ett isolerande lager. Vi kommer också att ta hänsyn till både den fördröjning som sker i fastighetens väggar och den direkta uppvärmning och avkylning som sker med solinstrålning genom fönster respektive ofrivillig ventilation i form av drag.

På sikt hoppas vi att vårt arbete ska leda fram till ett modell som ger ett värde på hur mycket energi som behöver tillföras fastigheten i varje givet ögonblick, beroende på vilka värden väderstationen tidigare har tagit emot. I förlängningen ska detta kunna leda till energibesparande åtgärder för både den här och äldre fastigheter.

\subsection{Dokumentets disposition}

Hur dokumentet är uppbyggt.

\subsection{Johanneberg 7:8 – en beskrivning av fastigheten på Walleniusgatan}

\subsubsection{Huset som bostad}
% Det är så här stort och har så här många rum, så här högt i tak o.s.v. 

% De olika gränsytornas material och uppbyggnad.
\subsubsection{Väggarna}

\subsubsection{Taket}

\subsubsection{Grunden}


\subsubsection{Huset byggandstil och histora}
% Hur tänkte de när de byggde och renoverade huset?
% Vad ville de uppnå och vilka regler och normer hade man att hålla sig till?

\subsubsection{Byggnadsfysikens betydelse}
% Att huset är byggt så här vad betyder det för hur huset påverkas och hur huset är att bo i?

\subsubsection{Uppvärmning}
% Kortfattat vad finns det för värmeförsörjningssystem idag?
% Hur fungerar det? Varför valde man det?

\subsubsection{Ventilation}
% Hur fungerar ventilationen?
\section{Definition av väder för tillämpningar i denna rapport}
\label{subsec_weather}
Vår uppdragsgivare vill undersöka hur inomhustemperaturen påverkas av vädret. Tesen är att man kan få en mer korrekt styrning av inomhustemperaturen om man inte bara låter den påverkas av utomhustemperaturen, utan även av fler väderparamterar. Han har därför installerat en väderstation, se avnitt~ \ref{subsec_weathertransmitter}.

Begreppet väders vardagliga användningsområde är mycket brett och behöver därför avgränsas för att definiera de väderparameterar som behandlas inom projektet.

Kortfattat definieras vädret som det väderstation tillsammans med solintensitetsmätaren mäter. Så som finns beskrivet i avsnitt~\ref{subsec_weathertransmitter} mäter vi vädret med utrustning som tar in vindens hastighet och riktning, lufttemperaturen, lufttryck, relativ fuktighet samt regn och hagels varaktighet och intensitet. Vi kommer dock att bortse helt ifrån hagel då detta sker så sällan och i så korta perioder att det kan antas försumbart.  Dessutom mäter solintensitetsmätaren solens intensitet och varaktighet, se avsnitt~\ref{subsec:sunmeter}. % Källa på hur ofta (sällan) det haglar i Sverige. 

Tanken är att man ska kunna beskriva allt väder som en temperatur, antingen som den utomhustemperatur man bör reglera efter eller som den inomhustemperatur huset skulle få med befintlig aktivitet men utan uppvärmning, alltså ett mått på hur många grader man måste värma. Dessa två mått kallas ekvivalent temperatur och free-running temperature vilka beskrivs i avsnitt~\ref{sec:ekv_temp} respektive avsnitt~\ref{sec:freerunningtemp}. 

Från studier med hjälp av beräkningstjänsten Wolfram Alpha\cite{wolframalpha} av hur luftfuktighet kan påverkar luftens värmeledningsförmåga, får vi att den har väldigt liten betydelse. Man kan se en liten skillnad vid mycket höga luftfuktigheter (upp emot 90 \%) vid de högre temperaturerna, över $\unit[25]{^\circ C}$. Detta torde vara försumbart eftersom skillnaden är liten och endast vid väderförhållanden som inträffar relativt sällan i vårt klimat.

Hur regn och fukt påverkar fastighetensklimat har inte behandlats inom det här projektet.
Det kan dock antas att en hel del energi försvinner när väggen blir blöt och vattnet avdunstar. Troligen kyler regnet även luften.

Ytterligare en parameter som inte behandlas till är snö. När snön har lagt sig på taket kan man anta att den har en isolerande effekt. Vi kan inte mäta om och i så fall hur mycket snö det ligger på taket. Enligt SMHI\cite{SMHIdata}
rör det sig enbart om 25-50 dygn med snö i Göteborg per år. Detta påverkar dessutom främst de översta lägenheterna, de på vinden. Har man däremot en enplansvilla i Norrland kan man anta att detta är en mer betydande parameter, men det är alltså inget vi kommer att undersöka.

\subsection{Väderstationen}
\label{subsec_weathertransmitter}
I rapporten låter vi de parametrar som väderstationen tar in definiera vädret, se avnitt~\ref{subsec_weather}. Väderstationen som vår uppdragsgivare installerat är en Vaisala Weather Transmitter WXT520. Den mäter sju olika värden: vindens hastighet och riktning, lufttemperaturen, lufttrycket, den relativa fuktigheten samt regn och hagels varaktighet och intensitet. I tabell \ref{tbl:weathertransmitter} beskrivs stationens mätområde, noggrannhet och upplösning för de olika parametrarna. Vi har således väldigt liten nytta av att låta våra beräkningar vara noggrannare än väderstationen kan mäta.

\begin{table}[htdp]
\caption{Tekniska data för väderstationen, \cite{datasheet_weathertransmitter}}

\begin{center}
\begin{tabular}{|l | l l l|}
\hline
\textbf{Väder} & \textbf{Mätområde} % range
 & \textbf{Noggrannhet} % accuracy
 & \textbf{Upplösning} \\ % resolution
\hline
\rule{0pt}{3ex}Vindhastighet & $0$ -- $\unit[60]{m~s^{-1}}$ & $\pm3$ -- $5\%$ & $\unit[0,1]{m~s^{-1}}$ \\ 
\rule{0pt}{3ex}Vindriktning & alla riktningar & $\pm 3^{\circ}$ & $1^{\circ}$ \\
\rule{0pt}{3ex}Temperatur & $-52$ -- $\unit[+60]{^{\circ}C}$ & & $\unit[0,1]{^{\circ}C}$ \\
\rule{0pt}{3ex}Lufttryck & $600$ -- $\unit[1100]{hPa}$ & $0,5$ -- $\unit[1]{hPa}$ & $\unit[0,1]{hPa}$ \\
\rule{0pt}{3ex}Luftfuktighet & $0$ -- $\unit[100]{\%RH}$ & $\pm3$ -- $\unit[ 5]{\%RH}$ & $\unit[0,1]{\%RH}$ \\
\rule{0pt}{3ex}Regn &  & $\unit[5]{\%}$ & \unit[0,01]{mm} \\
~varaktighet & & & $\unit[10]{s}$\\
~intensitet & $\unit[0\mhyphen 200]{mm~h^{-1}}$ & & $\unit[0,1]{mm~h^{-1}}$ \\
\rule{0pt}{3ex}Hagel &  &  & 0,1 $\unit{cm^2}$ \\
~varaktighet & & från första träffen & 10 s\\
~intensitet & & & 0,1 $\unit{cm^{-2}~h^{-1}}$\\
\hline
\end{tabular}
\end{center}
\label{tbl:weathertransmitter}
\end{table}

\subsection{Solintensitetsmätaren}\label{subsec:sunmeter}
Mätaren för solintensitet som finns monterad på fastigheten är en Pyranometer CMP3 av märket Kipp \& Zonen. Den mäter våglängder från $300$ till $\unit[2800]{nm}$, vilket täcker in större delen av den solstrålning som når jorden. Ur databladet fås också att osäkerhet för en dag kan väntas vara under $\unit[10]{\%}$. Den största möjliga instrålningen den klarar av att mäta är $\unit[2000]{W m^{-2}}$ vilket är väl över maximala möjliga värde på jorden om man enbart mäter strålning från solen. Den uppfyller gott och väl behoven för studien.\cite{datasheet_sun}






\section{Definition av väder för tillämpningar i denna rapport}
\label{subsec_weather}
Vår uppdragsgivare vill undersöka hur inomhustemperaturen påverkas av vädret. Hans tes är att man kan få en mer korrekt styrning av inomhustemperaturen om man inte bara låter den påverkas av utomhustemperaturen, utan även av fler väderparamterar. Han har därför installerat en väderstation, se avsnitt \ref{subsubsec_weathertransmitter}.

Väder är något som många pratar om och alla påverkas av. Begreppets vardagliga användningsområde är mycket brett och vi ska därför avgränsa vad vi avser med termen väder. 

Kortfattat låter vi vädret definieras som det som vår väderstation mäter. Så som finns beskrivet i avsnitt \ref{subsubsec_weathertransmitter} mäter vi vädret med utrustning som tar in vindens hastighet och riktning, lufttemperaturen, lufttryck, relativ fuktighet samt regn och hagels varaktighet och intensitet. Vi kommer dock att bortse helt ifrån hagel då detta sker så sällan och i så korta perioder att det kan antas försumbart.  Dessutom finns en mätare för solen som tar in solens intensitet och varaktighet, se avsnitt \ref{subsec:sunmeter}. % Källa på hur ofta (sällan) det haglar i Sverige. % Finns det en solmätare?

Tanken är att under en vindstilla natt värms luften närmast fastigheten upp av den värme som går ut ur huset genom väggen. Ligger solen på ökas uppvärmningen av husfasaden medan om det blåser så cirkulerar luften och hinner aldrig värmas upp. Tar man hänsyn till allt detta fås en annan temperatur att räkna som utomhustemperatur än den faktiska utomhustemperaturen som väderstationen mäter upp. Denna nya temperatur kallas free-running temperature, se avsnitt \ref{sec:freerunningtemp}. %vad svenska?
% finns det parametrar som kyler så pass att vi kan få en lägre temperatur än den uppmätta?

\subsection{Regn och luftfuktighet}


\subsection{Annat väder}
En parameter som vi inte tar hänsyn till är snö. När snön har lagt sig på taket kan man anta att den har en isolerande effekt. Vi kan inte mäta om och i så fall hur mycket snö det ligger på taket. Enligt SMHI %(http://www.smhi.se/klimatdata/meteorologi/sno/Normalt-antal-dygn-med-snotacke-per-ar-1.7937)
rör det sig enbart om 25-50 dygn med snö i Göteborg per år. Detta påverkar dessutom främst de översta lägenheterna, de på vinden. Har man däremot en enplansvilla i Norrland kan man anta att detta är en mer betydande parameter, men det är alltså inget vi kommer att undersöka.

\subsection{Väderstationen}
\label{subsec_weathertransmitter}
I rapporten låter vi de parametrar som väderstationen tar in definiera vädret, se avsnitt \ref{subsec_weather}. Väderstationen som vår uppdragsgivare installerat är en Vaisala Weather Transmitter WXT520. Den mäter sju olika värden: vindens hastighet och riktning, lufttemperaturen, lufttryck, relativ fuktighet samt regn och hagels varaktighet och intensitet. I tabell \ref{tbl:weathertransmitter} kan vi se stationens mätområde, noggrannhet och upplösning för de olika parametrarna. Vi har således väldigt liten nytta av att låta våra beräkningar vara noggrannare än väderstationen kan mäta.

\begin{table}[htdp]
\caption{Tekniska data för väderstationen, \cite{datasheet_weathertransmitter}}
% Pärmen flik 1 eller www.vaisala.com/Vaisala%20Documents/Brochures%20and%20Datasheets/WXT520-Datasheet-B210417EN-H-LOW-v1.pdf (12 mars 2012)
\begin{center}
\begin{tabular}{|l | l l l|}
\hline
\textbf{Väder} & \textbf{Mätområde} % range
 & \textbf{Noggrannhet} % accuracy
 & \textbf{Upplösning} \\ % resolution
\hline
\rule{0pt}{3ex}Vindhastighet & $\unit[0\mhyphen 60]{m/s}$ & $\pm 3\mhyphen 5\%$ & $\unit[0,1]{m/s}$ \\ 
\rule{0pt}{3ex}Vindriktning & alla riktningar & $\pm 3^{\circ}$ & $1^{\circ}$ \\
\rule{0pt}{3ex}Temperatur & $\unit[-52\mhyphen +60]{^{\circ}C}$ & & $\unit[0,1]{^{\circ}C}$ \\
\rule{0pt}{3ex}Lufttryck & $\unit[600\mhyphen 1100]{hPa}$ & $\unit[0,5\mhyphen 1]{hPa}$ & $\unit[0,1]{hPa}$ \\
\rule{0pt}{3ex}Luftfuktighet & $\unit[0\mhyphen 100]{\%RH}$ & $\unit[\pm3\mhyphen 5]{\%RH}$ & $\unit[0,1]{\%RH}$ \\
\rule{0pt}{3ex}Regn &  & $\unit[5]{\%}$ & \unit[0,01]{mm} \\
~varaktighet & & & $\unit[10]{s}$\\
~intensitet & $\unit[0\mhyphen 200]{mm/h}$ & & $\unit[0,1]{mm/h}$ \\
\rule{0pt}{3ex}Hagel &  &  & 0,1 träffar/$\unit{cm^2}$ \\
~varaktighet & & från första träffen & 10 s\\
~intensitet & & & 0,1 träffar/$\unit{cm^2h}$\\
\hline
\end{tabular}
\end{center}
\label{tbl:weathertransmitter}
\end{table}

\subsection{Solintensitetsmätaren}
\label{subsec:sunmeter}
Mätaren för solintensitet som finns monterad på fastigheten är en Pyranometer CMP3 av märket Kipp & Zonen. Den mäter våglängder från 300 till 2800 nm, vilket täcker in större delen av den solstrålning som når jorden. Ur databladet fås också att en dag osäkerhet på under 10\% är att vänta. Den maximala instrålningen den klarar av att mäta är $\unit[2000]{W/m^2}$ vilket är väl över maximala möjliga värde på jorden om man enbart mäter strålning från solen. Den uppfyller gott och väl behoven för studien.\cite{datasheet_sun}




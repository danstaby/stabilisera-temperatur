\section{Definition av väder för tillämpningar i denna rapport}
\label{subsec_weather}
Vår uppdragsgivare vill undersöka hur inomhustemperaturen påverkas av vädret. Tesen är att man kan få en mer korrekt styrning av inomhustemperaturen om man inte bara låter den påverkas av utomhustemperaturen, utan även av fler väderparamterar. Han har därför installerat en väderstation, se avnitt~ \ref{subsec_weathertransmitter}.

Begreppet väders vardagliga användningsområde är mycket brett och behöver därför avgränsas för att definiera de väderparameterar som behandlas inom projektet.

Kortfattat definieras vädret som det väderstation tillsammans med solintensitetsmätaren mäter. Så som finns beskrivet i avsnitt~\ref{subsec_weathertransmitter} mäter vi vädret med utrustning som tar in vindens hastighet och riktning, lufttemperaturen, lufttryck, relativ fuktighet samt regn och hagels varaktighet och intensitet. Vi kommer dock att bortse helt ifrån hagel då detta sker så sällan och i så korta perioder att det kan antas försumbart.  Dessutom mäter solintensitetsmätaren solens intensitet och varaktighet, se avsnitt~\ref{subsec:sunmeter}. % Källa på hur ofta (sällan) det haglar i Sverige. 

Tanken är att man ska kunna beskriva allt väder som en temperatur, antingen som den utomhustemperatur man bör reglera efter eller som den inomhustemperatur huset skulle få med befintlig aktivitet men utan uppvärmning, alltså ett mått på hur många grader man måste värma. Dessa två mått kallas ekvivalent temperatur och free-running temperature vilka beskrivs i avsnitt~\ref{sec:ekv_temp} respektive avsnitt~\ref{sec:freerunningtemp}. 

Från studier med hjälp av beräkningstjänsten Wolfram Alpha\cite{wolframalpha} av hur luftfuktighet kan påverkar luftens värmeledningsförmåga, får vi att den har väldigt liten betydelse. Man kan se en liten skillnad vid mycket höga luftfuktigheter (upp emot 90 \%) vid de högre temperaturerna, över $\unit[25]{^\circ C}$. Detta torde vara försumbart eftersom skillnaden är liten och endast vid väderförhållanden som inträffar relativt sällan i vårt klimat.

Hur regn och fukt påverkar fastighetensklimat har inte behandlats inom det här projektet.
Det kan dock antas att en hel del energi försvinner när väggen blir blöt och vattnet avdunstar. Troligen kyler regnet även luften.

Ytterligare en parameter som inte behandlas till är snö. När snön har lagt sig på taket kan man anta att den har en isolerande effekt. Vi kan inte mäta om och i så fall hur mycket snö det ligger på taket. Enligt SMHI\cite{SMHIdata}
rör det sig enbart om 25-50 dygn med snö i Göteborg per år. Detta påverkar dessutom främst de översta lägenheterna, de på vinden. Har man däremot en enplansvilla i Norrland kan man anta att detta är en mer betydande parameter, men det är alltså inget vi kommer att undersöka.

\subsection{Väderstationen}
\label{subsec_weathertransmitter}
I rapporten låter vi de parametrar som väderstationen tar in definiera vädret, se avnitt~\ref{subsec_weather}. Väderstationen som vår uppdragsgivare installerat är en Vaisala Weather Transmitter WXT520. Den mäter sju olika värden: vindens hastighet och riktning, lufttemperaturen, lufttrycket, den relativa fuktigheten samt regn och hagels varaktighet och intensitet. I tabell \ref{tbl:weathertransmitter} beskrivs stationens mätområde, noggrannhet och upplösning för de olika parametrarna. Vi har således väldigt liten nytta av att låta våra beräkningar vara noggrannare än väderstationen kan mäta.

\begin{table}[htdp]
\caption{Tekniska data för väderstationen, \cite{datasheet_weathertransmitter}}

\begin{center}
\begin{tabular}{|l | l l l|}
\hline
\textbf{Väder} & \textbf{Mätområde} % range
 & \textbf{Noggrannhet} % accuracy
 & \textbf{Upplösning} \\ % resolution
\hline
\rule{0pt}{3ex}Vindhastighet & $0$ -- $\unit[60]{m~s^{-1}}$ & $\pm3$ -- $5\%$ & $\unit[0,1]{m~s^{-1}}$ \\ 
\rule{0pt}{3ex}Vindriktning & alla riktningar & $\pm 3^{\circ}$ & $1^{\circ}$ \\
\rule{0pt}{3ex}Temperatur & $-52$ -- $\unit[+60]{^{\circ}C}$ & & $\unit[0,1]{^{\circ}C}$ \\
\rule{0pt}{3ex}Lufttryck & $600$ -- $\unit[1100]{hPa}$ & $0,5$ -- $\unit[1]{hPa}$ & $\unit[0,1]{hPa}$ \\
\rule{0pt}{3ex}Luftfuktighet & $0$ -- $\unit[100]{\%RH}$ & $\pm3$ -- $\unit[ 5]{\%RH}$ & $\unit[0,1]{\%RH}$ \\
\rule{0pt}{3ex}Regn &  & $\unit[5]{\%}$ & \unit[0,01]{mm} \\
~varaktighet & & & $\unit[10]{s}$\\
~intensitet & $\unit[0\mhyphen 200]{mm~h^{-1}}$ & & $\unit[0,1]{mm~h^{-1}}$ \\
\rule{0pt}{3ex}Hagel &  &  & 0,1 $\unit{cm^2}$ \\
~varaktighet & & från första träffen & 10 s\\
~intensitet & & & 0,1 $\unit{cm^2~h^{-1}}$\\
\hline
\end{tabular}
\end{center}
\label{tbl:weathertransmitter}
\end{table}

\subsection{Solintensitetsmätaren}\label{subsec:sunmeter}
Mätaren för solintensitet som finns monterad på fastigheten är en Pyranometer CMP3 av märket Kipp \& Zonen. Den mäter våglängder från $300$ till $\unit[2800]{nm}$, vilket täcker in större delen av den solstrålning som når jorden. Ur databladet fås också att osäkerhet för en dag kan väntas vara under $\unit[10]{\%}$. Den största möjliga instrålningen den klarar av att mäta är $\unit[2000]{W m^{-2}}$ vilket är väl över maximala möjliga värde på jorden om man enbart mäter strålning från solen. Den uppfyller gott och väl behoven för studien.\cite{datasheet_sun}




\subsection{Väder}

Väder är något som många pratar om och alla påverkas av. Begreppets vardagliga användningsområde är mycket brett och vi ska därför avgränsa vad vi avser med termen väder. 

Kortfattat låter vi vädret definieras som det som vår väderstation mäter. Så som finns beskrivet i avsnitt \ref{subsec_weatertransmitter} mäter vi vädret med utrustning som tar in vindens hastighet och riktning, lufttemperaturen, lufttryck, relativ fuktighet samt regn och hagels varaktighet och intensitet. Vi kommer dock att bortse helt ifrån hagel då detta sker så sällan och i så korta perioder att det kan antas försumbart.

Kvar har vi då vinden, temperaturen, trycket, fuktigheten och regnet.

% Hur mycket av det ska vi använda? Det kanske beror på hur det går?

\subsubsection{Regn}
Bekräfta att:
- Då det regnar är det 100\% luftfuktighet.
- Övre gräns: ingen isolerande luft. ref-temp = randtemp.
Det ger en uppskattning på hur mycket extra värme som försvinner vid regn. 

% Ungefär vilka temperatursvängningar, nederbörd o.s.v. är det i området.

Snö: isolerar. Men vi har ingen mätutrustning.
\section{Dokumentets disposition}

För att göra beräkningar av energiflödena genom fastigheten använder vi oss av några olika beräkningsmodeller. Dessa leder sedan via våra metoder fram till resultaten. Nedan återfinns en beskrivning av hur detta går till.

Efter en inledning med beskrivning av utgångspunkterna för arbetet kommer en del med delvis ganska tung teori. Där beskrivs de olika sätt värme kan överföras på – konvektion, ledning och strålning – och hur man kan räkna på dem i våra applikationer på energiflödena genom en fastighet. Här beskrivs kort hur de olika delarna av teorin knyter an till våra metoder och resultat. Det är inte nödvändigt för läsaren att tillgodogöra sig all teori för att kunna ta del av resultatet i rapporten. Vi hoppas ändå att det ska kunna visa på att våra resultat står på en gedigen grund.

Strålningen, både från solen och från omgivningen när den har en annan temperatur än fastigheten, beskrivs som 
svartkroppsstrålning. Vi har speciellt tagit upp när solen skiner in genom ett fönster och då 
modellerat det hela relativt enkelt från solsystemets geometri. På så sätt får vi veta hur 
mycket inomhusluften och väggen värms upp vid soligt väder.
För värmeledning och konvektion är den främsta beräkningsmetoden finita 
elementmetoden. För att räkna på konvektionen använder man Navier-Stokes ekvationer 
som modellerar luftflöden. De används även av beräkningsprogrammet \emph{Comsol Multiphysics} när vi 
modellerar luftflödena i ofrivillig ventilation. Det ekvationssystem som fås ur finita
 elementmetoden optimeras sedan med Newton-Raphsons metod.

De olika energiflödena kan med utgångspunkt i dessa metoder beräknas vid olika väderförhållanden. Läggs de 
sedan samman fås en bild av hur mycket energi som måste tillföras fastigheten för att 
bibehålla en konstant inomhustemperatur.

De begrepp som sammanfattar energiflödena är free-running temperature – som visar på 
vilken temperatur fastigheten skulle ha om man nyttjade den som idag men utan 
uppvärmnopen ing; och ekvivalent temperatur – som sammanfattar vädret i en ny utomhustemperatur. Dessa båda blir givetvis olika för olika väderförhållanden.

Vår uppdelning ger att manoop kan vi se var byggnadens största energitjuvar finns och 
olika energibesparande åtgärder kan vägas mot varandra. Om det läcker mest energi ur en del av byggnaden är det inte optimalt att åtgärda en annan del där det läcker betydligt mindre. Vi kan också se vilka väderförhållanden som påverkar energiflödet mest. Exempelvis om ett väderförhållande som stjäl mycket energi inträffar relativt sällan tjänar man kanske mindre på att åtgärda det än ett som stjäl mindre energi men inträffar väldigt ofta.

Slutligen diskuterar vi var husets största problem ur energisynpunkt finns och hur lämpliga olika åtgärder är.

\section{Finita elementmetoden}

Inom fysiken modelleras ofta fysikaliska fenomen med partiella differentialekvationer.
Dessa kan ofta vara icketriviala att evaluera analytiskt och exakt. Det finns en
rad olika metoder för att hantera detta problem genom approximationer. En av
dessa metoder kallas finita elementmetoden. Denna går ut på att den
partiella differentialekvationen eller systemet av kopplade partiella
differentialekvationer delas upp i små delar som är mer lätthanterliga.
Denna metod tillåter också evalueringen av problem vars egenskaper i
definitionsmängden är ickehomogena och diskontinuerliga. Dessutom kan
problemen lösas för godtycklig geometri med godtyckliga randvillkor. Alla
dessa egenskaper gör att finita elementmetoden är ett väldigt kraftfult redskap
vid beräkning av svårtacklade problem. Denna metod kommer senare i arbetet
att användas för att räkna på värme och luftflöden.

Den undermetod av finita elementmetoden som skall användas i denna
kandidatuppsats är Galerkins metod.
En finita elementlösning med Galerkins metod kräver att problemet reduceras till
ett ekvivalent variationsproblem.
Här är $L$ differentialoperatorn
som betecknar differentialekvationen i \eqref{eq:femtheoryprob}.
Här är $u = u(\mathbf{r},t)$ en funktion av de spatiala koordinaterna samt tiden.

\begin{equation}
\label{eq:femtheoryprob}
L(u) = 0
\end{equation}

\noindent
$\Phi$ definieras sedan som rummet av alla testfunktioner $\phi$ vilka
 är kontinuerliga i
definitionsmängden $\Omega$ samt vars derivator är bitvis kontinuerliga på randen
$\Gamma$. Testfunktionerna $\phi \in \Phi$ måste även vara $L^2$ integrabla.
Variationsproblemet blir då: Sök $u\in\Phi$
som uppfyller ekvation \eqref{eq:femtheoryweak}. Det är tydligt
att alla $u$ som uppfyller $L(u) = 0$ är ortogonala mot alla
$L^2$ integrabla testfunktioner ty integranden kommer då alltid vara identiskt noll.

\begin{equation}
\label{eq:femtheoryweak}
\int_\Omega L(u)\phi(\mathbf{r}) d\Omega = 0
\end{equation}

\noindent
Om differentialekvationen innehåller andra ordningens deriveringsoperatorer behövs
det lite analys. Vissa testfunktinoner som ibland används klarar ej att deriveras
två gånger. Dessutom så ger kommande beräkningar oss en trevlig metod syftad
till att implementera Neumannvillkor (se sektion \ref{subsec:boundaryenforcement}).

Divergensteoremet i två dimensioner \eqref{eq:femtheorygausstheorem}
med $\mathbf{A} = (uw, 0)$ eller $\mathbf{A} = (0, uw)$ ger då ekvation
\eqref{eq:femtheory:partint}. Slutligen summeras uttrycket över alla
$k$ för att få \eqref{eq:femtheory:green}. Här är $w$ utbytt mot $w_i$ med
påståendet att $\mathbf{w} = (w_1, w_2) = \nabla v$. \cite{johnson2009}

\begin{equation}
\label{eq:femtheorygausstheorem}
\int_\Omega \nabla\cdot \mathbf{A} d\Omega = \int_\Gamma \mathbf{A}\cdot\mathbf{n} d\Gamma
\end{equation}

\begin{equation}
\label{eq:femtheory:partint}
\int_\Omega w\frac{\partial u}{\partial x_k} d\Omega +
\int_\Omega u\frac{\partial w}{\partial x_k} d\Omega =
\int_\Gamma uwn_k d\Gamma\mbox{,   }k=1,2
\end{equation}

\begin{equation}
\label{eq:femtheory:green}
\int_\Omega u\Delta v d\Omega =
\int_\Gamma u\nabla v\cdot\mathbf{n}d\Gamma-\int_\Omega \nabla v\cdot\nabla u d\Omega
\end{equation}

\noindent
Termer som innehåller laplaceoperatatorn kan enligt ovanstående recept utvecklas
till ekvation \eqref{eq:femtheorylaplacerecepie}.

\begin{equation}
\label{eq:femtheorylaplacerecepie}
\int_\Omega \phi\Delta u d\Omega = -\int_\Omega \nabla\phi\nabla u d\Omega +
\int_\Gamma \phi \mathbf{n}\cdot\nabla u d\Gamma
\end{equation} 

\noindent
Slutligen så ansätts ekvation \eqref{eq:femtheoryansatz} vilket stoppas
in i den saga formuleringen i ekvation \eqref{eq:femtheoryweak}. Nu kan
konstanterna i linjärkombinationen av testfunktioner plockas ut ur integralerna
och integralerna av testfunktionerna evalueras. Genom detta så kan
sedan konstanterna lösas ut genom lösning av ekvationssystemet som
fås då man låter den fria testfunktionen löpa över alla okända noder.
För exempel på dessa beräkningar se avsnitten \ref{sec:femconvection} och
\ref{sec:femheat}.

\begin{equation}
\label{eq:femtheoryansatz}
u(\mathbf{r}) \sum_n \approx u_n \phi_n(\mathbf{r})
\end{equation}


\subsection{Spatial diskretisering}

Lösningen av de galerkinformulerade ekvationerna kräver att en basfunktion väljs.
Ett i populärt val i litteraturen är linjära triangulära element på grund av att de
är lätta att använda. Det är dock bara fantasin som begränsar valet av basfunktion
och elementgeometri.
\cite{johnson2009}\cite{lewis04}\cite{reddy93}\cite{fem50} Området den
partiella differentialekvationen skall lösas i delas upp i triangulära element.
Trianglarnas hörn blir då noder. Elementen uppför sig genom att i en basfunktion
är funktionen ett för noden som har samma nummer som basfunktionen. Denna funktion
antar sedan värdet noll på alla motstående kanter på alla trianglar som innehar noden
med samma nummer som basfunktionen. Mellan dessa positioner bildar basfunktionen
linjära plan på det sätt att basfunktionen blir kontinuerlig med en bitvis
kontinuerlig derivata.
Utanför dessa trianglar är basfunktionen noll. \cite{johnson2009}

\subsection{Påtvingande av randvillkor}
\label{subsec:boundaryenforcement}
Två typer av randvillkor är relevanta för detta arbete främst i lösning med Galerkins 
metod. Det ena är dirichletvillkor som
innebär att sökt funktion antar ett visst värde på randen. Det andra är neumannvillkor
vilket sätter derivatan i randnormalriktningen till ett värde. Tillvägagångssättet
för påtvingande av dessa villkor i Galerkins metod är lite olika. Dirichletvillkor
påtvingas lättast genom att noderna som innehar denna typ av villkor sätts till ett värde
och att ekvationerna uppdateras med detta värde. Neumannvillkor påtvingas istället
genom att värdet på dessa rander sätts in i integralerna som går över randen. Det
är viktigt att tillräckligt många randvillkor sätts. Om inte detta genomförs
så får problemet en lösning vars egenskaper inte är explicit bestämda eller
i värsta fall en trivial lösning. Vilka dessa randvillkor är beror helt
på applikationen.

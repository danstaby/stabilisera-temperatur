\section{Kvantifiering av konstanta energiflöden}

Flera av energiflödena genom fastigheten är relativt konstanta sett till en längre tidsperiod. Det gäller främst värme från elektriska apparater, så som kylskåp och datorer, människors kroppsvärme och varmvattencirkulation. På så sätt kan aktivt tillförd energi, det vill säga den energitillförsel som kan regleras och tillförs via radiatorerna, enkelt regleras med hänsyn till dessa, om man bara känner dess storlek.

\subsection{Uppvärmning från människor}
Den utstrålade kroppsvärmen från människor kan beräknas genom att de antas vara svartkroppar. Stefan-Boltzmanns lag säger då att utstrålade energi per yt- och tidsenhet är $j=\sigma T^4$, där $T$ är temperaturen och $\sigma=\unit[5.6705\cdot 10^8]{Wm^{-2}K^{-4}}$ \cite{physicshandbook}, se teoriavsnitt \ref{sec:blackbody}. På samma sätt beräknas den energi som strålas in mot kroppen från omgivningen. Nettostrålningen från en människa kan då ses i ekvation \eqref{eq:constantsources:stefan} där $T_k=37^{\circ}C=310K$ är kroppstemperaturen och $T_r=20^{\circ}C=293K$ är rumstemperaturen. Multipliceras det med en människas area, ungefär $\unit[2]{m^2}$, fås en nettoeffekt på $\unit[211]{W}$. I själva verket reduceras denna effekt av en rad faktorer. Exempelvis är hudens temperatur lägre än kroppstemperaturen samtidigt som klädesplagg reducerar effektutstrålningen något. Professor Göran Grimvall skriver i NyTeknik att en rimlig nettoeffekt vid vila är cirka 1 W per kilogram kroppsvikt, alltså ungefär $\unit[50-100]{W}$\cite{Grimvall}.

\begin{equation}
\label{eq:constantsources:stefan}
j=\sigma \left( T_k^4 - T_r^4 \right)
\end{equation}
\noindent

\subsection{Varmvattencirkulation i fastigheten}
I huset cirkulerar hela tiden varmvattnet för att alltid kunna tillgodose de boendes behov av varmvatten utan dröjsmål. Efter en tur i systemet sjunker temperaturen på varmvattnet med tre grader och flödet är $\unit[800]{l/h}$. Detta motsvarar en energitillförsel till fastigheten på $\unit[2,8]{kW}$.

\subsection{Energi från elektrisk apparatur}
Den största delen av energin som driver en elektrisk apparat blir till värme. Därför låter vi energiflödet från elektriska apparatur motsvaras av fastighetens energiförbrukning, vilken kan läsas av kontinuerligt och på så sätt bli en del av reglersystemet.


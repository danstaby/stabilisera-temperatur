\section{Konstanta energiflöden}

En del av energiflödena är relativt konstanta sett till en längre tidsperiod. Dessa är främst värme från elektriska apparater, så som kylskåp och datorer, människors kroppsvärme och varmvattencirkulation. Även kylningen från husets grund anses vara konstant.

Väldigt liten del av den energi som förbrukas i en elektrisk apparat blir till något annat än värme. Därför låter vi det energiflödet motsvaras av fastighetens energiförbrukning. Den kan läsas av kontinuerligt och på så sätt reglera aktivt tillförd energi\footnote{Aktivt tillförd energi: Den energitillförsel som kan regleras och tillförs via radiatorerna.}.

Människornas utstrålade kroppsvärme kan beräknas genom att de antas vara svartkroppar. Stefan-Boltzmanns lag säger då att utstrålade energi per yt- och tidsenhet är $j=\sigma T^4$, där $T$ är temperaturen och $\sigma=\unit[5.6705\cdot 10^8]{Wm^{-2}K^{-4}}$ \cite{physicshandbook}. På samma sätt beräknas den energi som strålas in mot kroppen från omgivningen. Nettostrålningen från en människa kan då ses i ekvation \eqref{eq:constantsources:stefan} där $T_k=37^{\circ}C=310K$ är kroppstemperaturen och $T_r=20^{\circ}C=293K$ är rumstemperaturen. Multiplicerat med en människas area, ungefär $\unit[2]{m^2}$, fås en nettoeffekt på $\unit[211]{W}$. I själva verket reduceras denna effekt av en rad faktorer. Exempelvis är hudens temperatur lägre än kroppstemperaturen samtidigt som klädesplagg reducerar effektutstrålningen något. Det är allmänt vedertaget att man kan sätta nettoeffekten till ungefär $\unit[50-100]{W}$.

\begin{equation}
\label{eq:constantsources:stefan}
j=\sigma \left( T_k^4 - T_r^4 \right)
\end{equation}
\noindent
I huset cirkulerar hela tiden varmvattnet för att alltid kunna tillgodose de boendes behov av varmvatten utan dröjsmål. Efter en tur i systemet sjunker temperaturen på varmvattenet med tre grader. Detta motsvarar en energitillförsel till fastigheten på \textcolor{red}{??} $\unit[]{W}$.

Grunden huset står på har en randtemperatur på ungefär $6^{\circ}\mbox{C}$ från marken. Den kan antas vara konstant då huset grund ligger en bit ned under marken. Enligt SLU, Sveriges Lantbruksuniversitet, kan marktemperaturen antas vara konstant från $\unit[1,6]{m}$ under markytan \cite{SLU}. Deras data kommer från Uppsala vilket borde vara jämförbart med Göteborg. Huset står på en grund som består av ett luftgap och sedan $\unit[0,25]{m}$ betong. Längst ned i huset finns en källare som är ouppvärmd \textcolor{red}{??}. Det energiläckage som då blir genom husets grund ned i marken kan beräknas uppgå till \textcolor{red}{??} $\unit[]{W}$.

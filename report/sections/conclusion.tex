\chapter{Slutsats}

%Sammanfattning av resultat

I detta arbete har olika väderparametrars effekt på inomhusklimat och energiåtgång studerats.
Det har framgått att både solinstrålning och infiltrationsförluster på grund av ofrivillig
ventilation kan leda till stora energibidrag. Inom arbetet har även olika förbättringsåtgärder
studerats. Det har framkommit att mängden sparad energi för tillägsisolering för vöst och sydväggen
kan uppgå mot $\unit[23]{\%}$ i extremfallet. Besparingen för andra åtgärder som att ta hänsyn till solsken
kan ge en besparing på $\unit[50]{\%}$ över ett soligt dygn.  Nu är självklart inte alla dagar soliga
så den faktiska energibesparingen kommer bli lägre under hela eldningssäsongen.
Att ta hänsyn till vinden kommer att stabilisera
temperaturen dock så har denna väderparameter en kylande effekt vilket inte nödvändigtvis leder
till en lägre energiförbrukning. Detta då inomhustemperaturen sjunker under önskvärd temperatur. 

Det kan också konstateras att de olika väggarna har extremt olika tidskonstanter vilket man måste ta hänsyn till vid regleringen med olika reglersystem för olika delar i byggnaden.

\emph{\color{red} Lägg in lite ekonomiska siffror och jämförelser för andra besparande åtgärder.}

%Bevara frågeställningar
Det har framkommit att det är en god idé att ta hänsyn till olika väderparametrar då
detta kan ge en ganska markat besparing under vissa förutsättningar. Självklart är
besparingen olika beroende på fastighetens egenskaper och fastighetens geografiska position.

%Vad kunde vi inte besvara
Det var svårt att med någon större noggrannhet kvantifiera alla energiflödens storlekar. Vi
anser dock att våra approximationer ger en god idé angående hur stora energiflödena är i jämförelse
med varandra. För att veta exakta siffror skulle det vara tvunget att implementera de olika
förbättringarna i fastigheten och mäta hur mycket energi som sparats. Alternativt skulle
bättre modeller kunna ge en exaktare bild. Problemet med detta är att modeller tenderar att
bli väldigt beräkningsintensiva och komplexa.


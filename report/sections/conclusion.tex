\chapter{Slutsats}

%Sammanfattning av resultat

I detta projekt har olika väderparametrars effekt på inomhusklimat och energiåtgången i en fastighet studerats.
Det har framgått att både solinstrålning och infiltrationsförluster på grund av ofrivillig
ventilation kan leda till stora energiflöden. Inom arbetet har även olika förbättringsåtgärder
studerats. Det har visat sig att man genom att dra nytta av solens värmande effekt kan få 17~\% lägre energiåtgång över ett soligt dygn, vilket motsvarar 4~\% lägre energiåtgågng över alla dygn, både klara och molnig, under hela eldningsäsningen. Detta kan jämföras med mängden sparad energi för tilläggsisolering för syd- och västväggarna som uppgår till 17~\% av dagens energiåtgång. Det kan också konstateras att de olika väggarna har olika tidskonstanter vilket man måste ta hänsyn till vid regleringen genom att ha olika reglersystem för olika delar i byggnaden. Denna effekt skulle dock minska något med en tilläggsisolering av syd- och västväggarna.

Att i regleringen ta hänsyn till vinden kommer inte nödvändigtvis att leda till en lägre energiförbrukning då vinden har en kylande effekt. När det blåser sjunker inomhustemperaturen under önskad temperatur med dagens system. Visserligen finns det en risk att man idag överkompenserar när det inte blåser, eftersom det är svårt att veta när det ska blåsa, och givetvis kan man då spara genom att bara ha ett högre energiinflöde när det faktiskt blåser. Att ta hänsyn till vinden leder alltid till en stabilare inomhustemperatur eftersom man eliminerar vindens temperatursänkning.

%Ekonomiska aspekter
Energibesparande åtgärder utöver den direkta styrningen kan också genomföras. För att gå vidare i processen antingen med termostater till alla radiatorer eller tilläggsisolering av de idag inte isolerade fasaderna behöver föreningen utarbeta en plan för hur de vill att fastigheten ska vara rustad i framtiden. Vidare behöver man också utreda när fasaden behöver renoveras och om man får tilläggsisolera på utsidan med resultatet att det blir en puts- istället för en tegelfasasd.

Att montera termostater på elementen är billigare än att tilläggsisolera fasaden. Enligt våra beräkningar och information från leverantörer ger det även en större energibesparing. En annan aspekt är att det tillför mervärde för de boende. De boende vill troligtvis kunna reglera temperaturen själva, och vid försäljning av lägenheter bör det kunna öka dess värde.

I samband med en installation av ett reglersystem bör även elektriska termostater installeras för att minimera tidskonstanterna och för att ge de boende möjligheten att själva bestämma vilken temperatur de vill ha hemma.

%Bevara frågeställningar
Det det har visat sig vara en god idé att ta hänsyn till olika väderparametrar då
detta kan ge en markant besparing under rätt förutsättningar. Självklart är
besparingen olika beroende på fastighetens egenskaper och fastighetens geografiska läge.

%Vad kunde vi inte besvara
Det var svårt att med någon större noggrannhet kvantifiera storleken på de olika energiflödena. Vi
anser dock att våra approximationer ger en god uppfattning om hur stora energiflödena är i förhållande
till varandra. För att få veta exakta värden skulle det vara nödvändigt att implementera de olika
förbättringarna i fastigheten och mäta hur mycket energi som sparats. Alternativt skulle
bättre modeller kunna ge en exaktare bild. Problemet med det är att mer komplexa modeller tenderar att
bli mer beräkningsintensiva.


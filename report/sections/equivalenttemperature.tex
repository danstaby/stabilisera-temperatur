\section{Ekvivalent temperatur}

Kort sagt är ekvivalent temperatur det värde man ersätter utomhustemperaturen för att ta 
hänsyn till fler väderparametrar än just utomhustemperaturen i styrningen av en 
klimatanläggning.

Att bara ta hänsyn till utomhustemperaturen vid injustering av klimatsystem är enkelt. 
Tyvärr är det lite för enkelt för att det ska bli riktigt bra, eftersom flera andra 
väderparametrar, främst sol och vind, värmer och kyler fastigheten i olika grad. I resultatdelen i den här rapporten visas det mer exakt hur mycket.

De energiflöden i väggen som orsakas av väderleken summeras med energiflödet 
som uppkommer av utomhustemperaturen. Dessa jämförs sedan med energiflöden från 
olika utomhustemperaturer, utan övrig väderpåverkan, och den temperatur som ger 
samma energiflöde kallas ekvivalent temperatur. Denna kan sedan ersätta 
utomhustemperaturen som indata till klimatsystemet och ge ett jämnare inomhusklimat.

% är det bra
% varför används det


% Tidigare text:
%Antag att det bara finns ett sorts väder, där temperaturen är den enda variabeln. Det
% skulle innebära att man kan hänföra hur mycket energi som går åt för att värma upp 
% någonting direkt till utetemperaturen. Det finns oändligt antal olika vädertyper, och fler 
% parametrar måste tas i beaktning då man räknar ut hur mycket energi som måste tillföras
%  huset. En ekvivalent temperatur för en viss vädertyp skulle således motsvara den 
%  temperaturen, i ett optimalt klimat, som kräver tillförsel av samma energimängd för att 
%  upprätthålla efterfrågat klimat.
\section{Ekvivalent temperatur  – en konceptuell beskrivning}

Ekvivalent temperatur är, precis som free-running temperature, ett begrepp som sammanfattar vädrets påverkan på en byggnad i en siffra. Kort sagt är ekvivalent temperatur det värde man ersätter utomhustemperaturen för att ta 
hänsyn till fler väderparametrar än just utomhustemperaturen i styrningen av en 
klimatanläggning.

Att bara ta hänsyn till utomhustemperaturen vid injustering av klimatsystem är enkelt. 
Tyvärr är det lite för enkelt för att det ska bli riktigt bra, eftersom flera andra 
väderparametrar, främst sol och vind, värmer och kyler fastigheten i olika grad. I resultatdelen i den här rapporten visas det mer exakt hur mycket.

De energiflöden i väggen som orsakas av väderleken summeras med energiflödet 
som uppkommer av utomhustemperaturen. Dessa jämförs sedan med energiflöden från 
olika utomhustemperaturer, utan övrig väderpåverkan, och den temperatur som ger 
samma energiflöde kallas ekvivalent temperatur. Denna kan sedan ersätta 
utomhustemperaturen som indata till klimatsystemet och ge ett jämnare inomhusklimat. 

Användingsområdet för free-running temperature och ekvivalent temperatur är det samma, det är bara två olika angreppssätt på samma problem. Nämligen att vädret spelar in på energiflödena genom en vägg i en byggnad.  Båda systemen kan enkelt implementeras i befintliga reglersystem och stora omkostnader kan därför undvikas. Skillanden är att  free-running temperature beskriver byggnadens reaktion på vädret medan ekvivalent temperatur beskriver vädret. Ekvivalent temperatur är alltså inte ett byggnadsspecifikt värde utan beror enbart av vädret. Båda varianterna är dock betydligt bättre än att bara titta på utomhustemperaturen, så som görs i många fastigheter idag.


% Tidigare text:
%Antag att det bara finns ett sorts väder, där temperaturen är den enda variabeln. Det
% skulle innebära att man kan hänföra hur mycket energi som går åt för att värma upp 
% någonting direkt till utetemperaturen. Det finns oändligt antal olika vädertyper, och fler 
% parametrar måste tas i beaktning då man räknar ut hur mycket energi som måste tillföras
%  huset. En ekvivalent temperatur för en viss vädertyp skulle således motsvara den 
%  temperaturen, i ett optimalt klimat, som kräver tillförsel av samma energimängd för att 
%  upprätthålla efterfrågat klimat.
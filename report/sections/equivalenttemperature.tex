\section{Ekvivalent temperatur  – en konceptuell beskrivning}
\label{sec:ekv_temp}

Ekvivalent temperatur är, precis som free-running temperature, ett begrepp som sammanfattar vädrets påverkan på en byggnad i ett värde. Ekvivalent temperatur beskriver energibalansen på en yta och är det värde man ersätter utomhustemperaturen med för att ta 
hänsyn till fler väderparametrar än just utomhustemperaturen i styrningen av en 
klimatanläggning.

Att bara ta hänsyn till utomhustemperaturen vid injustering av klimatsystem är enkelt. 
Tyvärr är det lite för enkelt för att det ska bli riktigt bra, eftersom flera andra 
väderparametrar, främst sol och vind, värmer och kyler fastigheten i olika grad. I resultatdelen i den här rapporten visas det mer exakt hur stora dessa bidrag är.

De energiflöden genom klimatskalet som orsakas av väderleken summeras med energiflödet som uppkommer på grund av utomhustemperaturen. Denna summa jämförs sedan med energiflödet genom klimatskalet vid olika utomhustemperaturer utan väder, det vill säga ingen sol, inget regn och ingen vind. Den temperatur som ger 
samma energiflöde som det väderleksorsakade energiflödet kallas ekvivalent temperatur för den väderleken. Denna kan sedan ersätta 
utomhustemperaturen som indata till klimatsystemet och ge ett jämnare inomhusklimat. 

Användingsområdet för free-running temperature och ekvivalent temperatur är detsamma, det är bara två olika angreppssätt på samma problem. De beskriver båda hur vädret spelar in på energiflödena genom en vägg i en byggnad.  Båda systemen kan enkelt implementeras i befintliga reglersystem och stora omkostnader kan därför undvikas. Skillnaden är att free-running temperature är temperaturen inomhus medan ekvivalent temperatur är ytterväggens temperatur. Ekvivalent temperatur tar inte hänsyn till solinstrålning genom fönster vilket är möjligt med free-running temperature. Båda varianterna är dock betydligt bättre än att bara titta på utomhustemperaturen, så det vis som görs i många fastigheter idag.

\subsection{Beräkning av ekvivalent temperatur}

Energibalansen på utsidan av en vägg beror på solinstrålning, strålning från omgivningen, konvektion med omgivande luft och värmeflöde inifrån.

Alla energiflödena kan sammanfattas med Fouriers värmelag i ekvationen 

\begin{equation}
\label{eq:walltemp}
\frac{\mathrm{d}T_\text{vägg}}{\mathrm{d}t} = 
I\cdot \alpha_\text{abs} + I_\text{omg} + (T_\text{ute} - T_\text{vägg}) h +( T_\text{vägg} - T_\text{inne} ) U - I_\text{vind}.
\end{equation}

När sedan $\frac{\mathrm{d}T_\text{vägg}}{\mathrm{d}t}=0$ fås en jämviktslösning för väggens temperatur vid olika väder. $h$ är konvektionsparametern för luft, som varierar med vinden, se avsnitt~\ref{subsec:windconv}. I avsnitt~\ref{subsec:darcy} beskrivs $I_\text{vind}$.
$U$ är väggens U-värde, som i vårt fall är $\unit[1,2]{W~m^{-2}~{K^{-1}}}$ för den oisloerade söderväggen respektive $\unit[0,3]{W~m^{-2}~K^{-1}}$ för den isolerade norrväggen. $I_\text{omg}$ är all svartkroppsstrålning från omgivningen, som kan delas upp i strålning från atmosfären och från den övriga omgivningen, se avsnitt~\ref{sec:bb_sur}. Vid detta jämviktsläge är $T_\text{vägg}$ ekvivalent temperatur.

\begin{equation}\boxed{ \; \; \;
\label{eq:ekvtemp}
T_\text{ekv} = 
\frac{I\cdot \alpha_\text{abs} + I_\text{omg} + T_\text{ute} \cdot h - T_\text{inne} \cdot U - I_\text{vind}}{h-U}.
\; \; \;}\end{equation}

På samma sätt kan man ta hänsyn till fler effekter allt eftersom man utrett deras påverkan på byggnaden.

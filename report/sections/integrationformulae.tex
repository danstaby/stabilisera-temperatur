\section{Formler för linjära triangulära element}
\label{sec:integrationformulae}
\rhead{Appendix \ref{sec:integrationformulae}}

Vid skapande av stelhetsmatriser och lastvektorer behövs ett antal integraler
evaluras. Detta appendix syftar till att vara en formelsamling för några
nödvändiga beräkningar för linjära triangulära element. I ekvation
\eqref{eq:formula:area} beskrivs arean av en triangel. $(x,y)$ betecknar koordinaterna
av hörnen $(i,j,k)$. Ekvation \eqref{eq:formula:volume} beskriver integralen för ett
antal basfunktioner $\phi$ över en triangel $\Omega$. Liknande formulering för två dimensioner
återfinns i ekvation \eqref{eq:formula:rand}. I denna betecknas linjen mellan kanterna
$i$ och $j$ som $\Gamma$. \cite{lewis04}

Slutligen förekommer det även derivator av basfunktionerna. Alla derivator i en triangel
kan med lätthet beräknas genom att använda en determinant enligt ekvation
\eqref{eq:formula:derivative}. Här löper $l$ över $l=i,j,k$ och $\mathbf{r} = (x,y)$ som är
de rumsliga koordinaterna. \cite{fem50}

\begin{align}
\label{eq:formula:area}
A &=
\frac{1}{2}
\begin{vmatrix}
1 & 1 & 1 \\
x_i & x_j & x_k \\
y_i & y_j & y_K
\end{vmatrix} \\
\label{eq:formula:volume}
\int_\Omega \phi^a_i\phi^b_j\phi^c_j d\Omega &=
\frac{a!b!c!2A}{(a+b+c+2)!} \\
\label{eq:formula:rand}
\int_\Gamma \phi^a_i \phi^b_j d\Gamma &=
\frac{a!b!l}{(a+b+1)!} \\
\label{eq:formula:derivative}
\frac{\partial \phi_l}{\partial \mathbf{r} } &=
\begin{pmatrix}
1 & 1 & 1 \\
x_i & x_j & x_k \\
y_i & y_j & y_k  
\end{pmatrix}^{-1}
\begin{pmatrix}
0 & 0 \\
1 & 0 \\
0 & 1
\end{pmatrix}
\end{align}


\subsubsection{Solstrålning genom fönster}

För att beräkna den totala effekt solstrålning tillför byggnaden behövs fönstrenas vinkelberoende g-värden (presenterat i avsnitt \ref{gvalue}). 

För att beräkna g-värdet ur \eqref{eq:radiationwindowstheory:gvalue} behöver parametern $z = \theta/90$ beräknas, där $\theta$ är vinkeln mellan solstrålingens riktning och fönstrets normal. Detta kan göras genom att utgå från aktuellt datum och tid på dygnet.

En metod för att räkna ut solens position presenteras i \cite{walraven78} och en Matlabfunktion baserad på samma artikel kan ses i appendix. % Hänvisa till kod

Om azimuthala och altitudinella vinklarna ($\beta$ respektive $\alpha$) relativt ett väderstreck respektive horisonten tillhandahålls beräknas infallsvinkeln mot glaset, $\theta$, på följande vis:

\begin{equation} 
\theta = \arccos{\left( \cos{\left(\beta - \gamma\right)}\cos{\left(\alpha\right)}\right)}
\end{equation}

där $\gamma$ är vinkeln mellan fönstrets normal och väderstrecket mot vilken azimuthala vinkeln anges. Notera att alla vinklar utom $\theta$ anges i radianer.

Med dessa samband tillgängliga kan ett Matlabprogram för beräkning av effektflödet på grund av solstrålning genom fönster skapas, och ett exempel kan ses i appendix. % Hänvisa till appendix.

% Behöver: longitud, latitud, vinkeln relativt väderstreck, fönsters area, g-värde

\subsubsection{Inverkan av skuggor, gardiner och dylikt}

% Beräkna för vilka vinklar skuggor faller över fönstren

% Gardiner, persienner och interiör förändrar situationen, kolla källan nedan
\begin{comment}
Simmler & Binder
Experimental and numerical determination of the total solar energy transmittance of glazing with venetian blind shading
\end{comment}

% Be Särnöe om specifikationer:
% - I vilket väderstreck är normalen riktad?
% - Vilket g-värde har fönstren?

\begin{comment}
I diskussion:
- Hur kan man koppla detta till värmesystemet?
	- Registrera intensitet, tid på dygnet och datum
	- Beräkna ungefärlig tillförd effekt
	- Kompensera genom att säga till värmesystemet att minska/stänga inflödet
- Blir det lättare att helt enkelt mäta temperaturen i rummet och gå utifrån det? Vad är mer kostnadseffektivt?
\end{comment}

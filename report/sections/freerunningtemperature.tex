\section{Free-running temperature – en konceptuell beskrivning}
\label{sec:freerunningtemp}

Free-running temperature är ett begrepp som används för att sammanfatta olika 
värmeflödens påverkan på byggnader. Det finns tyvärr ingen bra svensk översättning. Det kan beskrivas som den inomhustemperatur som fås om byggnaden används 
normalt men den aktivt tillförda energin, det som normalt kallas uppvärmning via radiatorer, stängs av.

I en fastighet finns det många olika värmekällor, så som värme från elektriska apparater, 
människorna som vistas där, belysning och varmvatten för hushållsbruk. All den värmen
 bidrar till att värma upp huset. På grund av termodynamikens huvudsatser vet vi att 
 kroppar i kontakt alltid strävar efter jämvikt och på så sätt får vi ytterligare energiflöden på 
 grund av vind, sol och utomhustemperatur.

Normalt används sedan den tillförda energin till att utjämna detta. Genom att inte värma 
fastigheten kan man istället räkna på hur mycket energi man behöver tillföra fastigheten 
vid olika tidpunkter för att nå önskad inomhustemperatur.

Denna storhet kan givetvis mätas, men eftersom man vill bibehålla aktiviteten är detta troligen inte så populärt hos de som använder byggnaden, speciellt inte med det klimat vi har i Sverige. Istället används oftast olika modeller för att beräkna den.

Måttet free-running temperature kan användas till flera olika saker. Det enklaste är att 
jämföra olika byggnader, där man i och med den fortsatt aktiviteten i byggnaden jämför
 dem med hänsyn till vad de används till – ett vilohem eller en idrottshall har troligen 
 ganska olika free-running temperature även om de skulle ha exakt samma 
 byggnadstekniska specifikation. Vid beräkning av värdet behöver man givetvis inte ta 
 hänsyn till detta men det kan ändå vara intressant för undersöka energibehovet.

Ett annat användningsområde är att sätta upp en statistisk modell där man kan visa hur 
energibehovet förändras beroende på verksamhet och väderparametrar. Denna applikation ligger betydligt närmare till hands för det här arbetet. Här används dock en variant av free-running temperature, där energiutflödet vid konstant inomhustemeperatur har beräknats, alltså hur mycket energi som måste tillföras för att upprätthålla önskad inomhustemperatur.

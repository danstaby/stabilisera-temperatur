\section{Datorsimulering av ofrivillig ventilation}

Ett hus är i praktiken omöjligt att göra helt tätt. Då vinden ligger på
får man därför ett drag genom huset, en ofrivillig ventilation. Då vinden
sällan är lika varm som inomhusluften leder detta till en energiförlust.
Den har beräknats med hjälp av programvaran Comsol. Problemets geometri har
setts upp enligt figur \ref{fig:windmethod:tri}.


\begin{figure}
\centering
\includegraphics[width=127mm,height=76mm]{images/triinfiltration.eps}
\caption{Triangulering samt definitionsmängd uppsatt för problemet}\label{fig:windmethod:tri}
\end{figure}

Från vänstra kanten så har luften blåst in med en konstant vindhastiget som är varierad i olika
experiment. På andra sidan av fastigheterna är det satt ett konstant lufttryck som motsvarar en
atmosfärs tryck. På ränderna som ligger mot mark eller mot hus är vindhastigheten satt till noll.
Slutligen utför ej luftmassan ovanför definitionsmängden någon kraft på luften som ligger längs den
övre randen.

Trycket inne i fastigheterna har sedan beräknats genom att luftläckaget antagits homogent utspritt över fastigheternas
väggar och att inget luft läckt genom taket. Därefter har Darcys lag satts upp med antagande om jämvikt så att lika mycket
luft som flödar in även kommer ut. Då både trycket inomhus och på ränderna är kända kan läckaget beräknas med Darcy's lag
eller med någon annan exponent. Här är antagandet gjort att huset läcker mycket och har $C(50)^{0,60} = 1,2$. Dock
så kommer även exponenten ha betydelse för läckaget.\cite{sasic}

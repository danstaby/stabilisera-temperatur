\subsection{Johanneberg 7:8 – en beskrivning av fastigheten på Walleniusgatan}

Huset uppfördes 1935 och sedan kom det att dröja ända till 1988 innan den första större ombyggnationen gjordes. Då gjordes två lägenheter om till kontor, stammar byttes och fasand och tak rustades upp. Tilläggsisolering? I samband med detta installerades nya värme- och ventialtionssystem. Under 2009 (blev det så?) togs ett datoriserat styr- och övervakningssytem för fastighetens energiförsörjning i drift.

Idag har fastigheten 13 lägenheter samt en större kontorslokal fördelat på åtta (är krypgrund inräknat? visa gärna ritning och fråga Peter) våningar som totalt utgör $\unit[1450]{m^2}$.

Källa:
föreningens årsredovsining 2008, http://www.psarneo.se/bilder/redovisning_2008.pdf
Ritning från stadsbyggnadskontorets arkiv.



\subsubsection{Huset som bostad}
  
% Det är så här stort och har så här många rum, så här högt i tak o.s.v. 

% De olika gränsytornas material och uppbyggnad.
\subsubsection{Väggarna}

\subsubsection{Taket}

\subsubsection{Grunden}


\subsubsection{Huset byggandstil och histora}
% Hur tänkte de när de byggde och renoverade huset?
% Vad ville de uppnå och vilka regler och normer hade man att hålla sig till?

\subsubsection{Byggnadsfysikens betydelse}
% Att huset är byggt så här vad betyder det för hur huset påverkas och hur huset är att bo i?

\subsubsection{Uppvärmning}
% Kortfattat vad finns det för värmeförsörjningssystem idag?
% Hur fungerar det? Varför valde man det?

\subsubsection{Ventilation}
% Hur fungerar ventilationen?
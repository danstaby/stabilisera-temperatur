\chapter{Diskussion}

% Intro. Diskutera syftet – har det uppfyllts?
Syftet med projektet var att undersöka energiflödena genom en byggnads gränsytor och hur dessa påverkas av vädret. Projektet har tittat på energiflöden genom fastighetens olika typer av väggar, genom tak och fönster samt genom grunden. Dessa har undersökts med avseende på de tre formerna av värmeöverföring: strålning, ledning och konvektion. Vi har också tagit hänsyn till den värme som produceras inne i fastigheten av människor och deras verksamhet. På så sätt har vi kunnat få fram hur mycket energi fastigheten förlorar eller tillförs vid olika väderförhållanden.

% TIDIGARE TEXT
% Under vår undersökning av hur byggnadens energiflöden påverkas av olika 
% väderparametrar har vi stött på problem ibland varit tvungna att göra grova 
% approximationer. Här diskuterar vi problemen närmare och utvecklar resonemanget om
% vilka fel de kan ha gett upphov till. Vidare kommer några förslag på hur arbetet kan
% appliceras för att minska energiflödet ut ur den aktuella byggnaden på Walleriusgatan 
% likväl som mer allmänt. Slutligen ges några förslag på hur man kan gå vidare och 
% vidareutveckla det här arbetet. 

\section{Diskussion kring metoden}\label{sec:discmethod}

I avsikt att nå projektets mål har en rad olika metoder använts. För att utföra beräkningar på strålning från både solen och omgivningen har principen kring svartkroppsstrålning använts. För beräkningar på ledning utnyttjas termodynamikens nollte huvudsats och dessa beräkningar genomförs med Fouriers värmelag. Konvektion sker främst på grund av cirkulation i luften och för att beräkna hur detta sker har ett flertal strömningsmekaniska metoder använts.

I början av projektet fanns det förhoppningar om att fastighetens väderstation skulle kunna tillhandahålla väderdata att använda för att sätta upp statistiska modeller. 
Ganska snart insågs dock att detta inte skulle bli verklighet och helt analytiska modeller blev istället vår metod. Senare visade sig att statistiken hade varit bra för att närmare kunna undersöka hur stor inverkan olika väderparametrar har vid just fastigheten på Walleriusgatan. 
Detta fick istället lösas med statistik från SMHI:s hemsida. Statistiken var tyvärr något otillräcklig.

För att sätta upp de fysikaliska modellerna behövdes också värden för ett antal materialkonstanter för luft och för tegel. 
Dessa beror dock av en eller flera väderparametrar såsom tryck, temperatur och fukt och således är det tveksamt om de överhuvudtaget borde kallas konstanter. 
Det visade sig dock vara mycket svårt att få fram tillförlitlig och tillräckligt högupplöst data som visade detta, än mindre att hitta uttryck på ekvationsform.
Vid diskussion med sakkunniga fick vi uppfattningen att det finns en tydlig konsensus inom branschen vad som är relevant och inte, men ingen tydlig dokumentation på området.

Vid lösning av de relevanta differentialekvationerna behövdes problemets geometri och randvillkor sättas upp. För
att förenkla beräkningarna har ofta antalet dimensioner i problemen reducerats. Bland annat så har väggarna antagits vara oändligt
långa vilket är ekvivalent med att de är perfekt isolerade på korsidorna, alltså alla andra sidor än utsidan och insidan. Detta stämmer inte
i praktiken och kan således vara en källa till fel. En liknande approximation genomfördes för grunden, där fastigheten
antagits vara oändligt lång. I det här fallet är den undersökta fastigheten mittendelen av en lång fastighet vilket
gör att approximationen är giltig men så pass grov att den ändå är en källa till fel. Bergets material antas vara homogen granit vilket
heller inte är helt korrekt. 

I alla lösningar har differentialekvationernas beteende utanför definitionsmängderna approximerats. Beteendet på randerna
är väldigt beroende på omgivningen och detta kommer även vara en stor källa till fel. Här har bland annat svartkroppsstrålning,
solinstrålning och konvektionsparametern approximerats. Dessa approximationer är en nödvändighet för att man inte ska behöva
simulera hela stadsdelen eller till och med staden vilket kommer bli väldigt beräkningsintensivt. Vi hoppas ändå att våra
värden ger en idé om hur det kan se ut i verkligheten och att det ger en god idé om hur stora energiflödena är i jämförelse med varandra.

Som ett extra steg har förslag på ytterligare energibesparande åtgärder tagits fram och dessa har jämförts med att ta hänsyn till väderparametrar på det sätt som projektet fått fram. För att sätta dessa i proportion till varandra har även konstnader för de olika åtgärderna tagits fram. Det har dock varit svårt att få exakta värden på kostnader då författarna för projektet är tredje part och inte direkt företräder fastighetens ansvariga.


\section{Diskussion kring resultatet}

\emph{\color{red} Vad resultatet faktiskt säger. Vad betyder olika storlekar på energiflödet vid olika tidpunkter.}

Resultaten är mycket osäkra.

En stor del av energiflödet in, men framför allt ut, genom byggnaden gick genom fönstren. Det var inte något som uppmärksammades förrän under senare delen av arbetet mycket beroende på att fastigheten har relativt moderna fönster och detta inte framstod som något som behövde åtgärdas. Det finns ett flertal olika metoder som lämpar sig för olika byggnader och klimat och det finns inget enkelt sätt att säga vilken som är bäst för just den här fastigheten.

Resultatet tar ingen hänsyn till eventuell fukt eller regn. Troligen tar väggarna åt sig fukt vid blött väder vilket kan ändra väggens värmeledande och värmelagrande egenskaper. Regn kan också tänkas störa luftflödena kring byggnaden och även direkt kyla den genom att vara kallt och ha hög värmekapacitet. Den störta förlusten tros dock vara avdunstning. I det här projektet tas dock ingen hänsyn till det. Vid uppehåll och torrt väder borde våra resultat vara fullt tillämpbara.

% I de här två avsnitten kommer feluppskattningar och uppkomna problem in.

% \section{Felkällor}\label{sec:errors}

% ELIMINERAT AVSNITT

Onoggrannheten i beräkningsmodellerna.

h-värdet är väldigt approximativt. Vind!

Alla approximationer – hur stora fel.



% Koppla mot avgränsningar.
% Vad har de här avgränsningarna lett till för fel.
 ELIMINERAT
% \subsection{Problem som vi stött på}

 ELIMINERAT

\section{Jämförelse med andra energibesparande åtgärder}

En övergripande likhet mellan energibesparande åtgärder är att många av dem kostar pengar. Innan det går att överväga något av dessa alternativ måste det klargöras vilka mervärden eller biverkningar olika system ger samt hur de kan utnyttjas maximalt. De finns lösningar som lämpar sig bättre och sämre för specifika byggnader. I det här avsnittet kommer en diskussion föras angående fyra alternativ till det momentant väderbaserade system rapporten undersöker i övrigt.
Som vi kan se i resultatet påverkar vädret energiflödet ut genom byggnaden högst väsentligt. Enligt \ref{} beror det mest på vind och solinstrålning och dess effekter fördröjs inte av trögheten i väggarna, vilket var en av utgångspunkterna för vårt. Solinstrålningen går främst genom fönster, och vinden går in i otätheter och för på så sätt in kallare luft i byggnaden utan fördröjning.

\paragraph{Besparingar}
Under rubrikerna besparingar kommer det presenteras vilka olika former av besparingar, effekter eller andra fördelar man får av att investera i den presenterade energisparformen. All energi som används till bostadsuppvärmning enligt \cite{energideklaration} antas ligga under vinterhalvåret, då uppvärming förekommer. Det antas inte försvinna energi till kyla under sommarhalvåret.  [Resultat ylva].

\paragraph{Ekonomi}
Under rubriken kommer förutsättningar i ekonomin beskrivas. Den är väldigt approximativ då vi inte har haft befogenheter att begära in riktiga offerter från bygg samt installationsföretag, då detta kostar dem pengar, i form av tid, samt att vi inte ens haft för avsikt att använda offerten, vilket är allmänt ojuste.

Vi använder genomgående Payoff-metoden, väldigt förenklad, med vilken man visar hur myket en investering betalar sig. Metoden tar inte hänsyn till att det är olika stora investeringar, då det inte framgår vilken man kan spara mest pengar på, utan vilken som lönar sig snabbast. Det finns inget restvärde för någon av investeringarna. Beräkningarna har gjorts för mest positiva samt mest negativa möjliga utfall, vilket ger ett spann på ett antal år. Förenklingarna leder till att vi varken tittar på kalkylräntor eller en eventuell ränta i det fallet investeringen måste finansieras med ett banklån.

\begin{equation} \label{eq:payoff}
\text{Antalet år}=\frac{\text{Grundinvestering}}{\text{Sparat belopp per år}}
\end{equation}

\subsection{Tilläggsisolering}
Våra beräkningar visar att väggen inte absorberar tillräckligt mycket energi för att det ska vara värt att inte isolera den. Se figurer i avsnitt \ref{sec:steadystatewall}.
När man gjorde en omfattande renovering i slutet av 1980-talet byggde man en yttre tilläggsisolering, med tio centimeter mineralull på den norra sidan, \cite{arsredovisning} innebärande 26 \% av husets yta utåt, grunden borträknat. Isoleringen gav en ungefärlig faktor en fjärdedel på U-värdet.Vi ser här det låga u-värdet på norrväggen kontra det fortfarande höga på sydsidan. \ref{tbl:uvalue} Dessa differenser ger oss möjligheten att göra förbättringar där det inte är tillräckligt isolerat, det vill säga till exempel där U-värdet ligger över ett.

En isolering av både syd samt västväggen är en isolering av en yta om 212 $\unit{m^2}$, som rent teoretiskt skulle få ett bättre U-värde. Det innebär 21,8 \% av ytan på fastigheten där energi kan ledas ut. \ref{tbl:uvalue}.
En tilläggsisolering kan göras på två olika sätt, inifrån eller utifrån. Med båda metoderna finns för samt nackdelar. Båda metoderna innebär betydande insatser i fastigheten. Det finns dock mervärden att ta under beaktande. 

\paragraph{Besparingar}
Fasader behöver med jämna mellanrum renoveras, och i samband med ett projekt av de proportionerna får man en fasadrenovering. Svenskt tegel har en ungefärlig livslängd på 50 år\cite{magnus}, dock ska tegelfasaden ha renoverats med ny impregnering i samband med renoveringen 1988. Teglet i fastigheten har troligtvis även en bättre livslängd än 50 år, kvaliteten när huset byggdes motsvarar det danska teglet, och då handlar det om cirka 100 år. Det största mervärdet ur vår synvinkel, vilket också framgick vid beställning att temperaturerna i bostäderna blir mycket jämnare [Referens från resultat]. Det beror på, vilket nämns tidigare i kapitlet att fasaden inte fungerar som den buffert eller ”element” som man tror att den gör.

\paragraph{Tilläggsisolering utifrån}
En tilläggsisolering utifrån är ett ingrepp som medför en stor kostnad. Områdena som bedöms vara lämpliga att tilläggsisolera är alla väggar med tegelyta, då de inte har någon tilläggsisolering sedan tidigare och U-värdet skulle då kunna förändras på sammma sätt som norrsidan när den isolerades vid renoveringen. Då det är en betydande investering måste mervärden och andra kostnader som kan tänkas uppstå tas under beaktande. Fasader behöver med jämna mellanrum isoleras. beräknat det genomsnittliga U-värdet per kvadratmeter för både fallet utan ytterligare isolering samt ett fall med isolering av alla tegelytor. Det handlar om skillnader i det genomsnittliga U-värdet för hela huset på ZZ \%[Referens]

\paragraph{Tilläggsisolering inifrån}
En tilläggsisolering inifrån innebär ingrepp i lägenheterna, bland de boende. Det kan medföra komplikationer med personer som inte vill göra lägenheten mindre, vilket är en mindre bieffekt av projektet. Hurvida de boende behöver kompenseras för ingreppet genom dubbelt boende eller ekonomiskt är inte klarlagt, men det behövs ta ställning till. Ingreppet blir inte lika komplett som att isolera utifrån, man kan helt enkelt inte isolera lika stor yta då innerväggar samt golvplan ansluter mot ytterväggen och bildar köldbryggor utåt. Summering ger att den beräknade ytan att isolera blir mindre, nedåt endast hälften mot den tidigare angivna samt att en isolering inifrån troligtvis blir mindre kostsam, men undersökningar har inte gjorts.

\paragraph{Ekonomi}
Vi har inte räknat på en isolering inifrån då det inte har varit möjligt att få en offert eller prisförslag på ett sådant arbete. Det är inte heller möjligt att göra på grund av de boendes preferenser. Genom en tjänst på internet fick vi visserligen kontakt med ett byggföretag som hävdade att de skulle kunna göra arbetet för 200 000 kronor, vilket enligt lite uppskattningar över kostnader för material, arbetskraft byggställningar och så vidare framstår som väldigt billigt. Dock har vi en uppskattad besparingsmöjlighet på 25 \%, vilket ger pay-off tiden dryga 10 år, vilket för en sådan typ av investering får anses vara helt okej.

\subsection{Termostater på element}
Styrning av rumstemperatur via termostater på element, är den tredje åtgärden att diskutera. I dagsläget regleras flödet manuellt via vred under elementen. De ställs in av fastighetsskötaren och kan regleras upp eller ner om bostadsrättsinnehavaren är missnöjd med inomhusklimatet. Termostater finns både elektriska samt 
Elektriska termostater finns i olika varianter, men det är främst en ett koncept från Danfoss vi tittat på. Det finns i två varianter vilka ska gå att implementera på i princip alla befintliga uppvärmningssystem. Man ställer in temperaturen man vill ha i grader, antingen i det billigare systemet direkt på termostaterna via en lite lcd-display, eller i det dyrare systemet som fungerar trådlöst mot en huvudenhet med en större färgdisplay.
Båda systemen kan bryta tillförseln vid tillfälliga kyltoppar, till exempel vid vädring. Det går genom systemet att hålla lägre temperatur i vissa rum, till exempel sovrum. Systemen kan programmeras att ta hänsyn både till solinstrålning genom fönster samt sänka dag, natte eller semestertid, när ingen är hemma.

\paragraph{Sidoeffekt}
En bieffekt att energianvändningen kan öka om det finns tillräcklig med energi i systemet och de boende vill ha varmare än vad som i dagsläget erbjuds. Det är lätt att begränsa temperaturintervallet, men det kan skapa irritationer om de boende upplever fel temperatur och inte kan ändra den när de har fått ett fint system för det.

\paragraph{Ekonomi}
Det är tveksamt vilken längd man skulle kunna tänka sig på en avskrivning av det här slaget, men en  rimlig avskrivningstid är fem år då det innefattar ett datorbaserat system. I kostnaden har endast materialet tagits med, det vill säga termostater samt huvudenheter. Dessa komponenter skulle kosta föreningen knappa 100 000 kronor, och med en besparing på upp till 45 \% per år blir payoff-tiden endast 2,85 år. Det är ungefär motsvarande en halv rimlig avskrivningstid, vilket är en bra investering sett till de perspektiven.

\subsection{Miljöinformation}
I samband med ett system där de boende själva kan reglera temperaturen kan man behöva informera de boende på ett attraktivt sätt, meningen är ju att både kostnaden ska sjunka, samt att miljöpåverkan ska minskas. Börjar temperaturen då smyghöjas i lägenheterna blir så inte fallet, det börjar kosta mer i el, samt förslitning av pumpar m.m.  \cite{viivilla}


\subsection{Prognosstyrning}
Prognosstyrning är en möjligheten för att styra inomhustemperaturen efter vädret, vilket skulle kunna vara ett alternativ för byggnaden vi har studerat.
Fördelarna med prognosstyrning kontra att styra direkt mot väderstationen är att de flesta markanta väderpåverkningarna är momentana, det vill säga att det inte sker någon fördröjning innan de kommer in i byggnaden. Då krävs framförhållning för att kunna ta hänsyn till dessa, vilka är främst vind och solinstrålning, vilket kan ges av prognosstyrning. Sveriges meterologiska och hydrologiska institut, SMHI står för prognoserna som skickas till systemet, vilket i sin tur har installerats av deras samarbetspartner. En prognos skickas egentligen aldrig, utan det som skickas är en styrsignal som baseras på väderprognoser samt data om byggnadens energibalans som skickas till styrsystemet, som sedna försöker optimera energiåtgång och boendekomfort efter givna värden.

\paragraph{Besparingar}
SMHI påstår att det kan ge kostnadsbesparingar på 5-10 \% på uppvärmningen, och att förutsäga exakt hur mycket är väldigt svårt, då deras modell angående vilka parametrar styrningen beror på är kommersiell och således inte delges allmänheten. Antydningar från mötet med SMHI menade på att den här fastigheten inte låg i den övre delen av skalan, men det beror ju helt på hur olika parametrar viktas, och en korrektare bedömning kräver troligtvis att man har för avsikt att köpa det.

\subsection{Ekonomi}
Antalet år det tar innan en investering betalar sig enligt \ref{eq:payback} ges i tabell \ref{tbl:payback}

\begin{table}[hbtp]
\centering
\caption{Paybacktid för olika investeringar}
\label{tbl:payback}

\begin{tabular}
{|l|r|r|}
\hline
\textbf{Investering} & \textbf{Minimal paybacktid} &{\textbf{Maximal paybacktid} \\
\hline
Prognosstyrning &  1 & 2,9 \\ 
\hline
Väderstation & 1,7 & 5,2 \\
\hline
Termostater & 2,9 & 12,8 \\
\hline
Tilläggsisolering & 8,7 & 52,2 \\
\hline
\end{tabular}
\end{table}





\paragraph{Ekonomi}
Den stora fördelen med det här systemet är SMHIs prismodell. Den innebär att investeringskostnaden ska kunna betala sig inom två år, samt att abonnemangsavgiften inte är högre än maximalt hälften av vad man sparar varje år. Det är således väldigt intressant då man egentligen inte behöver några ekonomiska muskler för att börja använda produkten, även en förening som redan har mycket lån och nästan går med minus kan köpa systemet, för det blir billigare i vilket fall. Det bidrar till att samtidigt som SMHI kan ta betalt för sina prognoser, så sparar de boende pengar och mindre energi behöver produceras. SMHI lejer ut installationen till sina samarbetspartners. Systemet är enkelt och lätt att installera, vilket tillsammans med det låga priset bidrar till att göra det attraktivt.

Med sina två års payback-tid är prognosstyrning det alternativet som verkar bäst enligt payback-metoden. Här visar det sig att de största besparingarna dock inte alltid hittas med Paybackmetoden. Skulle föreningen investera i systemet, som man enligt SMHI ska kunna få installerat för cirka 10 000kronor, sparar man dock inte så mycket per år. Det är metoden med klart sämst förbättringspotential, samtidigt som SMHI kräver en abonnemangsavgift varje år, dock inte högre än runt 30 \% av vad man sparar
 % Förbättningsåtgärder och ekonomi

\section{Tillämpningar}

De två primära tillämpningarna av det här arbetet är att identifiera energiläckor för att kunna göra sitt hus mer energieffektivt och att låta sitt energiförsörjningssystem vara väderberoende.

Att energieffektivisera sitt hus kan, om man väljer rätt metod, löna sig både ekonomiskt och miljömässigt. Det leder dessutom till mindre fluktuationer i inomhustemperaturen, speciellt om man har stor tröghet i sitt uppvärmningssystem.

Att låta energiförsörjningssystemet bero av väderdata från en vid fastigheten monterad väderstation kräver experimentella mätning på den aktuella fastigheten för att bli implementerbart.
Dessutom kommer direkta eneriflöden, så som solinstrålning och vind, att kompenseras för fördröjt, vilket kan bli ett problem om man har ett långsamt system för uppvärmning. Ett alternativ då är prognosstyrning men de färdiga system som finns på marknaden idag är så pass bra att det inte är av intresse att försöka bygga ett eget, inte ur ekonomisk synvinkel i alla fall.
 % Ska det vara kvar? Ska det stoppas in i någon annan del?

\section{Rekommendationer till fortsatt arbete}

Detta är det tredje arbetet som modellerat olika delar av fastigheten på Walleriusgatan för att
kvantifiera värderpåverkan. Detta har genomförts med olika metoder men slutsatserna från arbetena är ungefär desamma.
Därför känns det onödigt att genomföra ett fjärde arbete med samma infallsvinkel. Man kan bara komma så
långt med liknande metodik.

För fortsatt arbete rekommenderas att någon av två olika vägar väljs. De benämns
som produktspåret och grundforskningsspåret. Nedan följer en beskrivning av våra idéer kring dessa.

\paragraph{Produktspåret}
Uppdragsgivaren till detta projekt har som mål att helt automatisera värmeanläggningen i fastigheten och att stabilisera
temperaturen med avseende på väderpåverkan. För att genomföra detta behöver man implementera ett självförbättrande
reglertekniskt system, gärna i nära kontakt med industrin med sikte på att skapa en kommersiell produkt. 

Vid vårt möte med SMHI fick vi intrycket av att de skulle vara väldigt intresserade av att få data från fastigheten.
Detta inkluderar både data från värmeanläggningen och data från väderstationen. Därför kan det vara
en god idé att lägga fortsatta arbeten på is tills det finns några års data från fastigheten att tillgå.
Denna skulle då, tillsammans med erbjudande om ett kandidat- eller examensarbete, kunna bytas mot att få ett värmesystem implementerat till en låg kostnad. Datan skulle dessutom vara till stor nytta för att studera vad olika väder sk1apar för
krav på en värmeanläggning. 

Kompetensen som skulle krävas för ovanstående projekt ligger inom områdena reglerteknik, fysik, byggfysik samt vana med maskininlärning.
Dessutom skulle det vara en fördel om kompetens inom ekonomiska beräkningar finns inom projektgruppen för att
kunna tillgodose att produkten blir lönsam både för företaget och för kunden. Vi anser att detta skulle vara en god
idé till kandidat. eller examensarbete, dock med kravet att dessa kan samarbeta med
ett företag som håller på med denna typ av värmeanläggningar. Detta tror vi är nödvändigt för att projektet skall kunna mynna ut i en marknadsredo produkt.


\paragraph{Grundforskningsspåret}

De tidigare genomförda arbetena har enbart skummat lite på ytan angående vädrets påverkan på fastigheter och alla energiförluster
har kvantifierats väldigt approximativt. För att förbättra tidigare uppsattningar bör en djupare anayls av specifika energiförluster göras. Dessa skulle dock behöva vara kraftigt avgränsade. Några intressanta saker att studera
närmare är vädrets påverkan av konvektionsparametern. Detta skulle kunna genomföras med smarta experiment eller
med avancerade datormodeller. En annan idé är att närmare studera strålning som passerar genom byggnadens fönster. Till detta
skulle det krävas experiment eller data för fönstren för att få veta frekvensberoendet på absorbtions-, reflektions- samt
transmitanskoefficienterna. Med
denna data så skulle det sedan vara möjligt att räkna, alternativt modellera, strålningen genom fönstren. Detta skulle troligen
vara ett lämpligt projekt för några Fysik- och Kemiteknik-teknologer.

Om det skulle finnas några års data från fastighetens värmeanläggning samt väderstation så skulle det
även vara lämpligt att behandla denna med statistik. Genom att genomföra kan man förhoppningsvis
gå att få en ganska god bild över hur väder påverkar fastigheten i realiteten. Ett väl utfört arbete skulle
även kunna bidraga med ett recept för att göra liknande studier på andra fastigheter. Med lite tur skulle detta
kunna bidra med nya insikter i hur väder påverkar en fastighets uppvärmning och kunna leda till nya idéer angående
hur vädrets negativa inverkan kan minimeras och dess positiva inverkan utnyttjas maximalt.
 % Förslag till fortsatt arbete


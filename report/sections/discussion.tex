\chapter{Diskussion}

% Intro. Diskutera syftet – har det uppfyllts?
Syftet med projektet var att undersöka energiflödena genom en byggnads gränsytor och hur dessa påverkas av vädret. Projektet har tittat på energiflöden genom fastighetens olika typer av väggar, genom tak och fönster samt genom grunden. Dessa har undersökts med avseende på de tre formerna av värmeöverföring: strålning, ledning och konvektion. Vi har också tagit hänsyn till den värme som produceras inne i fastigheten av människor och deras verksamhet. På så sätt har vi kunnat få fram hur mycket energi fastigheten förlorar eller tillförs vid olika väderförhållanden.

% TIDIGARE TEXT
% Under vår undersökning av hur byggnadens energiflöden påverkas av olika 
% väderparametrar har vi stött på problem ibland varit tvungna att göra grova 
% approximationer. Här diskuterar vi problemen närmare och utvecklar resonemanget om
% vilka fel de kan ha gett upphov till. Vidare kommer några förslag på hur arbetet kan
% appliceras för att minska energiflödet ut ur den aktuella byggnaden på Walleriusgatan 
% likväl som mer allmänt. Slutligen ges några förslag på hur man kan gå vidare och 
% vidareutveckla det här arbetet. 

\section{Diskussion kring metoden}

I avsikt att nå målet har en rad olika metoder används. För strålning från både solen och omgivningen har svartkroppsstrålning används. För ledning utnyttjas det fysikaliska fenomenet att värme leds från varma till kalla objekt som står i kontakt med varandra. Detta beräknas med Fouriers värmelag. Konvektion sker främst på grund av cirkulation i luften och för att beräkna det har ett flertal strömningsmekaniska metoder använts.

\emph{\color{red} Motivera varför vi har valt våra metoder}

I början av projektet fanns det förhoppningar att fastighetens väderstation skulle kunna tillhandahålla väderdata att använda för att sätta upp statistiska modeller. 
Ganska snart insågs dock att detta inte skulle bli verklighet och helt analytiska modeller blev istället vår metod. Senare visade sig att statistiken hade varit bra för att närmare kunna undersöka hur stor inverkan olika väderparametrar har vid just fastigheten på Walleriusgatan. 
Detta fick istället lösas med statistik från SMHIs hemsida, som tyvärr var något otillräcklig.

För att sätta upp de analytiska modellerna behövdes också värden för ett antal materialkonstanter för luft och för tegel. 
Dessa beror dock av en eller flera väderparametrar så som tryck, temperatur och fukt och således är det tveksamt om de överhuvudtaget bör kallas konstanter. 
Det visade sig dock vara mycket svårt att få fram tillförlitlig och tillräckligt högupplöst data som visade detta än mindre uttryck av ekvationskaraktär.
Vid diskussion med sakkunniga fick vi uppfattningen att det finns en tydlig konsensus inom branchen vad som är relevant och inte, men ingen tydlig dokumentation på området.

Flera av de analytiska datormodellerna har varit svåra att få till konvergens och de har krävt tunga beräkningar. \emph{\color{red} Utvecklas av någon med koll.}

Som ett extra steg har olika förslag på energibesparande åtgärder tagits fram och dessa har jämförts med att ta hänsyn till väderparametrar på det sätt som projektet fått fram. För att sätta dessa i proportion till varandra har även konstnader för de olika åtgärderna tagits fram. Det har dock varit svårt att få exakta värden på kostnader då projektet är tredje part och inte direkt företräder fastighetens ansvariga.


\section{Diskussion kring resultatet}

\emph{\color{red} Vad resultatet faktiskt säger. Vad betyder olika storlekar på energiflödet vid olika tidpunkter.}

På grund av approximationerna som beskrivs i avsnitt \ref{sec:discmethod} är resultaten en uppskattning av energiflödenas storlek. Dessa ska ses som en fingervisning av vad man kan förvänta sig för energiflöden genom en byggnad i samma stil som den på Walleriusgatan vid olika årstider och väderlekar. Det intressanta är framför allt hur stora flödena är relativt varandra och hur de olika flödena förändras med de olika väderparametrarna.

En stor del av energiflödet in, men framför allt ut, genom byggnaden gick genom fönstren. Det var inte något som uppmärksammades förrän under senare delen av arbetet. Det beroende mycket på att fastigheten har relativt moderna fönster och detta inte framstod som något som behövde åtgärdas. Det finns ett flertal olika metoder för att åtgärda energiflödet genom fönster som lämpar sig för olika byggnader och klimat och det finns inget enkelt sätt att säga vilken som är bäst för just den här fastigheten.

Resultatet tar inte hänsyn till eventuell fukt eller regn. Troligen tar väggarna åt sig fukt vid blött väder vilket kan ändra väggens värmeledande och värmelagrande egenskaper. Regn kan också tänkas störa luftflödena kring byggnaden och även direkt kyla den genom att vara kallt och ha hög värmekapacitet. Den störta förlusten tros dock vara avdunstning. I det här projektet tas dock ingen hänsyn till det. Vid uppehåll och torrt väder borde resultaten vara fullt tillämpbara.

% I de här två avsnitten kommer feluppskattningar och uppkomna problem in.

% \section{Felkällor}

Onoggrannheten i beräkningsmodellerna.

Ingen hänsyn till att väggarna tar åt sig fukt vid blött väder och därmed ändrar egenskaper (t.ex. ändras värmeledningsförmågan). 

Alla approximationer – hur stora fel.

% Koppla mot avgränsningar.
% Vad har de här avgränsningarna lett till för fel. ELIMINERAT
% \section{Problem som vi stött på}

I början av arbetet hoppades vi att fastighetens väderstation skulle kunna tillhandahålla väderdata som vi kunde använda för att sätta upp statistiska modeller. 
Ganska snart insåg vi dock att detta inte skulle bli verklighet och vi planerade istället att sätta upp analytiska modeller. 
Senare visade sig dock att statistiken hade varit bra för att närmare kunna undersöka hur stor inverkan olika väderparametrar har vid just fastigheten på Walleriusgatan. 
Detta fick istället lösas med statistik från SMHIs hemsida, som dock visade sig vara något otillräcklig.

För att sätta upp de analytiska modellerna behövdes också värden för ett antal materialkonstanter för luft och för tegel. 
Dessa beror dock av en eller flera väderparametrar så som tryck, temperatur och fukt och således är det tveksamt om de överhuvudtaget bör kallas konstanter. 
Det visade sig dock vara mycket svårt att få fram tillförlitlig och tillräckligt högupplöst data som visade detta än mindre uttryck av ekvationskaraktär. 
Vid diskussion med sakkunniga fick vi uppfattningen att det finns en tydlig konsensus inom branchen vad som är relevant och inte, men ingen tydlig dokumentation på området.

Konvergens. Svårt att få våra datamodeller att konvergera. 

Svårt att få exakta värden på kostnader efter som vi är tredje part.
 ELIMINERAT

\section{Jämförelse med andra energibesparande åtgärder}

Det finns flera olika energibesparande åtgärder att överväga. Innan något av alternativen implementeras måste det klargöras vilka kostnader, mervärden och sidoeffekter olika system ger samt hur de kan utnyttjas maximalt. De finns lösningar som lämpar sig bättre och sämre för specifika byggnader. I det här avsnittet kommer en diskussion föras angående fyra alternativ till det momentant väderbaserade system rapporten i övrigt undersöker.
Som kan ses i resultatet påverkar vädret högst väsentligt energiflödet genom byggnaden. Enligt avsnittet \ref{resultsfreerunning} beror det främst på vind och solinstrålning och dess effekter fördröjs inte av trögheten i väggarna, vilket var en av utgångspunkterna för arbetet. För att kunna reglera enbart på redan insamlad data behöver det finnas en fördröjning innan vädret påverar inomhusklimatet för att reglersystemet ska hinna med. Solinstrålningen värmer direkt genom fönster, och vinden går in igenom otätheter och för på så sätt in kallare luft i byggnaden utan fördröjning.

All energi som används till bostadsuppvärmning enligt fastighetens energideklaration\cite{energideklaration} antas ligga under vinterhalvåret, då uppvärmning förekommer. Det antas inte försvinna energi till kyla under sommarhalvåret, då fastigheten idag inte har något kylsystem.

\subsection{Tilläggsisolering}
Våra beräkningar visar att väggen inte absorberar tillräckligt mycket energi för att det ska vara värt att inte isolera den. Se figurer i avsnitt~\ref{sec:steadystatewall}.
När man gjorde en omfattande renovering i slutet av 1980-talet byggde man en yttre tilläggsisolering, med tio centimeter mineralull på den norra sidan\cite{arsredovisning}, som motsvarar $\unit[26]{\%}$ av husets yta utåt, grunden borträknad. Isoleringen gav en sänkte U-värdet med en fjärdedel. Norrväggen fick då ett lägre U-värde medan syd- och västväggarna fortfarande hade ett relativt högt, se tabell {tbl:uvalue}.
Dessa differenser belyser var byggnaden är dåligt isolerad och det är lämpligt att göra förbättningar, till exempel där U-värdet ligger över ett.

En isolering av både syd- och västväggarna är en isolering av en yta om 212 $\unit{m^2}$, som rent i och med isolering skulle få ett bättre U-värde. Det motsvarar $\unit[21,8]{\%}$ av fastighetens ytan där energi kan ledas ut, se tabell \ref{tbl:uvalue}.
En tilläggsisolering kan göras på två olika sätt, inifrån eller utifrån. Med båda metoderna finns för samt nackdelar. Båda metoderna innebär också betydande insatser i fastigheten. Det finns dock mervärden att ta under beaktande. 

\paragraph{Sidoeffekter och besparingar}
Fasader behöver med jämna mellanrum renoveras, och i samband med tilläggsisolering kan det vara lämpligt att även fasadrenovera. Svenskt tegel har en ungefärlig livslängd på 50 år\cite{magnus}, dock ska tegelfasaden ha renoverats med ny impregnering i samband med renoveringen 1988. Teglet i fastigheten har troligtvis även en bättre livslängd än 50 år,  då tegel höll högre kvalitetet när huset byggdes och då handlar det om en livslängd på ungefär 100 år. Det största mervärdet utifrån det som initierade det här projektet är att temperaturerna i bostäderna blir jämnare, se figur~\ref{fig:energyflow_stst} och \ref{fig:wall_dec}. Det beror på att fasaden inte fungerar så bra som att spara energi i från den varmare till den kallare delen av dygnet. Dess buffrande egenskaper är inte så stora som man vill tro.

\paragraph{Tilläggsisolering utifrån}
En tilläggsisolering utifrån är ett ingrepp som medför en stor kostnad. Områdena som bedöms vara lämpliga att tilläggsisolera är alla väggar med tegelyta, då de inte har någon tilläggsisolering sedan tidigare och U-värdet skulle då kunna sänkas på samma sätt som norrsidan när den isolerades. U-värdet för hela fastigheten skulle då kunna sänkas från 0,6 till 0,4. Men då tilläggsiolering är en betydande investering måste mervärden och andra kostnader som kan tänkas uppstå tas under beaktande. Då alla fasader med jämna mellanrum behöver renoveras kan det vara lämpligt att göra isoleringen i samband med detta. Vissa kostnader skulle då kunna minskas, till exempel den för uppsättande av byggnadsställningar.

\paragraph{Tilläggsisolering inifrån}
En tilläggsisolering inifrån innebär ett ingrepp i lägenheterna och förutom den direkta nackdelen att det stör dem som bor där medför det också att lägenheterna blir något mindre. Huruvida de boende behöver kompenseras ekonomiskt eller i form av alternativt boende i samband med renoveringen har inte tagits ställning till. Vidare blir ingreppet inte lika komplett som vid isolering utifrån, eftersom det inte går att isolera där innerväggar och golvplan ansluter till ytterväggen och bildar köldbryggor utåt. Den beräknade ytan att isolera blir nedåt hälften mot att isolera utifrån. Detta gör den troligen själva isoleringen mindre kostsam, men det har inte undersökts närmare.

\subsection{Termostater på radiatorererna}
Den tredje åtgärden som undersökts är styrning av rumstemperatur via termostater på radiatorerna. I dagsläget regleras flödet manuellt via vred under elementen. De ställs in av fastighetsskötaren och kan regleras om bostadsrättsinnehavaren är missnöjd med inomhusklimatet. Elektriska termostater finns i olika varianter, men det är främst en ett koncept från Danfoss som undersökts. Det finns i två varianter vilka går att implementera på i princip alla befintliga uppvärmningssystem. Temperatur ställs in i önskat antal grader, i det billigare systemet direkt på termostaterna via en lite lcd-display, eller i det dyrare systemet med en större färgdisplay som är ansluten trådlöst till en huvudenhet. Båda systemen kan automatiskt bryta tillförseln av energi vid tillfälliga kyltoppar, till exempel vid vädring. Det går genom systemet att hålla lägre temperatur i vissa rum, till exempel sovrum. Systemen kan programmeras att ta hänsyn till solinstrålning genom fönster och att sänka temperaturen under dagen, natten eller semestern. Enligt resultaten i avsnitt~\ref{resultsfreerunning} vet vi att kompensering för solinstrålningen kan ge en stor energibesparing.

\paragraph{Sidoeffekter och besparingar}
En negativ bieffekt är att energianvändningen kan öka om det finns tillräcklig med energi i systemet och de boende vill ha varmare än vad som i dagsläget erbjuds, vilket leder till att det går åt mer energi än tidigare. Det är lätt att begränsa temperaturintervallet för alla, men det kan skapa irritation hos de boende om de upplever fel temperatur och inte kan ändra, när systemet ska klara det. I samband med ett system där de boende själva kan reglera temperaturen implementeras bör man informera de boende på ett attraktivt sätt – meningen är ju att både kostnaden ska sjunka, samt att miljöpåverkan ska minskas. Höjs då temperaturen i lägenheterna blir så inte fallet utan det börjar istället kosta mer i både uppvärmning och förslitning av pumpar. \cite{viivilla}

\subsection{Prognosstyrning}
Prognosstyrning är en möjligheten för att styra inomhustemperaturen efter vädret, vilket skulle kunna vara ett alternativ till styrning efter väderstationen för byggnaden som har studerats.
Fördelarna med prognosstyrning jämfört med att styra direkt mot väderstationen är att de flesta markanta väderpåverkningarna är momentana, det vill säga att det inte sker någon fördröjning innan de påverkan inomhusklimatet. Då krävs framförhållning för att kunna ta hänsyn till dessa vilket kan ges av prognosstyrning. Sveriges meterologiska och hydrologiska institut, SMHI, står för prognoserna som skickas till systemet, vilket i sin tur har installerats av deras samarbetspartner. En prognos skickas egentligen aldrig, utan det som skickas är en styrsignal som baseras på väderprognoser samt data om byggnadens energibalans som skickas till styrsystemet. Styrsystemet försöker sedan optimera energiåtgång och boendekomfort efter de givna värden.

\paragraph{Sidoeffekter och besparingar}
SMHI påstår att deras prognosbaserade system ger kostnadsbesparingar på 5 till 10 \% på uppvärmningen. Att förutsäga exakt hur mycket är väldigt svårt då deras modell angående vilka parametrar styrningen beror på är kommersiell och således inte delges allmänheten. Antydningar från vårt möte med SMHI menade att den här fastigheten inte låg i den övre delen av skalan, men det beror ju helt på hur olika parametrar viktas, och för en korrektare bedömning kräver troligtvis att man har för avsikt att köpa det.

\subsection{Ekonomiska uppskattningar kring de olika åtgärderna}
Vilka ekonomiska förutsättningar som krävs för de olika åtgärden kan bara uppskattas då vi inte har haft befogenheter att begära in riktiga offerter från bygg- och installationsföretag.

För att beräkna kostnaderna användes genomgående en förenklad version av payoff-metoden, som visar hur många år det tar innan en investering betalar av sig. Metoden tar inte hänsyn till att det är olika stora investeringar, och det framgår således inte vilken metod man kan spara mest pengar på utan vilken som lönar sig snabbast. Det finns inget restvärde för någon av investeringarna. Beräkningarna har gjorts för mest positiva samt mest negativa möjliga utfall, vilket ger ett spann på ett antal år. Förenklingarna leder till att varken kalkylräntor eller en eventuell ränta i det fallet investeringen måste finansieras med ett banklån tas med i beräkningen. Den enklaste formen ges enligt ekvation\cite{ind.ek}

\begin{equation} \label{eq:payback}
\text{Antalet år}=\frac{\text{Grundinvestering}}{\text{Sparat belopp per år}}
\end{equation}

Antalet år det tar innan en investering betalar sig enligt \eqref{eq:payback} ges i tabell \ref{tbl:payback} nedan.

\begin{table}[hbtp]
\centering
\caption{Paybacktid för olika investeringar}
\label{tbl:payback}

\begin{tabular}
{|l|r|r|}
\hline
\textbf{Investering} & \textbf{Minimal paybacktid[år]} &{\textbf{Maximal paybacktid[år]}} \\
\hline
Tilläggsisolering & 8,7 & 52,2 \\
\hline
Termostater & 2,9 & 12,8 \\
\hline
Prognosstyrning &  1 & 2,9 \\ 
\hline
Väderstation & 1,7 & 5,2 \\
\hline

\end{tabular}
\end{table}

\paragraph{Isolering}
Vi har inte räknat på en isolering inifrån då det inte har varit möjligt att få en offert eller prisförslag på ett sådant arbete. Det är inte heller säkert att det är möjligt att göra på grund av de boendes preferenser. Genom en internetbaserad förfrågningstjänst en tjänst på internet fick vi visserligen kontakt med ett byggföretag som hävdade att de skulle kunna göra arbetet för 200 000 kronor, vilket enligt lite uppskattningar över kostnader för material, arbetskraft och byggställningar framstår som väldigt billigt. Dock har vi en uppskattad besparingsmöjlighet på 25 \%, vilket ger payoff-tiden dryga 10 år, vilket för en sådan typ av investering får anses vara låg. Kostanden 200 000 kronor är troligtvis för låg, då det endast var en bedömning utifrån ytan som skulle isoleras samt att det var upp till 8 våningar över mark. En seriös offertförfrågning skulle ge en rättvisare approximation, men det bedömdes att de befogenheterna saknades inom projektet.

\paragraph{Termostater}
Det bästa fallet är framräknat utan avseende på varken installationskostnad eller räntor på kapital som måste lånas in. Sämsta fallet framräknat med en installationskostnad lika hög som materialet, vilket bör kunna motsvara ett rimligt scenario. En investering av det här slaget bör ha en avskrivningstid på fem år, vilket är intressant att jämföra payback-tiden mot. Materialet i sig skulle kosta föreningen knappa 100 000 kronor, och med en besparing på upp till 46 \% \cite{danfoss} per år blir payback-tiden under tre år. En payback-tid på runt halva uppskattade avskrivningstiden är en bra investering. Det är dock ett orimligt scenario, men räknat på dubbla kostnaden och endast 25 \% besparing är Paybacktiden tolv år, vilket är tilltaget i överkant. Rimlig Paybacktid skulle kunna vara runt 5-6 år, med den dyrare varianten av systemet. Det finns också möjlighet att sätta in det billigare, utan färgskärmarna. Det ger halva kostnaden, men ger inte samma mervärde och enkelhet till de boende.

\paragraph{Prognosstyrning från SMHI}
Den stora fördelen med SMHIs system för prognosstyrning är SMHIs prismodell. Den innebär att investeringskostnaden ska kunna betala sig inom två år, samt att abonnemangsavgiften inte är högre än maximalt hälften av vad man sparar varje år. Det är således väldigt intressant eftersom man egentligen inte behöver några ekonomiska muskler för att börja använda produkten – även en förening som redan har mycket lån och nästan går med minus kan köpa systemet. Det bidrar till att samtidigt som SMHI kan ta betalt för sina prognoser, så sparar de boende pengar och mindre energi behöver produceras. SMHI lejer ut installationen till sina sammarbetspartners. Systemet är enkelt och lätt att installera, vilket tillsammans med det låga priset bidrar till att göra det attraktivt.

Med, som bäst, endast ett års payback-tid är prognossytrning det alternativet som verkar bäst enligt Payback-metoden. Det är dock känt att payback-metodern inte alltid hittar de största besparingarna. Skulle föreningen investera i systemet, som man enligt SMHI ska kunna få installerat till ett pris som betalar sig inom ett eller maximalt två år \cite{smhi1}\cite{smhi2}, tjänar man efter paybacktiden inte nödvändigtvis speciellt mycket pengar per år. Det är inte en hög procentsats av energiåtgången som sparas, samtidigt som de tar en abonnemangsavgift uppskattningsvis 20-40 \%. \cite{smhi1}\cite{smhi2}

\paragraph{Väderstation}
Beräkningarna för väderstationen är gjorda utifrån inköp av nödvändigt utrustning. För Bostadsrättsföreningen Wallerius är detta inte riktigt relevant då investeringen redan är gjord. I och med beräkningen finns det ändå en möjlighet för att se hur metoden står sig i förhållande till de andra där främst jämförelsen med prognosstyrning är intressant.
 % Förbättningsåtgärder och ekonomi

\section{Tillämpningar}

De två primära tillämpningarna av det här arbetet är att identifiera energiläckor för att kunna göra sitt hus mer energieffektivt och att låta sitt energiförsörjningssystem vara väderberoende.

Att energieffektivisera sitt hus kan, om man väljer rätt metod, löna sig både ekonomiskt och miljömässigt. Det leder dessutom till mindre fluktuationer i inomhustemperaturen, speciellt om man har stor tröghet i sitt uppvärmningssystem.

Att låta energiförsörjningssystemet bero av väderdata från en vid fastigheten monterad väderstation kräver experimentella mätning på den aktuella fastigheten för att bli implementerbart.
Dessutom kommer direkta eneriflöden, så som solinstrålning och vind, att kompenseras för fördröjt, vilket kan bli ett problem om man har ett långsamt system för uppvärmning. Ett alternativ då är prognosstyrning men de färdiga system som finns på marknaden idag är så pass bra att det inte är av intresse att försöka bygga ett eget, inte ur ekonomisk synvinkel i alla fall.
 % Ska det vara kvar? Ska det stoppas in i någon annan del?

\section{Rekommendationer till fortsatt arbete}
 % Förslag till fortsatt arbete


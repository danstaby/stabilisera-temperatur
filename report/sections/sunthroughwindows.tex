\subsection{Effektflöde genom fönster på grund av solstrålning}

\begin{comment}
Förhållande mellan väderstationens uppmätta värde och tillförd effekt till lägenhet:
- beroende av fönstrets och grannbyggnaders position
- beroende av tid på dygnet och året
- persienner och gardiner försvårar!
- hur mycket kan maximalt reflekteras ut?

I diskussion:
- Hur kan man koppla detta till värmesystemet?
	- Registrera intensitet, tid på dygnet och datum
	- Beräkna ungefärlig tillförd effekt
	- Kompensera genom att säga till värmesystemet att minska/stänga inflödet
- Blir det lättare att helt enkelt mäta temperaturen i rummet och helt enkelt gå utifrån det? Vad är mer kostnadseffektivt?

-Konduktion och konvektion räknas tillsammans med väggar.. 

-Strålning
\end{comment}

Solstrålning genom fönster orsakar snabba temperaturökningar i inomhusklimatet. Hur snabba och stora dessa temperaturökningar blir beror på en mängd parametrar varav de viktigaste omfattas av
\begin{itemize}
\item{
fönstrets utformning, det vill säga glasets reflektivitet och emmissitivitet.
}
\item{
vinkeln relativt fönstret som strålningen infaller vid, det vill säga tid på dagen och året. Detta är starkt förknippat med föregående punkt. 
}
\item{
de inomhus belägna ytorna som solstrålningen faller på, det vill säga persienner, gardiner, väggar, möbler, etcetera.
}
\end{itemize} 

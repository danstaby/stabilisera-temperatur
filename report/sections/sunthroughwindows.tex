\subsection{Effektflöde genom fönster på grund av solstrålning}

Solstrålning genom fönster orsakar snabba temperaturökningar i inomhusklimatet. Hur snabba och stora dessa temperaturökningar blir beror på en mängd parametrar varav de viktigaste omfattas av

\begin{itemize}
\item{
fönstrets utformning, det vill säga glasets reflektivitet och emmissitivitet.
}
\item{
vinkeln relativt fönstret som strålningen infaller vid, det vill säga tid på dagen och året. Detta är starkt förknippat med föregående punkt. 
}
\item{
de inomhus belägna ytorna som solstrålningen faller på, det vill säga persienner, gardiner, väggar, möbler, etcetera.
}
\end{itemize} 

\subsubsection{G-värden}

För att ange transmittansen av solstrålning genom fönster brukar man använda vad som kallas för g-värden (ibland även kallat ``Solar Factor''). Detta värde, mellan noll och ett, anger hur mycket av vinkelrätt infallande solstrålning som släpps igenom. Men eftersom ett sådant värde också beror på strålningens infallsvinkel är den ofta svår att beräkna.

Enligt \cite{karlssonroos99} förändras detta vinkelberoende främst med antalet glas (flerglasfönster) samt typ av eventuella beläggningar på glaset. I samma artikel visades också att g-värdenas vinkelberoende kan approximeras med ett polynom

\begin{equation}\label{eq:radiationwindowstheory:gvalue}
g = g_0 \left( 1 - az^{\alpha} - bz^{\beta} - cz^{\gamma} \right)
\end{equation}

där $g_0$ är g-värdet då strålningen infaller vinkelrätt mot ytan, $a+b+c=1$ och $z=\theta/90$ då $\theta$ är vinkeln, mätt i grader, mellan fönstrets normal och solstrålningens riktning. Koefficienterna och exponenterna i \eqref{eq:radiationwindowstheory:gvalue} beror på typen av fönster, och kan sättas till

\begin{eqnarray}
a = 8, & b = 0.25/q, & c = (1-a-b) \nonumber \\
\alpha = 5.2 + 0.7q, & \beta = 2, & \gamma = (5.26+0.06p) + (0.73+0.04p)q
\end{eqnarray}

där p är antalet rutor i fönstret (treglasfönster $\Rightarrow$ p = 3) och q är en parameter, $1 \le q \le 10$, som varierar beroende på beläggningar på glasets yta. Exempelvis har ett treglasfönster utan beläggningar värdet $q=4$.

Det beräknade g-värdet kan sedan användas för att uppskatta energiflödet genom fönstret. Anta att en pyranometer anger solstrålningsstyrkan $\unit{I\, }{W/m^2}$. Då ges det totala energiflödet Q av sambandet $Q = g \cdot A \cdot I$, där A är fönstrets area.

\subsubsection{Infallsvinkel relativt tid och datum}

Det återstår nu att i \eqref{eq:radiationwindowstheory:gvalue} kunna bestämma parametern $z = \theta/90$. Detta kan göras genom att utgå från aktuellt datum och tid på dygnet.

En metod för att räkna ut solens position presenteras i \cite{walraven78} och en Matlabfunktion baserad på samma artikel kan ses i appendix. % Hänvisa till kod

Om azimuthala och altitudinella vinklarna ($\beta$ respektive $\alpha$) relativt ett väderstreck respektive horisonten tillhandahålls beräknas infallsvinkeln mot glaset, $\theta$, på följande vis:

\begin{equation} 
\theta = \frac{360}{2\pi}\arctan{\left( \sqrt{\tan^2{\left(\alpha\right)}
+ \tan^2{\left(\beta - \gamma \right)}} \right)}
\end{equation}

där $\gamma$ är vinkeln mellan fönstrets normal och väderstrecket mot vilken azimuthala vinkeln anges. Notera att alla vinklar utom $\theta$ anges i radianer.

Ett Matlabprogram för beräkning av effektflödet på grund av solstrålning genom fönster kan ses i appendix. % Hänvisa till appendix.

% Behöver: longitud, latitud, vinkeln relativt väderstreck, fönsters area, g-värde

\subsubsection{Inverkan av skuggor, gardiner och dylikt}

% Beräkna för vilka vinklar skuggor faller över fönstren

% Gardiner, persienner och interiör förändrar situationen, kolla källan nedan
\begin{comment}
Simmler & Binder
Experimental and numerical determination of the total solar energy transmittance of glazing with venetian blind shading
\end{comment}

% Be Särnöe om specifikationer:
% - I vilket väderstreck är normalen riktad?
% - Vilket g-värde har fönstren?

\begin{comment}
I diskussion:
- Hur kan man koppla detta till värmesystemet?
	- Registrera intensitet, tid på dygnet och datum
	- Beräkna ungefärlig tillförd effekt
	- Kompensera genom att säga till värmesystemet att minska/stänga inflödet
- Blir det lättare att helt enkelt mäta temperaturen i rummet och gå utifrån det? Vad är mer kostnadseffektivt?
\end{comment}

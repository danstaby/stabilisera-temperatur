\subsection{Optimering med Newton-Raphsons metod}

När ett ekvationssystem är ickelinjärt kan ej exakta metoder som Gausseliminering användas
för ekvationslösning.
I dessa fall måste approximativa optimeringsmetoder utnyttjas. En sådan
metod är Newton-Raphsons metod. Denna bygger på trunkerad Taylorutveckling 
av en funktion för att linjarisera ett ickelinjärt ekvationssystem
$\mathbf{f}(\mathbf{x}) = 0$
vilket kan ses i ekvation \eqref{eq:newtonsmethod:taylor}. Här är
$\mathbf{J}_f(\mathbf{x})$ jacobianen för $\mathbf{f}(\mathbf{x})$. 

\begin{equation}
\label{eq:newtonsmethod:taylor}
\mathbf{f}(\mathbf{x} + \Delta\mathbf{x}) \approx \mathbf{f}(\mathbf{x}) +
\mathbf{J}_f(\mathbf{x})\Delta\mathbf{x}
\end{equation}

\noindent
Principen går ut på att algoritmen upprepat gissar nya lösningar där de
nya lösningarna följer den negativa jacobianen. Till en början är en god initial gissning
$\mathbf{x}_0$ ett kriterie för att Newton-Raphson skall konvergera. Därefter så beräknas
funktionsvärdet $\mathbf{f}(\mathbf{x}_0)$ samt jacobianen $\mathbf{J}_f(\mathbf{x}_0)$.
Dessa används för att beräkna nästa gissning genom att lösa
\eqref{eq:newtonsmethod:guess} och beräkna nästa $\mathbf{x}$ med
\eqref{eq:newtonsmethod:nextx}. \cite{heath2002}

\begin{equation}
\label{eq:newtonsmethod:guess}
\mathbf{J}_f(\mathbf{x}_n)\Delta\mathbf{x}_n = -\mathbf{f}(\mathbf{x_n})
\end{equation}

\begin{equation}
\label{eq:newtonsmethod:nextx}
\mathbf{x}_{n+1} = \mathbf{x}_n + \Delta\mathbf{x}_n
\end{equation}

\noindent
Itereringen bör avbrytas då ett maxantal itereringar har uppnåtts och funktionen
ej har konvergerat alternativt när felet är tillräckligt litet. En av styrkorna 
med denna algoritm är dess snabba konvergens ($N^2$) mot enkelrötter. \cite{ympa95}
En svaghet med Newton-Raphson är att det i många fall ej är möjligt att analytiskt beräkna
jacobianen. Istället måste andra algoritmer untnyttjas som till exempel finita differensmetoden
för beräkning av jacobianen. Omvägar som denna bidrar till att lösningsprocessen blir mycket
mer omständig och processorintensiv. 

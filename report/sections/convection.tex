\subsection{Konvektion}
\label{section:convection}
I fasta ämnen går det utmärkt att approximera värmeflöde enbart med hjälp
av värmeledningsekvationen. Detta håller dock ej lika bra för fluider, det vill säga
material som deformeras då de utsätts för tryck ett litet tryck.
Under dessa förhållanden
måste det tas hänsyn till konservation av massa, energi samt rörelsemoment.
\\\\\\\\\\
DETTA MÅSTE HÄRLEDAS
\\\\\\\\\\

För homogena inkompressibla fluider i två dimensioner gäller då ekvationerna
\eqref{eq:convection:continuity}, \eqref{eq:convection:momentumx},
\eqref{eq:convection:momentumz} samt \eqref{eq:convection:energy}. Här
är $\mathbf{v} = (u,w)$ hastighetsvektorn, $\alpha$ är den termiska
diffusiviteten, $\nu$ är den kinematiska viskositeten, $p$ är trycket,
$\rho$ är densiteten, $g$ är den lokala tyngdaccelerationen
och slutligen är $\rho_0$ referensdensiteten.

\begin{equation}
\label{eq:convection:continuity}
\nabla\cdot\mathbf{v} = 0
\end{equation}

\begin{equation}
\label{eq:convection:momentumx}
\frac{\partial u}{\partial t} + \mathbf{v}\cdot\nabla u = 
-\frac{1}{\rho_0}\frac{\partial p}{\partial x} + 
\nu\Delta u
\end{equation}

\begin{equation}
\label{eq:convection:momentumz}
\frac{\partial w}{\partial t} + \mathbf{v}\cdot\nabla w = 
-\frac{1}{\rho_0}\frac{\partial p}{\partial z} + \nu\Delta w
\end{equation}

\begin{equation}
\label{eq:convection:energy}
\frac{\partial T}{\partial t} + \mathbf{v}\cdot\nabla T = \alpha\Delta T
\end{equation}

\subsubsection{Boussinesq approximation}

För flytkraftsdrivet flöde kan det vara lämpligt att använda sig av
Boussinesq approximation. Denna säger att det enda som påverkar trycket är
tyngdaccelerationen. Genom detta är det möjligt att sätta upp uttryck för densiteten
och tryckderivatorna enligt ekvationerna \eqref{eq:convection:density}
och \eqref{eq:convection:pressurez}. Här är
$\beta$ den volymetriska expansionskonstanten och
$T_0$ temperaturen som råder vid referensdensiteten $\rho_0$.

\begin{equation}
\label{eq:convection:density}
\rho = \rho_0[1-\beta(T-T_0)]
\end{equation}

\begin{equation}
\label{eq:convection:pressurez}
\frac{\partial p}{\partial z} = -\rho_0g
\end{equation}



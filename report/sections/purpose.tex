\subsection{Syfte}
Vi ska i detta arbete undersöka vilka energiförluster en fastighet har och hur dessa påverkas av olika väderrelaterade parametrar, som solinstrålning, utomhustemperatur och vind.

Arbetet syftar till att finna en kvanitativ beskrivning av hur energiflödena in och ut ur fastigheten påverkas av några olika väderparametrar. 

Detta ska i sin tur kunna leda till en modell för hur fastighetens reglerbara energiflöden ska kunna anpassas efter vädret så att en önskad inomhustemperatur bibehålls. Detta ska inte bara ge en trivsammare inomhusmiljö för de boende, utan även en energibesparing för fastigheten.

Vi avser att i vår modell att ta hänsyn till den fördröjning som sker i fastighetens väggar och den främst bygga på att luften närmast fastigheten värms upp och bildar ett isolerande lager.

På sikt hoppas vi att vårt arbete ska leda fram till ett uttryck som ger ett värde på hur mycket energi som behöver tillföras fastigheten i varje givet ögonblick, beroende på vilka värden väderstationen tidigare har tagit emot.
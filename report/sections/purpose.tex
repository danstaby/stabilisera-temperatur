\subsection{Syfte}
I detta arbete ska vi undersöka vilka energiförluster vi har i en fastighet och hur dessa påverkas av olika väderrelaterade parametrar.

Vi vill finna en modell som visar på hur fastighetens energiflöde ska regleras med avseende på de opåverkbara men föränderliga energiflöden som sker in och ut i fastigheten, till exempel på grund av väder, så att ett önskat inomhusklimat bibehålls. Vi avser då också att ta hänsyn till den fördröjning som sker i och med fastighetens relativt tjocka väggar. Mer konkret söker vi en ekvation som ger oss ett värde på hur mycket energi som behöver tillföras fastigheten i varje givet ögonblick, beroende på vilka värden väderstationen tidigare har tagit emot.

Vi hoppas också att detta ska ge, inte bara en trivsammare inomhusmiljö för de boende, utan även en energibesparing för fastigheten.

\subsubsection{Väder}
I rapporten låter vi de parametrar som väderstationen tar in definiera vädret. Väderstationen som vår uppdragsgivare installerat är en Vaisala Weather Transmitter WXT520. Den mäter sju olika värden: vindens hastighet och riktning, lufttemperaturen, lufttryck, relativ fuktighet samt regn och hagels varaktighet och intensitet.

\begin{table}[htdp]
\caption{Tekniska data för väderstationen, \cite{datasheet_weathertransmitter}}
% Pärmen flik 1.
\begin{center}
\begin{tabular}{|l | l l l|}
\hline
\textbf{Väder} & \textbf{Mätområde} % range
 & \textbf{Precision} % accuracy
 & \textbf{Upplösning} \\ % resolution
\hline
\rule{0pt}{3ex}Vindhastighet & $\unit[0-60]{m/s}$ & $\pm 3-5\%$ & $\unit[0,1]{m/s}$ \\ 
\rule{0pt}{3ex}Vindriktning & alla riktningar & $\pm 3^{\circ}$ & $1^{\circ}$ \\
\rule{0pt}{3ex}Temperatur & $\unit[-52-+60]{^{\circ}C}$ & & $\unit[0,1]{^{\circ}C}$ \\
\rule{0pt}{3ex}Lufttryck & $\unit[600-1100]{hPa}$ & $\unit[0,5-1]{hPa}$ & $\unit[0,1]{hPa}$ \\
\rule{0pt}{3ex}Luftfuktighet & $\unit[0-100]{\%RH}$ & $\unit[\pm3–5]{\%RH}$ & $\unit[0,1]{\%RH}$ \\
\rule{0pt}{3ex}Regn &  & $\unit[5]{\%}$ & \unit[0,01]{mm} \\
~varaktighet & & & $\unit[10]{s}$\\
~intensitet & $\unit[0-200]{mm/h}$ & & $\unit[0,1]{mm/h}$ \\
\rule{0pt}{3ex}Hagel &  &  & 0,1 träffar/$\unit{cm^2}$ \\
~varaktighet & & från första träffen & 10 s\\
~intensitet & & & 0,1 träffar/$\unit{cm^2h}$\\
\hline
\end{tabular}
\end{center}
\label{default}
\end{table}

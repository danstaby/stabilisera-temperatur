\section{Syfte}
Detta arbete syftar till att undersöka vilka energiförluster en fastighet har och hur dessa 
påverkas av olika väderrelaterade parametrar, som solinstrålning, utomhustemperatur 
och vind. Målet är att finna en kvantitativ beskrivning av hur energiflödena in och ut ur 
fastigheten påverkas av några olika väderparametrar. Eftersom vi inte har tillgång till data
 från fastigheten sker detta med analytiska och numeriska metoder baserade på 
väldokumenterade fysikaliska principer och statistik från SMHI. Primärt kommer vi att 
undersöka en fastighet på Walleriusgatan i Göteborg, men många av resultaten kommer 
att kunna appliceras även på andra byggnader.

Beskrivningen av hur vädret påverkar energiflödena skall sedan kunna leda till en modell 
för hur fastighetens reglerbara energiflöden ska kunna anpassas efter vädret så att en 
önskad inomhustemperatur bibehålls. I fastigheten som är associerad med projektet finns
 en väderstation monterad och det är ifrån den väderdata är tänkt att hämtas. Detta ska 
 inte bara ge en trivsammare inomhusmiljö för de boende, utan även en energibesparing 
 för fastigheten. Vi hoppas också kunna ge några byggnadstekniska förslag på åtgärder 
 och kvantifiera hur stora besparingar detta skulle kunna ge – både energimässiga och 
 ekonomiska.

Vi avser att bland annat bygga vår modell på att luften närmast fastigheten värms upp och 
bildar ett isolerande lager men också på värmeledning och värmestrålning. Vi kommer 
även att ta hänsyn till både den fördröjning som 
sker i fastighetens väggar och den direkta uppvärmning och avkylning som sker med 
solinstrålning genom fönster respektive ofrivillig ventilation i form av drag.

På sikt hoppas vi att vårt arbete ska leda fram till ett modell som ger ett värde på hur mycket energi som behöver tillföras fastigheten i varje givet ögonblick, beroende på vilka värden väderstationen tidigare har tagit emot. I förlängningen ska detta kunna leda till energibesparande åtgärder för både den här och äldre fastigheter.

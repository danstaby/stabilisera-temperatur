\section{Jämförelse med andra energibesparande åtgärder}

En övergripande likhet mellan andra energibesparande åtgärder är att många av dem kostar pengar. Innan det går att överväga något av dem måste man veta vilka biverkningar olika system ger samt hur de kan utnyttjas maximalt. De finns lösningar som lämpar sig bättre och sämre för specifika byggnader. I det här avsnittet kommer en diskussion föras angående fyra alternativ till ett momentant väderbaserat reglersystem.
Som vi kan se i resultatet påverkar vädret energiflödet ut genom byggnaden högst väsentligt. Enligt [Referens] beror det mest på vind och solinstrålning och dess effekter fördröjs inte av trögheten i väggarna, vilket var en av utgångspunkterna för vårt. Solinstrålningen går främst genom fönster, och vinden går in i otätheter och för på så sätt in kallare luft i byggnaden utan fördröjning.

\subsection{Prognosstyrning}
Ett steg ytterligare i att styra inomhustemperaturen efter vädret är så kallad prognosstyrning, vilket skulle kunna vara ett alternativ för byggnaden vi har studerat.
Fördelarna med prognosstyrning kontra metoden att styra direkt mot en väderstation är att de flesta markanta väderpåverkningarna är momentana, det vill säga att det inte sker någon fördröjning innan de kommer in i byggnaden. Då krävs framförhållning för att kunna ta hänsyn till dessa, främst vind och solinstrålning.

SMHI antyder att det kan ge kostnadsbesparingar på 5-10 procent på uppvärmningen, och att förutsäga exakt hur mycket är väldigt svårt, då deras modell angående vilka parametrar styrningen beror på är kommersiell och således inte delges allmänheten.  Antydningar från mötet med SMHI menade på att den här fastigheten inte låg i den övre delen av skalan, men det beror ju helt på hur olika parametrar viktas.

\paragraph{Ekonomi}
Den stora fördelen med det här systemet är SMHIs prismodell. Den innebär att investeringskostnaden ska kunna betala sig inom två år, samt att abonnemangsavgiften inte är högre än cirka hälften av vad man sparar varje år. Det gör att båda går vinnande ur situationen.
\subsection{Tilläggsisolering}
Våra beräkningar visar att väggen inte absorberar tillräckligt mycket energi för att det ska vara värt att inte isolera den.  [Dan länka till resultat]
När man gjorde en omfattande renovering i slutet av 1980-talet gjorde man en yttre tilläggsisolering, med tio centimeter mineralull på den norra sidan[Länka till inledning]. Det gav en ungefärlig faktor ¼ på U-värdet, vilket beskriver hur mycket effekt som går genom väggen. Det arbetet är gjort på knappt hälften av fastighetens ytterfasad, hela norrsidan, men man har inte gjort motsvarande arbete söderut eller på den övre delen av den västra väggen. [Länka till resultat]

Då det finns en stor yta på huset som rent teoretiskt skulle kunna få ett bättre U-värde, det handlar om ca XXX m2, vilket är YYY procent av ytan runt där enerig kan ledas ut. [Länka till inledning].
En tilläggsisolering kan göras på två olika sätt, inifrån eller utifrån. Med båda metoderna finns för samt nackdelar. Båda metoderna medför mycket jobb. Med siffrorna vi har fått fram är det inte säkert att energibesparingarna täcker investeringskostnaderna. Det finns dock mervärden att ta under beaktande. Fasader behöver ibland renoveras, och i samband med ett sånt här projekt får man en renovering, fasaden som finns är dock i tegel så behovet kan ifrågasättas. Det största mervärdet ur vår synvinkel, vilket också framgick vid beställning att temperaturerna i bostäderna blir mycket jämnare [Referens från resultat]. Det beror på, vilket nämns tidigare i kapitlet att fasaden inte fungerar som den buffert eller ”element” som man tror att den gör.
\subsubsection{Tillägsisolering utifrån}
En tilläggsisolering utifrån är en stor kostnad. Områdena som bedöms vara lämpliga att tilläggsisolera är alla väggar med tegelyta, och vi har i [Referens] beräknat det genomsnittliga U-värdet per kvadratmeter för både fallet utan ytterligare isolering samt ett fall med isolering av alla tegelytor. Det handlar om skillnader i det genomsnittliga U-värdet för hela huset på ZZ procent[Referens]
\subsubsection{Tillägsisolering inifrån}
En tilläggsisolering inifrån blir inte lika komplett som att isolera utifrån, man kan helt enkelt inte isolera lika stor yta, vilket innebär att inte hela den beräknade ytan får ett bättre U-värde. En isolering inifrån är troligtvis billigare.
\subsection{Termostater på element}
Styrning av rumstemperatur via termostater på element är en ide som har diskuterats. I dagsläget regleras flödet manuellt via vred under elementen. De ställs in av fastighetsskötaren och kan regleras om det brukar vara för kallt eller för varmt i lägenheten. 
Elektriska termostater finns i olika varianter, men det är främst en ett koncept från Danfoss vi tittat på. Det finns i två varianter vilka ska gå att implementera på i princip alla befintliga uppvärmningssystem. Man ställer in temperaturen man vill ha i grader, antingen i det billigare systemet direkt på termostaterna via en lite lcd-display, eller i det dyrare systemet som fungerar trådlöst mot en huvudenhet med en större färgdisplay. 
Båda systemen kan bryta tillförseln vid tillfälliga kyltoppar, till exempel vid vädring. Det går genom systemet att hålla lägre temperatur i vissa rum, till exempel sovrum. Systemen kan programmeras att ta hänsyn både till solinstrålning genom fönster samt sänka dag, natte eller semestertid, när ingen är hemma.
\subsubsection{Sidoeffekt}
En bieffekt att energianvändningen kan öka om det finns tillräcklig med energi i systemet och de boende vill ha varmare än vad som i dagsläget erbjuds. Det är lätt att begränsa temperaturintervallet, men det kan skapa irritationer om de boende upplever fel temperatur och inte kan ändra den när de har fått ett fint system för det.
\paragraph{Ekonomi}
Det är tveksamt vilken längd man skulle kunna tänka sig på en avskrivning av det här slaget, men för att inte tro på något overkligt är en rimlig avskrivningstid fem år då det innefattar datorbaserade system.
Framtida ideer
Adaptiva algoritmer som förändrar regulatorns styrlagar. Kräver väldigt mycket data. Kräver givare från varje rum med 

http://www.viivilla.se/Energi/Radiator/Ratt-radiator-och-termostat-45410
 



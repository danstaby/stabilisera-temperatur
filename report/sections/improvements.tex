\section{Jämförelse med andra energibesparande åtgärder}

En övergripande likhet mellan andra energibesparande åtgärder är att många av dem kostar pengar. Innan det går att överväga något av dem måste man veta vilka biverkningar olika system ger samt hur de kan utnyttjas maximalt. De finns lösningar som lämpar sig bättre och sämre för specifika byggnader. I det här avsnittet kommer en diskussion föras angående fyra alternativ till ett momentant väderbaserat reglersystem.
Som vi kan se i resultatet påverkar vädret energiflödet ut genom byggnaden högst väsentligt. Enligt [Referens] beror det mest på vind och solinstrålning och dess effekter fördröjs inte av trögheten i väggarna, vilket var en av utgångspunkterna för vårt. Solinstrålningen går främst genom fönster, och vinden går in i otätheter och för på så sätt in kallare luft i byggnaden utan fördröjning.

\paragraph{Besparingar}
Under rubrikerna besparingar kommer det presenteras . Blow door [Resultat ylva]

\subsection{Prognosstyrning}
Ett steg ytterligare i att styra inomhustemperaturen efter vädret är så kallad prognosstyrning, vilket skulle kunna vara ett alternativ för byggnaden vi har studerat.
Fördelarna med prognosstyrning kontra metoden att styra direkt mot en väderstation är att de flesta markanta väderpåverkningarna är momentana, det vill säga att det inte sker någon fördröjning innan de kommer in i byggnaden. Då krävs framförhållning för att kunna ta hänsyn till dessa, främst vind och solinstrålning.

\paragraph{Besparingar}
SMHI antyder att det kan ge kostnadsbesparingar på 5-10 procent på uppvärmningen, och att förutsäga exakt hur mycket är väldigt svårt, då deras modell angående vilka parametrar styrningen beror på är kommersiell och således inte delges allmänheten.  Antydningar från mötet med SMHI menade på att den här fastigheten inte låg i den övre delen av skalan, men det beror ju helt på hur olika parametrar viktas.

\paragraph{Ekonomi}
Den stora fördelen med det här systemet är SMHIs prismodell. Den innebär att investeringskostnaden ska kunna betala sig inom två år, samt att abonnemangsavgiften inte är högre än maximalt hälften av vad man sparar varje år. Det bidrar till att samtidigt som SMHI kan ta betalt för sina prognoser, så sparar de boende pengar och mindre energi behöver produceras.


Tilläggsisolering är en annat förlag, utanpåliggande anses falla lite på bristande behov av renovering av tegel. "fasaden som finns är dock i tegel så behovet kan ifrågasättas". Även tegel behöver renoveras eller bytas då och då, vi hade ett kandidatarbete på SriLanka för något år sedan det visade sig att modernt svenskt tegel håller 50år, danskt 100år och lankesiskt minst 4000år (dvs det skapades 
tegelbyggnader 2000fkr som fortfarande står oförändrade) Så om inget har gjort sedan väggen byggdes så hade det nog varit dags nu. Men vägen ska vara omfogad och impregnerad för 10-20 år sedan, så den håller nog ca 50 år till 


\subsection{Tilläggsisolering}
Våra beräkningar visar att väggen inte absorberar tillräckligt mycket energi för att det ska vara värt att inte isolera den.  [Dan länka till resultat]
När man gjorde en omfattande renovering i slutet av 1980-talet gjorde man en yttre tilläggsisolering, med tio centimeter mineralull på den norra sidan[Länka till inledning], innebärande 26 procent av husets yta utåt[Fotnot, ej grunden inräknad]. Isoleringen gav en ungefärlig faktor ¼ på U-värdet. [Länka till tabellen]Innan isolering ZZ till BB efter isoleringen. Motsvarande arbete är inte gjort på varken syd eller västväggen, varför det finns potentiella förbättringar att hämta inom området.

En isolering av både syd samt västväggen är en isolering av en yta om 212 m2, som rent teoretiskt skulle få ett bättre U-värde. Det innebär 21,8 procent av ytan på fastigheten där energi kan ledas ut. [Länka till inledning].
En tilläggsisolering kan göras på två olika sätt, inifrån eller utifrån. Med båda metoderna finns för samt nackdelar. Båda metoderna innebär betydande insatser i fastigheten. Enligt [figur?]  är det tveksamt att energibesparingarna täcker investeringskostnaderna. Det finns dock mervärden att ta under beaktande. 

Fasader behöver med jämna mellanrum renoveras, och i samband med ett projekt av de proportionerna får man en fasadrenovering. Svenskt tegel har en ungefärlig livslängd på 50 år, dock ska tegelfasaden ha renoverats med ny impregnering i samband med renoveringen -88. Teglet i fastigheten har troligtvis även en bättre livslängd än 50 år, kvaliteten när huset byggdes motsvarar det danska teglet, och då handlar det om cirka 100 år. [Personlig kontakt, Magnus Karlsteen, 5/5-2012]. Det största mervärdet ur vår synvinkel, vilket också framgick vid beställning att temperaturerna i bostäderna blir mycket jämnare [Referens från resultat]. Det beror på, vilket nämns tidigare i kapitlet att fasaden inte fungerar som den buffert eller ”element” som man tror att den gör.

\paragraph{Tilläggsisolering utifrån}
En tilläggsisolering utifrån är ett ingrepp som medför en stor kostnad. Områdena som bedöms vara lämpliga att tilläggsisolera är alla väggar med tegelyta, då de inte har någon tilläggsisolering sedan tidigare och U-värdet skulle då kunna förändras på sammma sätt som norrsidan när den isolerades vid renoveringen. Då det är en betydande investering måste mervärden och andra kostnader som kan tänkas uppstå tas under beaktande. Fasader behöver med jämna mellanrum isoleras. beräknat det genomsnittliga U-värdet per kvadratmeter för både fallet utan ytterligare isolering samt ett fall med isolering av alla tegelytor. Det handlar om skillnader i det genomsnittliga U-värdet för hela huset på ZZ procent[Referens]

\paragraph{Tilläggsisolering inifrån}
En tilläggsisolering inifrån innebär ingrepp i lägenheterna, bland de boende. Det kan medföra komplikationer med personer som inte vill göra lägenheten mindre, vilket är en mindre bieffekt av projektet. Hurvida de boende behöver kompenseras för ingreppet genom dubbelt boende eller ekonomiskt är inte klarlagt, men det behövs ta ställning till. Ingreppet blir inte lika komplett som att isolera utifrån, man kan helt enkelt inte isolera lika stor yta då innerväggar samt golvplan ansluter mot ytterväggen och bildar köldbryggor utåt. Summering ger att den beräknade ytan att isolera blir mindre, nedåt endast hälften mot den tidigare angivna samt att en isolering inifrån troligtvis blir mindre kostsam, men undersökningar har inte gjorts.

\paragraph{Besparingar}
En utvändig tilläggsisolering kan i bästa fall, 

\paragraph{Ekonomi}

\subsection{Termostater på element}
Styrning av rumstemperatur via termostater på element är en ide som har diskuterats. I dagsläget regleras flödet manuellt via vred under elementen. De ställs in av fastighetsskötaren och kan regleras om det brukar vara för kallt eller för varmt i lägenheten. 
Elektriska termostater finns i olika varianter, men det är främst en ett koncept från Danfoss vi tittat på. Det finns i två varianter vilka ska gå att implementera på i princip alla befintliga uppvärmningssystem. Man ställer in temperaturen man vill ha i grader, antingen i det billigare systemet direkt på termostaterna via en lite lcd-display, eller i det dyrare systemet som fungerar trådlöst mot en huvudenhet med en större färgdisplay.
Båda systemen kan bryta tillförseln vid tillfälliga kyltoppar, till exempel vid vädring. Det går genom systemet att hålla lägre temperatur i vissa rum, till exempel sovrum. Systemen kan programmeras att ta hänsyn både till solinstrålning genom fönster samt sänka dag, natte eller semestertid, när ingen är hemma.

\paragraph{Sidoeffekt}
En bieffekt att energianvändningen kan öka om det finns tillräcklig med energi i systemet och de boende vill ha varmare än vad som i dagsläget erbjuds. Det är lätt att begränsa temperaturintervallet, men det kan skapa irritationer om de boende upplever fel temperatur och inte kan ändra den när de har fått ett fint system för det.

\paragraph{Ekonomi}
Det är tveksamt vilken längd man skulle kunna tänka sig på en avskrivning av det här slaget, men för att inte tro på något overkligt är en rimlig avskrivningstid fem år då det innefattar datorbaserade system.

\subsection{Miljöinformation}
I samband med ett system där de boende själva kan reglera temperaturen kan man behöva informera de boende på ett attraktivt sätt, meningen är ju att både kostnaden ska sjunka, samt att miljöpåverkan ska minskas. Börjar temperaturen då smyghöjas i lägenheterna blir så inte fallet, det börjar kosta mer i el, samt förslitning av pumpar m.m.

 
http://www.viivilla.se/Energi/Radiator/Ratt-radiator-och-termostat-45410
 



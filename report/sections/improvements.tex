\section{Jämförelse med andra energibesparande åtgärder}

En övergripande likhet mellan andra energibesparande åtgärder är att många av dem kostar pengar. Innan det går att överväga något av dem måste man veta vilka biverkningar olika system ger samt hur de kan utnyttjas maximalt. De finns lösningar som lämpar sig bättre och sämre för specifika byggnader. I det här avsnittet kommer en diskussion föras angående fyra alternativ till ett momentant väderbaserat reglersystem.
Som vi kan se i resultatet påverkar vädret energiflödet ut genom byggnaden högst väsentligt. Enligt [Referens] beror det mest på vind och solinstrålning och dess effekter fördröjs inte av trögheten i väggarna, vilket var en av utgångspunkterna för vårt. Solinstrålningen går främst genom fönster, och vinden går in i otätheter och för på så sätt in kallare luft i byggnaden utan fördröjning.

\paragraph{Besparingar}
Under rubrikerna besparingar kommer det presenteras vilka olika former av besparingar, effekter eller andra fördelar man får av att investera i den presenterade energisparformen. All energi som används till bostadsuppvärmning enligt \cite{energideklaration} antas ligga under vinterhalvåret, då uppvärming förekommer. Det antas inte försvinna energi till kyla under sommarhalvåret.  [Resultat ylva].

\paragraph{Ekonomi}
Under rubriken kommer förutsättningar i ekonomin beskrivas. Den är väldigt approximativ då vi inte har haft befogenheter att begära in riktiga offerter från bygg samt installationsföretag, då detta kostar dem pengar, i form av tid, samt att vi inte ens haft för avsikt att använda offerten, vilket är allmänt ojuste.

\subsection{Prognosstyrning}
Prognosstyrning är en möjligheten för att styra inomhustemperaturen efter vädret, vilket skulle kunna vara ett alternativ för byggnaden vi har studerat.
Fördelarna med prognosstyrning kontra att styra direkt mot väderstationen är att de flesta markanta väderpåverkningarna är momentana, det vill säga att det inte sker någon fördröjning innan de kommer in i byggnaden. Då krävs framförhållning för att kunna ta hänsyn till dessa, vilka är främst vind och solinstrålning, vilket kan ges av prognosstyrning. Sveriges meterologiska och hydrologiska institut, SMHI står för prognoserna som skickas till systemet, vilket i sin tur har installerats av deras samarbetspartner. En prognos skickas egentligen aldrig, utan det som skickas är en styrsignal som baseras på väderprognoser samt data om byggnadens energibalans som skickas till styrsystemet, som sedna försöker optimera energiåtgång och boendekomfort efter givna värden.

\paragraph{Besparingar}
SMHI påstår att det kan ge kostnadsbesparingar på 5-10 procent på uppvärmningen, och att förutsäga exakt hur mycket är väldigt svårt, då deras modell angående vilka parametrar styrningen beror på är kommersiell och således inte delges allmänheten. Antydningar från mötet med SMHI menade på att den här fastigheten inte låg i den övre delen av skalan, men det beror ju helt på hur olika parametrar viktas, och en korrektare bedömning kräver troligtvis att man har för avsikt att köpa det.

\paragraph{Ekonomi}
Den stora fördelen med det här systemet är SMHIs prismodell. Den innebär att investeringskostnaden ska kunna betala sig inom två år, samt att abonnemangsavgiften inte är högre än maximalt hälften av vad man sparar varje år. Det är således väldigt intressant då man egentligen inte behöver några ekonomiska muskler för att börja använda produkten, även en förening som redan har mycket lån och nästan går med minus kan köpa systemet, för det blir billigare i vilket fall. Det bidrar till att samtidigt som SMHI kan ta betalt för sina prognoser, så sparar de boende pengar och mindre energi behöver produceras. SMHI lejer ut installationen till sina samarbetspartners. Systemet är enkelt och lätt att installera, vilket tillsammans med det låga priset bidrar till att göra det attraktivt.

\subsection{Tilläggsisolering}
Våra beräkningar visar att väggen inte absorberar tillräckligt mycket energi för att det ska vara värt att inte isolera den. Se figurer i avsnitt \ref{sec:steadystatewall}.
När man gjorde en omfattande renovering i slutet av 1980-talet byggde man en yttre tilläggsisolering, med tio centimeter mineralull på den norra sidan, \cite{arsredovisning} innebärande 26 procent av husets yta utåt, grunden borträknat. Isoleringen gav en ungefärlig faktor ¼ på U-värdet.Vi ser här det låga u-värdet på norrväggen kontra det fortfarande höga på sydsidan. \ref{tbl:uvalue} Dessa differenser ger oss möjligheten att göra förbättringar där det inte är tillräckligt isolerat, det vill säga till exempel där U-värdet ligger över ett.

En isolering av både syd samt västväggen är en isolering av en yta om 212 m2, som rent teoretiskt skulle få ett bättre U-värde. Det innebär 21,8 procent av ytan på fastigheten där energi kan ledas ut. \ref{tbl:uvalue}.
En tilläggsisolering kan göras på två olika sätt, inifrån eller utifrån. Med båda metoderna finns för samt nackdelar. Båda metoderna innebär betydande insatser i fastigheten. Det finns dock mervärden att ta under beaktande. 

\paragraph{Besparingar}
Fasader behöver med jämna mellanrum renoveras, och i samband med ett projekt av de proportionerna får man en fasadrenovering. Svenskt tegel har en ungefärlig livslängd på 50 år\cite{Magnus}, dock ska tegelfasaden ha renoverats med ny impregnering i samband med renoveringen -88. Teglet i fastigheten har troligtvis även en bättre livslängd än 50 år, kvaliteten när huset byggdes motsvarar det danska teglet, och då handlar det om cirka 100 år. [Personlig kontakt, Magnus Karlsteen, 5/5-2012]. Det största mervärdet ur vår synvinkel, vilket också framgick vid beställning att temperaturerna i bostäderna blir mycket jämnare [Referens från resultat]. Det beror på, vilket nämns tidigare i kapitlet att fasaden inte fungerar som den buffert eller ”element” som man tror att den gör.

\paragraph{Tilläggsisolering utifrån}
En tilläggsisolering utifrån är ett ingrepp som medför en stor kostnad. Områdena som bedöms vara lämpliga att tilläggsisolera är alla väggar med tegelyta, då de inte har någon tilläggsisolering sedan tidigare och U-värdet skulle då kunna förändras på sammma sätt som norrsidan när den isolerades vid renoveringen. Då det är en betydande investering måste mervärden och andra kostnader som kan tänkas uppstå tas under beaktande. Fasader behöver med jämna mellanrum isoleras. beräknat det genomsnittliga U-värdet per kvadratmeter för både fallet utan ytterligare isolering samt ett fall med isolering av alla tegelytor. Det handlar om skillnader i det genomsnittliga U-värdet för hela huset på ZZ procent[Referens]

\paragraph{Tilläggsisolering inifrån}
En tilläggsisolering inifrån innebär ingrepp i lägenheterna, bland de boende. Det kan medföra komplikationer med personer som inte vill göra lägenheten mindre, vilket är en mindre bieffekt av projektet. Hurvida de boende behöver kompenseras för ingreppet genom dubbelt boende eller ekonomiskt är inte klarlagt, men det behövs ta ställning till. Ingreppet blir inte lika komplett som att isolera utifrån, man kan helt enkelt inte isolera lika stor yta då innerväggar samt golvplan ansluter mot ytterväggen och bildar köldbryggor utåt. Summering ger att den beräknade ytan att isolera blir mindre, nedåt endast hälften mot den tidigare angivna samt att en isolering inifrån troligtvis blir mindre kostsam, men undersökningar har inte gjorts.

\paragraph{Ekonomi}
Vi har inte räknat på en isolering inifrån

\subsection{Termostater på element}
Styrning av rumstemperatur via termostater på element, är den tredje åtgärden att diskutera. I dagsläget regleras flödet manuellt via vred under elementen. De ställs in av fastighetsskötaren och kan regleras upp eller ner om bostadsrättsinnehavaren är missnöjd med inomhusklimatet. Termostater finns både elektriska samt 
Elektriska termostater finns i olika varianter, men det är främst en ett koncept från Danfoss vi tittat på. Det finns i två varianter vilka ska gå att implementera på i princip alla befintliga uppvärmningssystem. Man ställer in temperaturen man vill ha i grader, antingen i det billigare systemet direkt på termostaterna via en lite lcd-display, eller i det dyrare systemet som fungerar trådlöst mot en huvudenhet med en större färgdisplay.
Båda systemen kan bryta tillförseln vid tillfälliga kyltoppar, till exempel vid vädring. Det går genom systemet att hålla lägre temperatur i vissa rum, till exempel sovrum. Systemen kan programmeras att ta hänsyn både till solinstrålning genom fönster samt sänka dag, natte eller semestertid, när ingen är hemma.

\paragraph{Sidoeffekt}
En bieffekt att energianvändningen kan öka om det finns tillräcklig med energi i systemet och de boende vill ha varmare än vad som i dagsläget erbjuds. Det är lätt att begränsa temperaturintervallet, men det kan skapa irritationer om de boende upplever fel temperatur och inte kan ändra den när de har fått ett fint system för det.

\paragraph{Ekonomi}
Det är tveksamt vilken längd man skulle kunna tänka sig på en avskrivning av det här slaget, men en  rimlig avskrivningstid är fem år då det innefattar ett datorbaserat system. I kostnaden har endast materialet tagits med, det vill säga termostater samt huvudenheter. 

\subsection{Miljöinformation}
I samband med ett system där de boende själva kan reglera temperaturen kan man behöva informera de boende på ett attraktivt sätt, meningen är ju att både kostnaden ska sjunka, samt att miljöpåverkan ska minskas. Börjar temperaturen då smyghöjas i lägenheterna blir så inte fallet, det börjar kosta mer i el, samt förslitning av pumpar m.m. 

\cite{viivilla}
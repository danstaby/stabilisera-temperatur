\section{Jämförelse med andra energibesparande åtgärder}

Det finns flera olika energibesparande åtgärder att överväga. Innan något av alternativen implementeras måste det klargöras vilka kostnader, mervärden och sidoeffekter olika åtgärder ger samt hur de kan utnyttjas maximalt. De finns lösningar som lämpar sig bättre och sämre för specifika byggnader. I det här avsnittet kommer en diskussion föras angående tre alternativ till det momentant väderbaserade system som rapporten i huvudsak undersöker. 
Som kan ses i resultatet påverkar vädret högst väsentligt energiflödet genom byggnaden. Enligt avsnitt~\ref{resultsfreerunning} beror det främst på vind och solinstrålning och deras effekter fördröjs inte av trögheten i väggarna. För att kunna reglera enbart på data från väderstationen behöver det finnas en fördröjning innan vädret påverar inomhusklimatet, för att ett reglersystem ska hinna med. Solinstrålningen värmer direkt genom fönster, och vinden går in igenom otätheter och för på så sätt in kallare luft i byggnaden utan fördröjning.

All energi som används till bostadsuppvärmning enligt fastighetens energideklaration\cite{energideklaration} antas i följande jämförelse förbrukas under vinterhalvåret, då uppvärmning förekommer. Det antas inte försvinna energi till kyla under sommarhalvåret, då fastigheten idag inte har något kylsystem.

\subsection{Tilläggsisolering}
Våra beräkningar visar att väggen inte absorberar tillräckligt mycket energi för att det ska löna sig att inte isolera den. Se figurer i avsnitt~\ref{sec:steadystatewall}.
När man gjorde en omfattande renovering i slutet av 1980-talet utfördes en yttre tilläggsisolering, med tio centimeter mineralull på den norra sidan\cite{arsredovisning}, som motsvarar 26~\% av husets klimatskal, grunden borträknad. Isoleringen sänkte U-värdet med en fjärdedel. Norrväggen fick då ett lägre U-värde, medan syd- och västväggarna fortfarande hade ett relativt högt, se tabell \ref{tbl:uvalue}.
Dessa differenser belyser var byggnaden är dåligt isolerad och det är lämpligt att göra förbättningar, ett exempel är att då välja ytor där U-värdet är högre än 1,0.

En isolering av både syd- och västväggarna är en isolering av en yta om 212 $\unit{m^2}$, som genom en isolering skulle få ett bättre U-värde. Det motsvarar 21,8~\% av fastighetens yta där energi kan ledas ut, se tabell \ref{tbl:uvalue}.
En tilläggsisolering kan göras på två olika sätt, inifrån eller utifrån. Med båda metoderna finns det både för- och nackdelar. Båda metoderna innebär också betydande ingrepp i fastigheten. Det finns dock mervärden att ta under beaktande. 

\paragraph{Sidoeffekter och besparingar}
Fasader behöver med jämna mellanrum renoveras, och i samband med en sådan renovering blir det betydligt billigare att samtidigt tillägsisolera. Svenskt tegel har en ungefärlig livslängd på 50 år\cite{magnus}, dock ska tegelfasaden ha renoverats med ny impregnering i samband med renoveringen 1988. Teglet i fastigheten har troligtvis även en längre livslängd än 50 år,  då tegel höll högre kvalitetet när huset byggde uppskattas livslängden till ungefär 100 år.
Det främsta mervärdet, som också initierade det här projektet, är att temperaturerna i bostäderna blir jämnare, se figur~\ref{fig:energyflow_stst} och \ref{fig:wall_dec}. 
Det beror på att väggens buffrande egenskaper är inte så stora som man vill tro och värme sparas inte i väggen från den varmare till den kallare delen av dygnet i önskad utsträckning.

\paragraph{Tilläggsisolering utifrån}
En tilläggsisolering utifrån är ett ingrepp som medför en stor kostnad. Områdena som bedöms vara lämpliga att tilläggsisolera är alla väggar med tegelyta då de inte har någon tilläggsisolering sedan tidigare och U-värdet då skulle kunna sänkas på samma sätt som norrsidan när den isolerades. U-värdet för hela fastigheten skulle då kunna sänkas från 0,6 till 0,4 \ref{tbl:uvalue}. Då tilläggsiolering är en betydande investering måste emellertid mervärden och andra kostnader som kan tänkas uppstå tas under beaktande. Då alla fasader med jämna mellanrum behöver renoveras kan det vara lämpligt att genomföra isoleringen i samband med detta. Vissa kostnader skulle då kunna minskas, till exempel den för uppsättande av byggnadsställningar.

\paragraph{Tilläggsisolering inifrån}
En tilläggsisolering inifrån innebär ett ingrepp i lägenheterna och förutom den direkta nackdelen att det stör dem som bor där medför det också att lägenheterna blir något mindre. Huruvida de boende behöver kompenseras ekonomiskt eller i form av alternativt boende i samband med renoveringen har inte tagits ställning till. Vidare blir ingreppet inte lika komplett som vid isolering utifrån, eftersom det inte går att isolera där innerväggar och golvplan ansluter till ytterväggen och bildar köldbryggor utåt. Den beräknade ytan att isolera blir drygt hälften jämfört med då man isolerar utifrån. Detta gör troligen själva isoleringen mindre kostsam, men det har inte undersökts närmare.

\subsection{Termostater på radiatorererna}
Den tredje åtgärden som undersökts är styrning av rumstemperatur via termostater på radiatorerna. I dagsläget regleras flödet manuellt via vred under elementen. De ställs in av fastighetsskötaren och kan regleras om bostadsrättsinnehavaren är missnöjd med inomhusklimatet. Elektriska termostater finns i olika varianter, men det är främst ett koncept från Danfoss AB som undersökts. Det finns i två varianter vilka går att implementera på i princip alla befintliga uppvärmningssystem. Temperatur ställs in i önskat antal grader, i det billigare systemet direkt på termostaterna via en liten lcd-display, eller i det dyrare systemet med en större färgdisplay som är ansluten trådlöst till en huvudenhet. Båda systemen kan automatiskt bryta tillförseln av energi vid tillfälliga kyltoppar, till exempel vid vädring. Det går genom systemet att hålla lägre temperatur i vissa rum, till exempel sovrum. Systemen kan programmeras att ta hänsyn till solinstrålning genom fönster och att sänka temperaturen under dagen, natten eller semestern. Enligt resultaten i avsnitt~\ref{resultsfreerunning} vet vi att kompensering för solinstrålningen kan ge en stor energibesparing.

\paragraph{Sidoeffekter och besparingar}
En potentiell negativ bieffekt är att energianvändningen kan öka om det finns tillräcklig med energi i systemet och de boende vill ha varmare än vad som i dagsläget erbjuds, vilket leder till att det går åt mer energi än tidigare. Det är lätt att begränsa temperaturintervallet för alla, men det kan skapa irritation hos de boende om de upplever fel temperatur och inte kan ändra, när systemet ska klara det. I samband med att ett system där de boende själva kan reglera temperaturen implementeras bör man informera de boende på ett attraktivt sätt – meningen är ju att både kostnaden ska sjunka, samt att miljöpåverkan ska minskas. Höjs då temperaturen i lägenheterna blir så inte fallet utan det börjar istället kosta mer i både uppvärmning och förslitning av pumpar. \cite{viivilla}

\subsection{Prognosstyrning}
Prognosstyrning är en möjlighet för att styra inomhustemperaturen efter vädret, vilket skulle kunna vara ett alternativ till styrning efter väderstationen för byggnaden som har studerats.
Fördelarna med prognosstyrning jämfört med att styra direkt mot väderstationen är att de flesta markanta väderpåverkningar är momentana, det vill säga att det inte sker någon fördröjning innan de påverkar inomhusklimatet. Då krävs framförhållning för att kunna ta hänsyn till dessa vilket kan ges av prognosstyrning. Sveriges meteorologiska och hydrologiska institut, SMHI, står för prognoserna som behandlas och skickas till fastighetens reglersystemet, vilket i sin tur har installerats av deras samarbetspartner. En prognos skickas egentligen aldrig, utan det som skickas är en styrsignal som baseras på väderprognoser samt data om byggnadens energibalans som skickas till styrsystemet. Styrsystemet försöker sedan optimera energiåtgång och boendekomfort efter de givna värdena.

\paragraph{Sidoeffekter och besparingar}
SMHI påstår att deras prognosbaserade system ger kostnadsbesparingar på 5 till 10~\% på uppvärmningen. Att förutsäga exakt hur mycket är svårt då vilka parametrar styrningen i deras modell beror på är kommersiell och således inte delges allmänheten. Utifrån vårt möte med SMHI bedömer vi att fastigheten inte ligger i den kategorin som kan spara mest och för en korrekt bedömning krävs att man har för avsikt att abonnera på tjänsten.

\subsection{Ekonomiska uppskattningar kring de olika åtgärderna}
Vilka ekonomiska förutsättningar som krävs för de olika åtgärden kan bara uppskattas då vi inte har haft befogenheter att begära in riktiga offerter från bygg- och installationsföretag.

För att beräkna kostnaderna användes genomgående en förenklad version av payoff-metoden, som visar hur många år det tar innan en investering betalar av sig. Metoden tar inte hänsyn till att det är olika stora investeringar, och det framgår således inte vilken metod man kan spara mest pengar på utan vilken som lönar sig snabbast. Det finns inget restvärde för någon av investeringarna. Beräkningarna har gjorts för mest positiva samt mest negativa möjliga utfall, vilket ger ett spann på ett antal år. Förenklingarna leder till att varken kalkylräntor eller en eventuell ränta i det fallet investeringen måste finansieras med ett banklån tas med i beräkningen. Den enklaste formen av payback-metoden ges av

\begin{equation} \label{eq:payback}
\text{Antalet år}=\frac{\text{Grundinvestering}}{\text{Sparat belopp per år}}
\end{equation}

Antalet år det tar innan en investering betalar sig enligt \eqref{eq:payback} ges i tabell \ref{tbl:payback} nedan.

\begin{table}[hbtp]
\centering
\caption{Payback-tid för olika investeringar}
\label{tbl:payback}

\begin{tabular}
{|l|r|r|}
\hline
\textbf{Investering} & \textbf{Minimal payback-tid [år]} &{\textbf{Maximal payback-tid [år]}} \\
\hline
Tilläggsisolering & 8,7 & 52,2 \\
\hline
Termostater & 2,9 & 12,8 \\
\hline
Prognosstyrning &  1 & 2,9 \\ 
\hline
Väderstation & 1,7 & 5,2 \\
\hline

\end{tabular}
\end{table}

\paragraph{Isolering}
Vi har inte räknat på en isolering inifrån då det inte har varit möjligt att få en offert eller prisförslag på ett sådant arbete. Det är inte heller säkert att det är möjligt att göra med hänsyn till de boendes komfort. Genom en internetbaserad förfrågningstjänst fick vi visserligen kontakt med ett byggföretag som hävdade att de skulle kunna göra arbetet för 200~000 kronor, vilket enligt lite uppskattningar över kostnader för material, arbetskraft och byggnadsställningar framstår som väldigt billigt. Dock har vi en uppskattad besparingsmöjlighet på 25~\% av energiåtgången, vilket ger en payoff-tid på dryga 10 år, vilket för en sådan typ av investering får anses vara låg. Kostnden 200~000 kronor är troligtvis ett lockpris då den gjordes vid en första kontakt med företaget och utan att de hade möjlighet att se fastigheten. En seriös offertförfrågning skulle ge en rättvisare approximation, men det bedömdes att de befogenheterna saknades inom projektet.

\paragraph{Termostater}
Det bästa fallet är framräknat utan avseende på varken installationskostnad eller räntor på kapital som måste lånas in. Sämsta fallet framräknat med en installationskostnad lika hög som materialet, vilket är tänkt att motsvara ett verkligt scenario. En investering av det här slaget bör ha en avskrivningstid på fem år, vilket är intressant att jämföra payback-tiden mot. Materialet i sig skulle kosta föreningen knappt 100~000 kronor, och med en besparing på upp till 46~\% av energikostnaden\cite{danfoss} per år blir payback-tiden under tre år. En payback-tid på runt hälften av den uppskattade avskrivningstiden är en bra investering. Räknat på dubbla kostnaden och endast 25~\% besparing är payback-tiden tolv år, vilket är tilltaget i överkant. Rimlig payback-tid skulle kunna vara runt 5-6 år med den dyrare varianten av systemet. Det finns också möjlighet att sätta in de billigare varianten, utan färgskärmarna. Det ger halva kostnaden, men ger inte samma mervärde och enkelhet till de boende.

\paragraph{Prognosstyrning från SMHI}
Den stora fördelen med SMHI:s system för prognosstyrning är SMHIs prismodell. Den innebär att investeringskostnaden ska kunna betala sig inom två år, samt att abonnemangsavgiften inte är högre än maximalt hälften av vad man sparar varje år. Det är således väldigt intressant eftersom man egentligen inte behöver några ekonomiska muskler för att börja använda produkten – även en förening som redan har mycket lån och med dålig ekonomi kan köpa systemet. Det bidrar till att samtidigt som SMHI kan ta betalt för sina prognoser, så sparar de boende pengar och mindre energi behöver produceras. Systemet är enkelt och lätt att installera, vilket tillsammans med det låga priset bidrar till att göra det attraktivt.

Med i bästa fall endast ett års payback-tid är prognosstyrning det alternativ som verkar bäst enligt payback-metoden. Det är dock känt att man inte alltid finner de största besparingarna med payback-metoden. Skulle föreningen investera i systemet, som man enligt SMHI ska kunna få installerat till ett pris som betalar sig inom ett eller maximalt två år \cite{smhi1}\cite{smhi2}, tjänar man efter payback-tiden inte nödvändigtvis speciellt mycket pengar per år. Det är inte en hög procentsats av energiåtgången som sparas, samtidigt som de tar en abonnemangsavgift, uppskattningsvis 20-40~\%. \cite{smhi1}\cite{smhi2}

\paragraph{Väderstation}
Beräkningarna för väderstationen är gjorda utifrån inköp av nödvändig utrustning. För Bostadsrättsföreningen Wallerius är detta inte relevant då investeringen redan är gjord. I och med beräkningen finns det ändå en möjlighet att se hur metoden står sig i förhållande till de andra där främst jämförelsen med prognosstyrning är intressant.

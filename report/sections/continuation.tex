\section{Rekommendationer till fortsatt arbete}

Detta är det tredje arbetet som modellerat olika delar av fastigheten på Walleriusgatan för att
kvantifiera värderpåverkan. Detta har genomförts med olika metoder men slutsatserna från arbetena är ungefär desamma.
Därför känns det onödigt att genomföra ett fjärde arbete med samma infallsvinkel. Man kan bara komma så
långt med liknande metodik.

För fortsatt arbete rekommenderas att någon av två olika vägar väljs. De benämns
som produktspåret och grundforskningsspåret. Nedan följer en beskrivning av våra idéer kring dessa.

\paragraph{Produktspåret}
Uppdragsgivaren till detta projekt har som mål att helt automatisera värmeanläggningen i fastigheten och att stabilisera
temperaturen med avseende på väderpåverkan. För att genomföra detta behöver man implementera ett självförbättrande
reglertekniskt system, gärna i nära kontakt med industrin med sikte på att skapa en kommersiell produkt. 

Vid vårt möte med SMHI fick vi intrycket av att de skulle vara väldigt intresserade av att få data från fastigheten.
Detta inkluderar både data från värmeanläggningen och data från väderstationen. Därför kan det vara
en god idé att lägga fortsatta arbeten på is tills det finns några års data från fastigheten att tillgå.
Denna skulle då, tillsammans med erbjudande om ett kandidat- eller examensarbete, kunna bytas mot att få ett värmesystem implementerat till en låg kostnad. Datan skulle dessutom vara till stor nytta för att studera vad olika väder sk1apar för
krav på en värmeanläggning. 

Kompetensen som skulle krävas för ovanstående projekt ligger inom områdena reglerteknik, fysik, byggfysik samt vana med maskininlärning.
Dessutom skulle det vara en fördel om kompetens inom ekonomiska beräkningar finns inom projektgruppen för att
kunna tillgodose att produkten blir lönsam både för företaget och för kunden. Vi anser att detta skulle vara en god
idé till kandidat. eller examensarbete, dock med kravet att dessa kan samarbeta med
ett företag som håller på med denna typ av värmeanläggningar. Detta tror vi är nödvändigt för att projektet skall kunna mynna ut i en marknadsredo produkt.


\paragraph{Grundforskningsspåret}

De tidigare genomförda arbetena har enbart skummat lite på ytan angående vädrets påverkan på fastigheter och alla energiförluster
har kvantifierats väldigt approximativt. För att förbättra tidigare uppsattningar bör en djupare anayls av specifika energiförluster göras. Dessa skulle dock behöva vara kraftigt avgränsade. Några intressanta saker att studera
närmare är vädrets påverkan av konvektionsparametern. Detta skulle kunna genomföras med smarta experiment eller
med avancerade datormodeller. En annan idé är att närmare studera strålning som passerar genom byggnadens fönster. Till detta
skulle det krävas experiment eller data för fönstren för att få veta frekvensberoendet på absorbtions-, reflektions- samt
transmitanskoefficienterna. Med
denna data så skulle det sedan vara möjligt att räkna, alternativt modellera, strålningen genom fönstren. Detta skulle troligen
vara ett lämpligt projekt för några Fysik- och Kemiteknik-teknologer.

Om det skulle finnas några års data från fastighetens värmeanläggning samt väderstation så skulle det
även vara lämpligt att behandla denna med statistik. Genom att genomföra kan man förhoppningsvis
gå att få en ganska god bild över hur väder påverkar fastigheten i realiteten. Ett väl utfört arbete skulle
även kunna bidraga med ett recept för att göra liknande studier på andra fastigheter. Med lite tur skulle detta
kunna bidra med nya insikter i hur väder påverkar en fastighets uppvärmning och kunna leda till nya idéer angående
hur vädrets negativa inverkan kan minimeras och dess positiva inverkan utnyttjas maximalt.

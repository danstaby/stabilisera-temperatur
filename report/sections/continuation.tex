\section{Rekommendationer till fortsatt arbete}

\emph{\color{red} Skriv ingress till detta kapitel. Det känns som vi dyker in i handlingen lite väl snabbt}

Detta är det tredje arbetet som modellerat olika delar av fastigheten på Walleriusgatan för att
kvantifiera värderpåverkan. Detta har genomförts med olika metoder men slutsatserna från arbetena är ungefär desamma.
Därför känns det onödigt att genomföra ett fjärde arbete med samma infallsvinkel. Man kan bara komma så
långt med liknande metodik.

För fortsatt arbete rekommenderas att någon av två olika vägar väljs. De benämns
som produktspåret och grundforskningsspåret. Nedan följer en beskrivning av våra idéer kring dessa.

\paragraph{Produktspåret}
Uppdragsgivaren till detta projekt har som mål att helt automatisera värmeanläggningen i fastigheten och att stabilisera
temperaturen med avseende på väderpåverkan. För att genomföra detta behöver man implementera ett självförbättrande
reglertekniskt system, gärna i nära kontakt med industrin med sikte på att skapa en kommersiell produkt. 

Vid vårt möte med SMHI fick vi intrycket av att de skulle vara väldigt intresserade av att få data från fastigheten.
Detta inkluderar både data från värmeanläggningen och data från väderstationen. Av denna anledning kan det vara
en god idé att lägga fortsatta arbeten på is tills det finns några års data tillgängligt från fastigeten i fråga.
Denna skulle då tillsammans med ett par studenter kunna bytas mot att få ett värmesystem implementar för en billig
penning. Datan skulle dessutom vara till stor nytta för att studera vad olika väder skapar för faktiska
krav på en värmeanläggning. 

Kompentensen som skulle krävas för ovanstående projekt är reglerteknik, fysik/byggfysik samt vana med maskininlärning.
Dessutom så skulle det vara en fördel att kompetens angående ekonomiska beräkningar existerar inom projektet för att
kunna se till att produkten blir lönsam både för företaget och för kunden. Vi anser att detta skulle vara en god
idé till kandidatarbete eller exjobb för ett mindre antal studenter. Kravet är dock att dessa kan samarbeta med
ett företag som håller på med denna typ av värmeanläggningar för att projektet skall mynna ut i en marknadsredo produkt.


\paragraph{Grundforskningsspåret}
Arbetena hittills har enbart skummat lite på ytan angående väderpåverkan på fastigheter. Alla energiförluster
har kvantifierats väldigt approximativt. För att förbättra på tidigare arbeten skulle det gå att djupdyka mer
i specifika energiförluster. Dessa skulle dock behöva vara kraftigt avgränsade. Några intressanta saker att studera
djupare kan vara vädrets påverkan av konvektionsparametern. Detta skulle kunna genomföras med smarta experiment eller
med avancerade datormodeller. En annan idé är att närmare studera strålning som passerar genom byggnadens fönster. Till detta
skulle det krävas experiment eller data på fönster för att veta frekvensberoendet på absorbtions-, reflektions- samt
transmitans-koefficienterna. Med
denna data så skulle det sedan vara möjligt att räkna alternativt modellera strålning genom fönster. Detta skulle troligen
vara ett lämpligt projekt för några Fysik- och Kemi-teknologer.

Om det skulle finnas några års data från fastighetens värmeanläggning samt väderstation så skulle det
även vara lämpligt att behandla denna med statistik. Genom att genomföra detta skulle det förhoppningsvis
gå att få en ganska god bild över hur väder påverkar fastigheten i realiteten. Ett väl utfört arbete skulle
även kunna bidraga med ett recept för att studera liknande på andra fastigheter. Med lite tur skulle detta
kunna bidraga med nya insikter i hur väder påverkar en fastighets uppvärmning och kunna leda till nya idéer angående
hur vädrets negativa inverkan kan minimeras och dess positiva inverkan kan utnyttjas maximalt.

\begin{comment}
Vår första tanke på en fortsättning är att jämföra resultatet från våra modeller med 
statistik från fastighetens värmesystem tillsammans med data från väderstationen. Detta för att verifiera hur tillämpbara 
våra resultat är och hur väl de stämmer överens med verkligheten. Utifrån en utvärdering
av detta kan man gå vidare med att implementera den i ett, för fastigheten 
specialanpassat, reglersystem. Hur mycket man än räknar kan energiflödet inte minskas 
förrän något faktiskt implementeras.
 
 
Startkostnader för värmepumparna. Medeltemperaturskurva. Reglera pumparna. Businessperspektiv. Adaptivt. Lär sig själv med tiden. Samlar in energiåtgång. Trycktesta fastigheten.
Samma förstudie har nu gjorts tre gånger, och mer arbeten skulle inte ge bättre resultat. Framtida ideer är arbeten med mätdatan. In/ut, men det kräver att datan finns tillgänglig redan innan kandidatarbetet startar. Informationen är intressant ur många olika intressenters synvinkel. Både SMHI samt Angela Sasic är troligtvis intresserade av mätdata, och eventuellt kan tjänster och gentjänster utbytas.
Utgångspunkten för ett kommande arbete bör vara att gå direkt mot mätvärden och statistik. För att få ett avslut på arbetet med fastigheten bör det troligtvis läggas som ett exjobb. Kompetenser inom data/lågnivåspråk, statistik, samt inom maskininlärning är essensiella. Projektet bör genomföras i nära samarbete med leverantörerna av mätutrustningen.
 

Ett alternativ skulle kunna vara att göra experimentella mätningar på stegsvar för huset
vid olika väderlekar. Det finns dock en risk att det skulle kunna ta lång tid, eftersom man 
inte själv kan bestämma vilket väder man önskar ha när, och påverka de boende på ett 
icke önskvärt sätt.

Ett annat spår är att efter utvärdeingen av jämförelsen mellan våra resultat och statistik 
från väderstationen försöka förbättra modellen av huset. Kan huset modelleras med 
större noggrannhet blir resultatet förhoppningsvis både mer exakt och mer tillförlitligt. % Varför skulle man vilja göra detta?

Ytterligare en aspekt skulle kunna vara att ta in offerter för de olika förbättringsåtgärderna vi presenterar här för att kunna göra en mer realistisk kostnadsuppskattning. Eftersom 
SMHI är i en utvecklingsfas av sitt projekt med prognosstyrning skulle det vara värt att
 fråga om de skulle vara intresserade av att ge bostadsrättsföreningen ett rabatterat pris 
 på sin tjänst i utbyte mot statistik från husets väderstation.

Att bygga nya hus som är extremt energieffektiva är en branch på stark frammarsch men
 många hus är redan byggda och fortfarande fullt funktionsdugliga. Därför är detta ett 
viktigt område att arbeta vidare med och vi hoppas att fler vill engagera sig i frågan för 
energieffektivisering av äldre byggnader.

% REGN!!
\end{comment}

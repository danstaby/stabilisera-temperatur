\section{Rekommendationer till fortsatt arbete}

Detta är det tredje projektet som modellerat olika delar av fastigheten på Walleriusgatan för att
kvantifiera väderpåverkan. Detta har genomförts med olika metoder men slutsatserna från arbetena är ungefär desamma.
Därför känns det onödigt att genomföra ett fjärde arbete med samma infallsvinkel. 
Det går inte att komma längre med en liknande metodik.

För fortsatt arbete rekommenderas att en av de två olika vägar väljs. De benämns
som produktspåret och grundforskningsspåret. Nedan följer en beskrivning av våra idéer kring dessa.

\paragraph{Produktspåret}
Uppdragsgivaren till detta projekt har som mål att helt automatisera värmeanläggningen i fastigheten och att stabilisera
temperaturen med avseende på väderpåverkan. För att genomföra detta behöver man implementera ett självförbättrande
reglertekniskt system, gärna i nära kontakt med industrin med sikte på att skapa en kommersiell produkt. 

Vid vårt möte med SMHI fick vi intrycket av att de skulle vara väldigt intresserade av att få data från fastigheten.
Detta inkluderar både data från värmeanläggningen och data från väderstationen. Därför kan det vara
en god idé att lägga fortsatta arbeten på is tills det finns några års data från fastigheten att tillgå.
Denna skulle då, tillsammans med erbjudande om ett kandidat- eller examensarbete, kunna bytas mot att få ett reglersystem implementerat till en låg kostnad. Datan skulle dessutom vara till stor nytta för att studera vad olika väder ställer för
krav på en värmeanläggning. 

Kompetensen som skulle krävas för ovanstående projekt ligger inom områdena reglerteknik, fysik, byggfysik samt vana vid maskininlärning.
Dessutom skulle det vara en fördel om kompetens inom ekonomiska beräkningar finns inom projektgruppen för att
kunna tillgodose att produkten blir lönsam både för företaget och för kunden. Vi anser att detta skulle vara en god
idé till kandidat- eller examensarbete, dock med kravet att projektgruppen kan samarbeta med
ett företag som arbetar med denna typ av reglersystem till värmeanläggningar. Detta tror vi är nödvändigt för att projektet skall kunna mynna ut i en produkt färdig för marknaden.


\paragraph{Grundforskningsspåret}

De tidigare genomförda arbetena har enbart skrapat på ytan angående vädrets påverkan på fastigheter och alla energiförluster
har kvantifierats väldigt approximativt. För att förbättra tidigare uppskattningar bör en djupare analys
av specifika energiförluster genomföras. Dessa skulle dock behöva vara kraftigt avgränsade. En intressant sak att studera
närmare är vädrets påverkan av konvektionsparametern. Detta skulle kunna genomföras med mer detaljerade experiment eller
med avancerade datormodeller. En annan idé är att närmare studera strålning som passerar genom byggnadens fönster. Till detta
skulle det vara nödvändigt att utföra experiment på fönstren eller finna lämpliga data för att få veta frekvensberoendet på absorbtions-, reflektions- samt
transmissionskoefficienterna. Med
denna data så skulle det sedan vara möjligt att beräkna eller modellera strålningen genom fönstren. Detta skulle troligen
vara ett lämpligt projekt för några fysik- och kemiteknik-teknologer.

Om det skulle finnas några års data från fastighetens värmeanläggning och väderstation skulle det även vara lämpligt att behandla modellerna med statistik. Genom att genomföra detta kan det förhoppningsvis
gå att få en ganska god bild över hur väder påverkar fastigheten i realiteten. Ett väl utfört arbete skulle
även kunna bidra med ett recept för att göra liknande studier på andra fastigheter. Detta skulle förhoppningsvis 
kunna bidra med nya insikter i hur väder påverkar en fastighets uppvärmning och kunna leda till nya idéer angående
hur vädrets negativa inverkan kan minimeras och dess positiva inverkan maximeras.

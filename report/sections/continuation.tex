\section{Rekommendationer till fortsatt arbete}

Vår första tanke på en fortsättning är att jämföra resultatet från våra modeller med 
statistik som väderstationen har kunnat samla in. Detta för att verifiera hur tillämpbara 
våra resultat är och hur väl de stämmer överens med verkligheten. Utifrån en utvärdering
 av detta kan man gå vidare med att implementera den i ett, för fastigheten 
 specialanpassat, reglersystem. Hur mycket man än räknar kan energiflödet inte minskas 
 förrän något faktiskt implementeras.

Ett alternativ skulle kunna vara att göra experimentella mätningar på stegsvar för huset
 vid olika väderlekar. Det finns dock en risk att det skulle kunna ta lång tid, eftersom man 
 inte själv kan bestämma vilket väder man önskar ha när, och påverka de boende på ett 
 icke önskvärt sätt.

Ett annat spår är att efter utvärdeingen av jämförelsen mellan våra resultat och statistik 
från väderstationen försöka förbättra modellen av huset. Kan huset modelleras med 
större noggrannhet blir resultatet förhoppningsvis både mer exakt och mer tillförlitligt. % Varför skulle man vilja göra detta?

Ytterligare en aspekt skulle kunna vara att ta in offerter för de olika förbättringsåtgärderna vi presenterar här för att kunna göra en mer realistisk kostnadsuppskattning. Eftersom 
SMHI är i en utvecklingsfas av sitt projekt med prognosstyrning skulle det vara värt att
 fråga om de skulle vara intresserade av att ge bostadsrättsföreningen ett rabatterat pris 
 på sin tjänst i utbyte mot statistik från husets väderstation.

Att bygga nya hus som är extremt energieffektiva är en branch på stark frammarsch men
 många hus är redan byggda och fortfarande fullt funktionsdugliga. Därför är detta ett 
viktigt område att arbeta vidare med och vi hoppas att fler vill engagera sig i frågan för 
energieffektivisering av äldre byggnader.

% REGN!!
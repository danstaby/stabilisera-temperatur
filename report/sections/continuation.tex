\section{Rekommendationer till fortsatt arbete}

Vår första tanke på en fortsättning är att jämföra resultatet från våra modeller med 
statistik från fastighetens värmesystem tillsammans med data från väderstationen. Detta för att verifiera hur tillämpbara 
våra resultat är och hur väl de stämmer överens med verkligheten. Utifrån en utvärdering
av detta kan man gå vidare med att implementera den i ett, för fastigheten 
specialanpassat, reglersystem. Hur mycket man än räknar kan energiflödet inte minskas 
förrän något faktiskt implementeras.
 
 
Startkostnader för värmepumparna. Medeltemperaturskurva. Reglera pumparna. Businessperspektiv. Adaptivt. Lär sig själv med tiden. Samlar in energiåtgång. Trycktesta fastigheten.
Samma förstudie har nu gjorts tre gånger, och mer arbeten skulle inte ge bättre resultat. Framtida ideer är arbeten med mätdatan. In/ut, men det kräver att datan finns tillgänglig redan innan kandidatarbetet startar. Informationen är intressant ur många olika intressenters synvinkel. Både SMHI samt Angela Sasic är troligtvis intresserade av mätdata, och eventuellt kan tjänster och gentjänster utbytas.
Utgångspunkten för ett kommande arbete bör vara att gå direkt mot mätvärden och statistik. För att få ett avslut på arbetet med fastigheten bör det troligtvis läggas som ett exjobb. Kompetenser inom data/lågnivåspråk, statistik, samt inom maskininlärning är essensiella. Projektet bör genomföras i nära samarbete med leverantörerna av mätutrustningen.
 

Ett alternativ skulle kunna vara att göra experimentella mätningar på stegsvar för huset
vid olika väderlekar. Det finns dock en risk att det skulle kunna ta lång tid, eftersom man 
inte själv kan bestämma vilket väder man önskar ha när, och påverka de boende på ett 
icke önskvärt sätt.

Ett annat spår är att efter utvärdeingen av jämförelsen mellan våra resultat och statistik 
från väderstationen försöka förbättra modellen av huset. Kan huset modelleras med 
större noggrannhet blir resultatet förhoppningsvis både mer exakt och mer tillförlitligt. % Varför skulle man vilja göra detta?

Ytterligare en aspekt skulle kunna vara att ta in offerter för de olika förbättringsåtgärderna vi presenterar här för att kunna göra en mer realistisk kostnadsuppskattning. Eftersom 
SMHI är i en utvecklingsfas av sitt projekt med prognosstyrning skulle det vara värt att
 fråga om de skulle vara intresserade av att ge bostadsrättsföreningen ett rabatterat pris 
 på sin tjänst i utbyte mot statistik från husets väderstation.

Att bygga nya hus som är extremt energieffektiva är en branch på stark frammarsch men
 många hus är redan byggda och fortfarande fullt funktionsdugliga. Därför är detta ett 
viktigt område att arbeta vidare med och vi hoppas att fler vill engagera sig i frågan för 
energieffektivisering av äldre byggnader.

% REGN!!

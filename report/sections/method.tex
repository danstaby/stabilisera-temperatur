\chapter{Metod}

I detta kapitel presenteras den metodik som tillämpats vid beräkning av olika energiflöden och bygger på det teoretiska underlag som presenterats i föregående kapitel. Byggnaden kommer här delas upp i de två beståndsdelarna grunden – som påverkas av vädret via med marken som värmebuffert – samt byggnadsskalet – med väggar, tak och fönster vars yta direkt påverkas av vädret. Dessutom tillkommer solinstrålningen genom fönster och ofrivillig ventilation på grund av vind, som betraktas helt fristående.

Vi börjar med att beskriva hur ett statiskt värmeflöde genom en vägg kan beräknas går sedan direkt in på solinstrålningen som behandlats med analytiska beräkningar. Vidare beskrivs hur vi har valt våra randvärden och varför. Slutligen finns ett avsnitt som beskriver hur vi med hjälp av programmet Comsol beräknar påverkan på fastigheten från ofrivillig ventilation, det vill säga hur mycket energi som försvinner genom vind som penetrerar huset. 

\subsubsection{Solstrålning genom fönster}

För att beräkna den totala effekt solstrålning tillför byggnaden behövs fönstrenas vinkelberoende g-värden (presenterat i avsnitt \ref{gvalue}). 

För att beräkna g-värdet ur \eqref{eq:radiationwindowstheory:gvalue} behöver parametern $z = \theta/90$ beräknas, där $\theta$ är vinkeln mellan solstrålingens riktning och fönstrets normal. Detta kan göras genom att utgå från aktuellt datum och tid på dygnet.

En metod för att räkna ut solens position presenteras i \cite{walraven78} och en Matlabfunktion baserad på samma artikel kan ses i appendix. % Hänvisa till kod

Om azimuthala och altitudinella vinklarna ($\beta$ respektive $\alpha$) relativt ett väderstreck respektive horisonten tillhandahålls beräknas infallsvinkeln mot glaset, $\theta$, på följande vis:

\begin{equation} 
\theta = \frac{360}{2\pi}\arctan{\left( \sqrt{\tan^2{\left(\alpha\right)}
+ \tan^2{\left(\beta - \gamma \right)}} \right)}
\end{equation}

där $\gamma$ är vinkeln mellan fönstrets normal och väderstrecket mot vilken azimuthala vinkeln anges. Notera att alla vinklar utom $\theta$ anges i radianer.

Med dessa samband tillgängliga kan ett Matlabprogram för beräkning av effektflödet på grund av solstrålning genom fönster skapas, och ett exempel kan ses i appendix. % Hänvisa till appendix.

% Behöver: longitud, latitud, vinkeln relativt väderstreck, fönsters area, g-värde

\subsubsection{Inverkan av skuggor, gardiner och dylikt}

% Beräkna för vilka vinklar skuggor faller över fönstren

% Gardiner, persienner och interiör förändrar situationen, kolla källan nedan
\begin{comment}
Simmler & Binder
Experimental and numerical determination of the total solar energy transmittance of glazing with venetian blind shading
\end{comment}

% Be Särnöe om specifikationer:
% - I vilket väderstreck är normalen riktad?
% - Vilket g-värde har fönstren?

\begin{comment}
I diskussion:
- Hur kan man koppla detta till värmesystemet?
	- Registrera intensitet, tid på dygnet och datum
	- Beräkna ungefärlig tillförd effekt
	- Kompensera genom att säga till värmesystemet att minska/stänga inflödet
- Blir det lättare att helt enkelt mäta temperaturen i rummet och gå utifrån det? Vad är mer kostnadseffektivt?
\end{comment}

\section{Värmeflöde i vägg}

För att beskriva ett värmeflöde används ekvationerna Fouriers värmeekvation
\eqref{eq:conduction:fourier} och värmeledningsekvationen 
\eqref{eq:conduction:heateq}. 
I dessa är
$k$ värmeledningsförmågan\\i $\mbox{W}\mbox{m}^{-2}\mbox{K}^{-1}$ och
$\alpha$ är termisk diffusivitet i $\mbox{m}^2\mbox{s}^{-1}$. \cite{physicshandbook}

Vid statiskt värmeflöde kommer temperaturderivatan med avseende på tiden att vara noll.
Detta innebär att värmeledningsekvationen övergår i Laplaces ekvation
$\Delta{}T = 0$. I en dimension blir detta $d^2T/dx^2 = 0$ vilket innebär
att lösningen blir ett polynom av första ordningen.  

\begin{equation}
%\label{eq:staticwallmethod:fourier}
q_x = -k \frac{dT}{dx}
\tag{\ref{eq:conduction:fourier}}
\end{equation}

\begin{equation}
%\label{eq:staticwallmethod:heat}
\frac{\partial T}{\partial t} = \alpha \Delta T
\tag{\ref{eq:conduction:heateq}}
\end{equation}

\noindent
Vi vill nu beskriva det statiska värmeflödet genom en vägg som består
av flera olika material. De enda värmekällorna som påverkar väggen
är en isoterm $T = T_H$ på ena sidan av väggen
samt en isoterm $T = T_L$ på andra sidan av väggen.

För att göra beräkningarna enklare
ser vi väggen som en oändligt stor skiva. Detta innebär att vi kan räkna
på ekvationen i en dimension. Med hjälp av det härledda sambandet i avsnitt \ref{sec:heatconduction} kan ett ekvationssystem bildas.

\begin{equation}
\label{eq:staticwalltheory:rodmatrix}
\begin{pmatrix}
Q \\
-Q
\end{pmatrix} = 
\frac{k}{L}\begin{pmatrix}
1 & -1 \\
-1 & 1
\end{pmatrix}
\begin{pmatrix}
T_1 \\
T_2
\end{pmatrix}
\end{equation}

\noindent
Vi kan nu teckna dessa ekvationssystem för alla delar av väggen, som skrivs som 
en matris och sedan fylla ut matriserna med nollelement för att slutligen bilda 
en linjärkomination.
Då linjärkombinationen bildas kommer energiflödena i mitten av väggen att
vara noll. Detta överensstämmer väl med att vi har en statisk energifördelning
utan interna värmekällor.
Slutligen får vi ett ekvationsystem enligt ekvation
\eqref{eq:staticwallmethod:full} som enkelt kan lösas.
Matrisen $A$ är linjärkombinationen av nollutfyllda versioner av matrisen i ekvation
\eqref{eq:staticwalltheory:rodmatrix} enligt ekvation
\eqref{eq:staticwallmethod:example}.

\begin{equation}
\label{eq:staticwallmethod:example}
A = \frac{k_1}{L_1}
\begin{pmatrix}
1 & -1 & 0 &  \dots \\
-1 & 1 & 0 &   \\
0 & 0 & 0 &  \\
\vdots & & & \ddots
\end{pmatrix}
+
\frac{k_2}{L_2}
\begin{pmatrix}
0 & 0 & 0 & 0 & \dots \\
0 & 1 & -1 & 0 &  \\
0 & -1 & 1 & 0 & \\
0 & 0 & 0 & 0 & \\
\vdots & & & & \ddots
\end{pmatrix} + \dots
\end{equation}

\begin{equation}
\label{eq:staticwallmethod:full}
\begin{pmatrix}
Q\\0\\...\\0\\-Q
\end{pmatrix} = A
\begin{pmatrix}
T_H\\T_1\\...\\T_{n-1}\\T_L
\end{pmatrix}
\end{equation}

Här gör vi en jämförelse mellan en oisolerad, 50 cm tjock tegelvägg, motsvarande den som finns på fastighetens södersida, och en 50 cm tjock tegelvägg med 10 cm isolering, motsvarande den som finns på fastighetens norrsida.

\section{Finita element av värmeledningsekvationen}
\label{sec:femheat}
I detta avsnitt behandlas finita elementlösningen av värmeledningsekvationen.
Det är från avsnitt \ref{sec:heatconduction} givet att differentialekvationen
enligt ekvation \eqref{eq:femheateq} beskriver värmeflöde i ett material.

\begin{equation}
\label{eq:femheateq}
c_p\rho\frac{\partial T}{\partial t} = \nabla\cdot(k\nabla T)
\end{equation}

\noindent
För att finna en lösning integreras värmeledningsekvationen
multiplicerat med en $L^2$ integrabel testfunktion $\phi(\mathbf{r})$ över hela
definitionsmängden $\Omega$ vars rand benämns $\Gamma$.
Detta kan ses i ekvation \eqref{eq:femheatweak}.
Nu söks en funktion $T(\mathbf{r},t)$ som satisfierar nyss nämnda uttryck för
alla $L^2$ integrabla testfunktioner $\phi(\mathbf{r})$.

\begin{equation}
\label{eq:femheatweak}
\int_\Omega \left(c_p\rho\frac{\partial T}{\partial t} -
\nabla\cdot(k\nabla T)\right)\phi(\mathbf{r})d\Omega = 0
\end{equation}

\noindent
För att förenkla fortsatta beräkningar behövers det genomföras några
omskrivningar av uttrycket. Divergensteoremet används först för att 
eliminera divergensen i värmeledningsekvationens högerled. Detta ger då
ekvation \eqref{eq:femheatweakfull}. Här är $\mathbf{n}$ normalen till randen.

\begin{equation}
\label{eq:femheatweakfull}
\int_\Omega c_p\rho\frac{\partial T}{\partial t}\phi(\mathbf{r}) +
k\nabla T\nabla\phi(\mathbf{r}) d\Omega =
\int_\Gamma k\mathbf{n}\cdot\nabla Td\Gamma
\end{equation}

\noindent
Härnäst skall galerkinformuleringen skissas. Detta genomförs
genom att temperaturen $T$ samt tidsderivatan av temperaturen $\dot{T}$
enligt ekvationerna \eqref{eq:femheatt} och \eqref{eq:femheattdot}.

\begin{align}
\label{eq:femheatt}
T(\mathbf{r}) & \approx \sum_n T_n\phi(\mathbf{r}) \\
\label{eq:femheattdot}
\dot{T}(\mathbf{r}) & \approx \sum_n \dot{T}_n\phi(\mathbf{r})
\end{align}

\noindent
Ansatsen ovan stoppas härnäst in i den svaga formuleringen i ekvation
\eqref{eq:femheatweakfull} vilket ger ekvation \eqref{eq:femheatgalerkin}.
För att kunna lösa problemet för definitionsmängder som består av olika
homogena material väljs testfunktionen $\phi$ så att den försvinner vid
alla andra material än ett och värmeledningskonstanten kan då benämnas $k_n$.
Ekvationssystemet kan sedan skrivas i matrisform vilket kan ses i ekvation
\eqref{eq:femheatmatrix}. Här är $M$ massmatrisen, $A$ är stelhetsmatrisen och
$f$ är belastningsvektorn.

\begin{align}
\label{eq:femheatgalerkin}
\sum_n \dot{T}_n \int_\Omega c_p\rho\phi_i(\mathbf{r})
\phi_n(\mathbf{r})d\Omega
& + \sum_n T_n \int_\Omega k_n \nabla\phi_n(\mathbf{r})\nabla\phi_n(\mathbf{r})
d\Omega \\
&= \int_\Gamma k_i\phi_i\mathbf{n}\cdot\nabla Td\Gamma \Leftrightarrow
\nonumber
\end{align}

\begin{equation}
\label{eq:femheatmatrix}
M\dot{T} + AT = f \Rightarrow
\end{equation}

\begin{equation}
\label{eq:femheatmatrix2}
\dot{T} + M^{-1}AT = M^{-1}f
\end{equation}

\noindent
Som kan ses så är ovanstående uttryck ett system av kopplade ordinära
differentialekvationer vars lösning är trivial med hjälp av egenvärdesuppdelning.
Vektorerna $\{v\}^n_{i=1}$ definieras som egenvektorerna av
$M^{-1}A$ och $\lambda_i$ definieras som egenvärdena till samma matris.
Systemets homogena lösning kan då skrivas som ekvation
\eqref{eq:femheathom}.\cite{lay06}

\begin{equation}
\label{eq:femheathom}
T_h(t) = \sum_n = c_nv_ne^{-\lambda_nt}
\end{equation}

\noindent
Då ekvationen är inhomogen så återstår det att lösa systemets
partikulärlösning. Då inhomogeniteten är konstant så kan lämpligen
en konstant ansättas som partikulärlösning. Detta ger att
$T_p(t) = D$. Insättning i differentialekvationen ger
ekvation \eqref{eq:femheatinstopp} vilket gör att vi kan bestämma
$D$ genom ekvation \eqref{eq:femheatinstopp2}.

\begin{align}
\label{eq:femheatinstopp}
M^{-1}AD &= M^{-1}b \Rightarrow\\
\label{eq:femheatinstopp2}
D &= A^{-1}b
\end{align}

\noindent
Nu kan den fullständiga lösningen skissas som $T = T_h + T_p$ och om
tiden sätts till noll så kan konstanterna $c_n$ bestämmas genom
att $T$ sätts till problemets begynnelsevärden. För ett problem som
saknar tidsberoende eller som har nått en jämviktspunkt måste
tiden vara oändlig och de termer som innehar exponenter blir noll.
Detta innebär att partikulärlösningen $T_p$ är den tidsoberoende lösningen
till problemet. Detta kan enkelt verifieras genom att sätta $\dot{T} = 0$.
Problemet som återstår är då $AT = b$ vars lösning är $T_p$.


\subsection{Finita element av inkompressibel fluid}

För att lösa Navier-Stokes ekvationer kan lämpligen en datormodell användas.
Här består denna modell av ett system uppsatt med Galerkins metod.
I denna lösning så begränsar vi dock oss till att enbart behandla statiska flöden
vilket genomförs genom att sätta alla tidsderivator till noll.

För att hantera trycket i \eqref{eq:convection:continuity}-\eqref{eq:convection:energy} används den tidigare nämnda Boussinesq approximation
samt penalty metoden för att göra hastighetsvektorn källfri och uppfylla
kontinuitetsekvationen. Det finns således inget direkt behov av att räkna ut trycket.
Vid användning av många sorters elementtyper som inte uppfyller Babuska-Brezzikriteriet
är detta dessutom nödvändigt då det annars kan bildas oönskade trycknoder. 
En annan möjlighet är att välja divergensfria element. \cite{babuska1973}\cite{segal2011}

Genom penaltymetoden beskrives här trycket som $p$ enligt ekvation
\eqref{eq:femconvection:penalty}. Här är $p_s$ någon form av idealt statiskt
tryck som är önskat. Detta tryck följer Boussinesq approximation. Med dessa
idealiseringar kan differentialekvationerna sättas upp igen. \cite{heinrich88}\cite{taylor79}
Som kan ses så leder den godtyckliga penaltyparametern $\lambda$ till att justera trycket
om hastighetsfältets divergens ej är identiskt noll. I viss litteratur anges 
det att penaltyparametern skall vara i storleksordningen $10^7$ men att den
är väldigt applikationsberoende. En för liten vald penaltyparameter leder till att
trycket inte elimineras. Andra problem uppstår vid en för stor parameter. Ekvationssystemet
kan bli svårlöst och få stabilitetsproblem när parametern blir
för stor i jämförelse med de andra delarna i differentialekvationen.\cite{reddy93}\cite{roy05}\cite{basak04}\cite{segal2011}

\begin{equation}
\label{eq:femconvection:penalty}
p = p_s - \lambda\nabla\cdot\mathbf{v}
\end{equation}

\noindent
Fortsatt skall trycket deriveras med avseende på de rumsliga variablerna vilket möjliggör
att eliminera trycket från differentialekvationerna. Dessa deriveringar kan ses i ekvation
\eqref{eq:femconvection:partx} samt \eqref{eq:femconvection:partz}. Notera att det statiska trycket
$p_s$ ej beror på $x$ vilket resulterar i att derivatan är noll.

\begin{equation}
\label{eq:femconvection:partx}
\frac{\partial p}{\partial x} = \frac{\partial p_s}{\partial x} -
\frac{\partial}{\partial x} \lambda\nabla\cdot\mathbf{v} = -
\frac{\partial}{\partial x} \lambda\nabla\cdot\mathbf{v}
\end{equation}

\begin{equation}
\label{eq:femconvection:partz}
\frac{\partial p}{\partial z} = \frac{\partial p_s}{\partial z} -
\frac{\partial}{\partial z} \lambda\nabla\cdot\mathbf{v} =
-g\rho_0 - \frac{\partial}{\partial z} \lambda\nabla\cdot\mathbf{v}
\end{equation}

\noindent
Detta förs in i momentekvationerna vilket ger ekvationerna \eqref{eq:femconvection:u} -
\eqref{eq:femconvection:T}. Här är det ekvationssystem som syftar att lösas.

\begin{equation}
\label{eq:femconvection:u}
\mathbf{v}\cdot\nabla u =
\frac{\lambda}{\rho_0}\nabla\cdot\mathbf{v} +
\nu\Delta u
\end{equation}

\begin{equation}
\label{eq:femconvection:w}
\mathbf{v}\cdot\nabla w =
\frac{\lambda}{\rho_0}\nabla\cdot\mathbf{v} + \nu\Delta w +g\beta(T-T_0)
\end{equation}

\begin{equation}
\label{eq:femconvection:T}
\mathbf{v}\cdot\nabla T = \alpha\Delta T
\end{equation}

\subsubsection{Svag formulering}

En finita elementlösning med Galerkins metod kräver att problemet reduceras till
ett ekvivalent variationsproblem. Här söks $T\in\Phi$, $u\in\Phi$ och
$w\in\Phi$ som uppfyller ekvation \eqref{eq:femconvection:variation}. Här
betecknar brackets skalärprodukt, $\mathbf{L}$ är differentialoperatorn
som betecknar systemet av differentialekvationer som $\mathbf{L}(T,u,w) = 0$.
$\Phi$ är rummet av alla testfunktioner $\phi$ som är kontinuerliga i
definitionsmängden $\Omega$ samt vars derivator är bitvis kontinuerliga på randen
$/Gamma$. De måste även vara $L^2$ integrabla.

\begin{equation*}
\label{eq:femconvection:variation}
\langle \mathbf{L}(T,u,w), \phi \rangle = 0\mbox{,  } \forall \phi \in \Phi
\end{equation*}


\section{Datorsimulering av ofrivillig ventilation}

Ett hus är i praktiken omöjligt att göra helt tätt. Då vinden ligger på
får man därför ett drag genom huset, en ofrivillig ventilation. Då vinden
sällan är lika varm som inomhusluften leder detta till en energiförlust.
Den har beräknats med hjälp av programvaran Comsol. Problemets geometri har
setts upp enligt figur~ \ref{fig:windmethod:tri}. Bredvid fastigheten på Walleriusgatan ligger en annan byggnad och problemet med vind från de olika hållen blir symmetriskt. Blåser det från norr illustreras den för projektet aktuella fastigheten till höger i bild, och blåser det från söder påverkas den som den till vänster.

Från vänstra kanten så har luften blåst in med en konstant vindhastiget som varieras mellan olika
experiment. På andra sidan av fastigheterna har det satt ett konstant lufttryck som motsvarar en
atmosfärs tryck. På randerna som ligger mot mark eller mot hus är vindhastigheten satt till noll.
Slutligen utför ej luftmassan ovanför definitionsmängden någon kraft på luften som ligger längs den
övre randen.

\begin{figure}
\centering
\includegraphics[width=127mm,height=76mm]{images/triinfiltration.eps}
\caption{Triangulering samt definitionsmängd uppsatt för problemet.}\label{fig:windmethod:tri}
\end{figure}

Trycket inne i fastigheterna har sedan beräknats genom att luftläckaget antagits homogent utspritt över fastigheternas
väggar och att inget luft läckt genom taket. Därefter har Darcys lag satts upp med antagande om jämvikt så att lika mycket
luft som flödar in även kommer ut. Då både trycket inomhus och på ränderna är kända kan läckaget beräknas med Darcys lag
eller med någon annan exponent, se avsnitt~\ref{sec:darcy}. Här är antagandet gjort att huset läcker mycket och har $C(50)^{0,60} = 1,2$. \emph{\color{red} Vad menas här? Vad kom C ifrån? och siffrorna?} Dock kommer även exponenten ha betydelse för läckaget.\cite{sasic}


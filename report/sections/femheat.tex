\section{Finita elementlösning av värmeledningsekvationen}
\label{sec:femheat}
För att beräkna energiflödena har värmeledningsekvationen använts.
Det är från avsnitt \ref{sec:heatconduction} givet att differentialekvationen
enligt ekvation \eqref{eq:femheateq} beskriver värmeflöde i ett material.
För att lösa den har  finita elementmetoden tillämpats.
 
\begin{equation}
\label{eq:femheateq}
c_p\rho\frac{\partial T}{\partial t} = \nabla\cdot(k\nabla T)
\end{equation}

För att finna en lösning integreras värmeledningsekvationen
multiplicerat med en $L^2$-integrerbar testfunktion $\phi(\mathbf{r})$ över hela
definitionsmängden $\Omega$ vars rand benämns $\Gamma$ alltså

\begin{equation}
\label{eq:femheatweak}
\int_\Omega \left(c_p\rho\frac{\partial T}{\partial t} -
\nabla\cdot(k\nabla T)\right)\phi(\mathbf{r})\mathrm{d}\Omega = 0
\end{equation}

Nu söks en funktion $T(\mathbf{r},t)$ som satisfierar nyss nämnda uttryck för
alla $L^2$-integrerbara testfunktioner $\phi(\mathbf{r})$.

För att förenkla fortsatta beräkningar genomförs några
omskrivningar av uttrycket. Först används divergensteoremet för att 
eliminera divergensen i värmeledningsekvationens högerled. Detta ger då

\begin{equation}
\label{eq:femheatweakfull}
\int_\Omega c_p\rho\frac{\partial T}{\partial t}\phi(\mathbf{r}) +
k\nabla T\nabla\phi(\mathbf{r}) \mathrm{d}\Omega =
\int_\Gamma k\mathbf{n}\cdot\nabla T\mathrm{d}\Gamma
\end{equation}

där $\mathbf{n}$ är normalen till randen.

Härnäst skall galerkinformuleringen skissas (för detaljer kring galerikinformulering och den svaga formuleringen se avsnitt \ref{sec:femtheory}). Detta genomförs
genom att temperaturen $T$ och dess tidsderivata $\dot{T}$ definieras
enligt

\begin{align}
\label{eq:femheatt}
T(\mathbf{r}) & \approx \sum_n T_n\phi(\mathbf{r}) \\
\label{eq:femheattdot}
\dot{T}(\mathbf{r}) & \approx \sum_n \dot{T}_n\phi(\mathbf{r}).
\end{align}

Ansatsen ovan stoppas härnäst in i den svaga formuleringen i ekvation
\eqref{eq:femheatweakfull} vilket ger

\begin{align}
\label{eq:femheatgalerkin}
\sum_n \dot{T}_n \int_\Omega c_p\rho\phi_i(\mathbf{r})
\phi_n(\mathbf{r})\mathrm{d}\Omega
& + \sum_n T_n \int_\Omega k_n \nabla\phi_i(\mathbf{r})\nabla\phi_n(\mathbf{r})
\mathrm{d}\Omega \\
&= \int_\Gamma k_i\phi_i\mathbf{n}\cdot\nabla T\mathrm{d}\Gamma \Leftrightarrow
\nonumber
\end{align}

För att kunna lösa problemet för definitionsmängder som består av olika
homogena material väljs testfunktionen $\phi$ så att den försvinner vid
alla material utom ett visst utvalt och värmelednings\-konstanten kan då benämnas $k_n$.
Ekvationssystemet kan sedan skrivas i matrisform

\begin{equation}
\label{eq:femheatmatrix}
M\dot{T} + AT = f \Rightarrow
\end{equation}

\begin{equation}
\label{eq:femheatmatrix2}
\dot{T} + M^{-1}AT = M^{-1}f
\end{equation}

där $M$ är massmatrisen, $A$ är stelhetsmatrisen och $f$ är belastningsvektorn.

Som kan ses så är ovanstående uttryck ett system av kopplade ordinära
differential\-ekvationer vars lösning är trivial med hjälp av egenvärdes\-uppdelning.
Vektorerna $\{v\}^n_{i=1}$ definieras som egenvektorerna av
$M^{-1}A$ och $\lambda_i$ definieras som egenvärdena till samma matris.
Systemets homogena lösning kan då skrivas som\cite{lay06}

\begin{equation}
\label{eq:femheathom}
T_h(t) = \sum_n = c_nv_ne^{-\lambda_nt}.
\end{equation}

Då ekvationen är inhomogen så återstår det att finna systemets
partikulärlösning. Då inhomogeniteten är konstant så kan lämpligen
en konstant ansättas som partikulärlösning. Detta ger att
$T_{p}(t) = D$. Insättning i differentialekvationen ger

\begin{align}
\label{eq:femheatinstopp}
M^{-1}AD &= M^{-1}b \Rightarrow\\
\label{eq:femheatinstopp2}
D &= A^{-1}b
\end{align}

där $D$ kan bestämmas med ekvationen efter implikationspilen.

Nu kan den fullständiga lösningen skissas som $T = T_h + T_p$ och om
tiden sätts till noll så kan konstanterna $c_n$ bestämmas genom
att $T$ sätts till problemets begynnelsevärden. För ett problem som
saknar tidsberoende eller som har nått en jämviktspunkt måste
tiden vara oändlig och de termer som innehar exponenter blir noll.
Detta innebär att partikulärlösningen $T_p$ är den tidsoberoende lösningen
till problemet. Detta kan enkelt verifieras genom att sätta $\dot{T} = 0$.
Problemet som återstår är då $AT = b$ vars lösning är $T_p$.

\subsection{Lösning av problem med tidsberoende randvillkor}
\label{subsec:mol}

För att studera vädrets påverkan är det i många fall nödvändigt att behandla
tidsberoende randvillkor. I detta arbete används framförallt två metoder för detta:
Method-of-lines, MOL, och semidiskret MOL. Den förstnämnda metoden går ut på att
ett system av första ordningens ordinära differentialekvationer löses analytiskt för
alla $t>0$. I den semidiskreta varianten approximeras istället randvillkoren vid
ett visst tillfälle till en lätthanterlig funktion som endast är giltig vid
den valda tidpunkten. Efter detta löser man differentialekvationerna upprepade gånger
med den förra itereringens värden som nya initialvärden och de nya uppdaterade
randvillkorenen. Denna metod är väldigt användbar då det är möjligt att behandla en nästintill godtycklig
funktion med den, så länge det finns  processorkraft att lägga på små tidssteg i jämförelse med
randvillkorens förändringshastiget.

För att lösa differentialekvationerna analytiskt för alla tider behöver man först
sätta upp en funktion som beskriver randvillkorens beteende. Efter detta sätts
dessa villkor i galerkinformuleringen. Dessa villkor kommer att bilda en inhomogenitet
i systemet av differentialekvationer. Metoden för lösning av dessa
blir då på samma sätt som i avsnittet ovan,
där systemet först diagonaliseras med till exempel egenvärdesdiagonalisering. Sedan är det bara att lösa systemet som vilka vanliga differentialekvationer som helst, för att sedan återföras till önskad storhet genom
invers diagonalisering $T = VY$.

Receptet för lösning blir då följande. Sätt upp ekvationssystemet

\begin{equation}
\label{eq:femheat:ourdiff}
\dot{T}(t) + AT(t) = b(t).
\end{equation}

Diagonalisera systemet genom att definiera $V$ som matrisen av egenvektorer av $A$ och $D$ som den
diagonala matrisen vars element är egenvärdena av $A$. Från elementär linjär algebra\cite{lay06} ges att

\begin{equation}
\label{eq:femheat:vdv}
A = VDV^{-1}.
\end{equation}

För att fortsätta beräkningarna skall ett variabelbyte genomföras. Ansätt att

\begin{equation}
\label{eq:femheat:vy}
T = VY.
\end{equation}

Nu kan ekvationerna \eqref{eq:femheat:vdv} och \eqref{eq:femheat:vy} sättas in i differentialekvationen
\eqref{eq:femheat:ourdiff}. Efter att termer flyttats och strukits ges

\begin{equation}
\dot{Y}(t) + DY(t) = V^{-1}b(t).
\end{equation}

Slutligen söks den homogena lösningen $Y_h$ samt partikulärlösningen $Y_p$ och en linjärkombination bildas.
Lösningen uttryckt som $T$ blir då

\begin{equation}
\label{eq:femheat:solution}
T(t) = V(Y_h+Y_p).
\end{equation}

där koefficienterna dimensioneras så att begynelsevillkoret uppfylls.

\section{Finita element av värmeledningsekvationen}

I detta avsnitt behandlas finita elementlösningen av värmeledningsekvationen.
Det är från avsnitt \ref{sec:heatconduction} givet att differentialekvationen
enligt ekvation \eqref{eq:femheateq} beskriver värmeflöde i ett material.

\begin{equation}
\label{eq:femheateq}
c_p\rho\frac{\partial T}{\partial t} = \nabla\cdot(k\nabla T)
\end{equation}

\noindent
För att finna en lösning integreras värmeledningsekvationen
multiplicerat med en $L^2$ integrabel testfunktion $\phi(\mathbf{r})$ över hela
definitionsmängden $\Omega$ vars rand benämns $\Gamma$.
Detta kan ses i ekvation \eqref{eq:femheatweak}.
Nu söks en funktion $T(\mathbf{r},t)$ som satisfierar nyss nämnda uttryck för
alla $L^2$ integrabla testfunktioner $\phi(\mathbf{r})$.

\begin{equation}
\label{eq:femheatweak}
\int_\Omega \left(c_p\rho\frac{\partial T}{\partial t} -
\nabla\cdot(k\nabla T)\right)\phi(\mathbf{r})d\Omega = 0
\end{equation}

\noindent
För att förenkla fortsatta beräkningar behövers det genomföras några
omskrivningar av uttrycket. Divergensteoremet används först för att 
eliminera divergensen i värmeledningsekvationens högerled. Detta ger då
ekvation \eqref{eq:femheatweakfull}. Här är $\mathbf{n}$ normalen till randen.

\begin{equation}
\label{eq:femheatweakfull}
\int_\Omega c_p\rho\frac{\partial T}{\partial t}\phi(\mathbf{r}) +
k\nabla T\nabla\phi(\mathbf{r}) d\Omega =
\int_\Gamma k\mathbf{n}\cdot\nabla Td\Gamma
\end{equation}

\noindent
Härnäst skall galerkinformuleringen skissas. Detta genomförs
genom att temperaturen $T$ samt tidsderivatan av temperaturen $\dot{T}$
enligt ekvationerna \eqref{eq:femheatt} och \eqref{eq:femheattdot}.

\begin{align}
\label{eq:femheatt}
T(\mathbf{r}) & \approx \sum_n T_n\phi(\mathbf{r}) \\
\label{eq:femheattdot}
\dot{T}(\mathbf{r}) & \approx \sum_n \dot{T}_n\phi(\mathbf{r})
\end{align}

\noindent
Ansatsen ovan stoppas härnäst in i den svaga formuleringen i ekvation
\eqref{eq:femheatweakfull} vilket ger ekvation \eqref{eq:femheatgalerkin}.
För att kunna lösa problemet för definitionsmängder som består av olika
homogena material väljs testfunktionen $\phi$ så att den försvinner vid
alla andra material än ett och värmeledningskonstanten kan då benämnas $k_n$.
Ekvationssystemet kan sedan skrivas i matrisform vilket kan ses i ekvation
\eqref{eq:femheatmatrix}. Här är $M$ massmatrisen, $A$ är stelhetsmatrisen och
$f$ är belastningsvektorn.

\begin{align}
\label{eq:femheatgalerkin}
\sum_n \dot{T}_n \int_\Omega c_p\rho\phi_i(\mathbf{r})
\phi_n(\mathbf{r})d\Omega
& + \sum_n T_n \int_\Omega k_n \nabla\phi_n(\mathbf{r})\nabla\phi_n(\mathbf{r})
d\Omega \\
&= \int_\Gamma k_i\phi_i\mathbf{n}\cdot\nabla Td\Gamma \Leftrightarrow
\nonumber
\end{align}

\begin{equation}
\label{eq:femheatmatrix}
M\dot{T} + AT = f \Rightarrow
\end{equation}

\begin{equation}
\label{eq:femheatmatrix2}
\dot{T} + M^{-1}AT = M^{-1}f
\end{equation}

\noindent
Som kan ses så är ovanstående uttryck ett system av kopplade ordinära
differentialekvationer vars lösning är trivial med hjälp av egenvärdesuppdelning.
Vektorerna $\{v\}^n_{i=1}$ definieras som egenvektorerna av
$M^{-1}A$ och $\lambda_i$ definieras som egenvärdena till samma matris.
Systemets homogena lösning kan då skrivas som ekvation
\eqref{eq:femheathom}.\cite{lay06}

\begin{equation}
\label{eq:femheathom}
T_h(t) = \sum_n = c_nv_ne^{-\lambda_nt}
\end{equation}

\noindent
Då ekvationen är inhomogen så återstår det att lösa systemets
partikulärlösning. Då inhomogeniteten är konstant så kan lämpligen
en konstant ansättas som partikulärlösning. Detta ger att
$T_p(t) = D$. Insättning i differentialekvationen ger
ekvation \eqref{eq:femheatinstopp} vilket gör att vi kan bestämma
$D$ genom ekvation \eqref{eq:femheatinstopp2}.

\begin{align}
\label{eq:femheatinstopp}
M^{-1}AD &= M^{-1}b \Rightarrow\\
\label{eq:femheatinstopp2}
D &= A^{-1}b
\end{align}

\noindent
Nu kan den fullständiga lösningen skissas som $T = T_h + T_p$ och om
tiden sätts till noll så kan konstanterna $c_n$ bestämmas genom
att $T$ sätts till problemets begynnelsevärden. För ett problem som
saknar tidsberoende eller som har nått en jämviktspunkt måste
tiden vara oändlig och de termer som innehar exponenter blir noll.
Detta innebär att partikulärlösningen $T_p$ är den tidsoberoende lösningen
till problemet. Detta kan enkelt verifieras genom att sätta $\dot{T} = 0$.
Problemet som återstår är då $AT = b$ vars lösning är $T_p$.

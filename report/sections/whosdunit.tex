\documentclass[12pt,a4paper]{article}
\usepackage[utf8]{inputenc}
\usepackage[T1]{fontenc}
\usepackage[swedish, english]{babel}
\usepackage{amsmath}
\usepackage{ae}
\usepackage{subfig}
\usepackage{units}
\usepackage{icomma}
\usepackage{color}
\usepackage{graphicx}
\usepackage{bbm}
\usepackage{textcomp}
\usepackage{float}
\usepackage{units}
\usepackage{amssymb}
\newcommand{\rd}{\ensuremath{\mathrm{d}}}
\usepackage{fullpage}

\newcommand{\id}{\ensuremath{\,\rd}}
\usepackage{hyperref}

% Ökar styckeavstånd
\setlength{\parskip}{2ex plus0.5ex minus0.2ex}
% Tar bort indrag i början av stycke
\addtolength{\parindent}{-0.6 cm}

\begin{document}
\selectlanguage{swedish}

\section*{Arbetsfördelning under projektet}

\emph{\color{red}Här kan ni fylla i vem som gjort vad.}
\emph{\color{red}Ny mall, nya grejer. Jag tycker/jag skulle vilja skriva ett löpande stycke under varje av de tre paragrapherna. När ni har gjort det, om ni tycker det är en bra idee, ta bort punktlistorna. Man kan själklart även skriva i punktlisteform, vad är bäst ?}

\paragraph{Ansvarsområden}

       Planering

       Informationsinhämtning/inläsningsdel

       Metoder -- val/utveckling 

       Genomförande 

\paragraph{Bidrag till problemlösning, syntes och analys}

       Problemlösning 

       Kreativitet, idérikedom

       Skapande av modell

       Analys av projektrelaterat material 

       Diskussionsbidrag

       Slutsatser 

\paragraph{Huvudansvarig författare av avsnitt}

       Avsnitten anges

       Eventuell redaktionell ansvarsfördelning bör anges"

\subsection*{Erik Ahlqvist}

\paragraph{Ansvarsområden}

\begin{itemize}
\item[-] Lagt mycket tid på inläsning av tidigare kandidatarbeten samt patent. Även en del artiklar, avhandlingar samt viss kurslitteratur. 
\item[-] Deltagit i planeringen, samt stor del skrivande i planeringsrapporten.
\item[-] Metoden för att göra ekonomiska uppskattningar.
\item[-] Hämtat all information kring andra energibesparande åtgärder. 
\end{itemize}


\paragraph{Bidrag till problemlösning, syntes och analys}

\begin{itemize}
\item[-] Deltagit i arbetet med att skriva om konduktion.
\item[-] Aktivt bidragit till att bygga upp en vettig diskussion, påverka innehållet och summera ihop det.
\item[-] Genom arbetet med diskussionen tagit en naturligt stor del även vid summeringen i slutsatserna.
\end{itemize}


\paragraph{Huvudansvarig författare av avsnitt}

\begin{itemize}
\item[-] 
\item[-] Huvudansvarig för hela jämförelsen med andra energibesparande åtgärder.
\item[-] Skrivit många utkast till resten av diskussion samt slutsats.
\end{itemize}




\subsection*{Ylva Dahl}

\paragraph{Ansvarsområden}

\item[-] Läst på om fastigheten och dess väderstation.
\item[-] Studerat in begreppen svartkroppsstrålning, free-running temperature, ekvivalent temperatur samt sökt information kring naturkonstanter för luft.
\item[-] Samlat in och sammanställt väderdata främst från SMHI:s databas för våra exempel.

\paragraph{Bidrag till problemlösning, syntes och analys}
\item[-] Summerat energiflöden.

\paragraph{Huvudansvarig författare av avsnitt}

\begin{itemize}
\item[-] Skrivit om fastigheten och dess väderstation.
\item[-] Presenterat naturkonstanter för luft och begreppen svartkroppsstrålning, free-running temperature och ekvivalent temperatur.
\item[-] Skrivit text till bilderna i resultatet – väggar, burspråk, tak och grund – och sammanfattat i bild och text.
\item[-] Skrivit dispositionen och introduktioner till avsnitten teori och resultat.
\end{itemize}


\subsection*{Mats Lindström}

\paragraph{Ansvarsområden}

\paragraph{Bidrag till problemlösning, syntes och analys}

\paragraph{Huvudansvarig författare av avsnitt}


\paragraph{Mats Lindström}
\begin{itemize}
\item A
\item B
\item ...
\end{itemize}



\subsection*{Dan Ståby}

\paragraph{Ansvarsområden}

\paragraph{Bidrag till problemlösning, syntes och analys}

\paragraph{Huvudansvarig författare av avsnitt}

\paragraph{Dan Ståby}
\begin{itemize}
\item Förstuderat samt implementerat finita-elementlösningarna av de studerade situationerna.
\item Skrivit textern om finita-elementmetoden i rapporten samt skrivit om närliggande ämnen som problemuppställning och dylikt. 
\item Implementerat Matlab-kod för att snyggt presentera data från datormodellerna samt producerat alla grafer från
nyss nämnda modeller.
\end{itemize}

\end{document}

\documentclass[12pt,a4paper]{article}
\usepackage[utf8]{inputenc}
\usepackage[T1]{fontenc}
\usepackage[swedish, english]{babel}
\usepackage{amsmath}
\usepackage{ae}
\usepackage{subfig}
\usepackage{units}
\usepackage{icomma}
\usepackage{color}
\usepackage{graphicx}
\usepackage{bbm}
\usepackage{textcomp}
\usepackage{float}
\usepackage{units}
\usepackage{amssymb}
\newcommand{\rd}{\ensuremath{\mathrm{d}}}
\usepackage{fullpage}

\newcommand{\id}{\ensuremath{\,\rd}}
\usepackage{hyperref}

% Ökar styckeavstånd
\setlength{\parskip}{2ex plus0.5ex minus0.2ex}
% Tar bort indrag i början av stycke
\addtolength{\parindent}{-0.6 cm}

\begin{document}
\selectlanguage{swedish}

\section*{Arbetsfördelning under projektet}

\emph{\color{red}Här kan ni fylla i vem som gjort vad.}
\emph{\color{red}Ny mall, nya grejer. Jag tycker/jag skulle vilja skriva ett löpande stycke under varje av de tre paragrapherna. När ni har gjort det, om ni tycker det är en bra idee, ta bort punktlistorna. Man kan själklart även skriva i punktlisteform, vad är bäst ?}

\paragraph{Ansvarsområden}
\begin{itemize}
\item[-] Planering
\item[-] Informationsinhämtning/inläsningsdel
\item[-] Metoder -- val/utveckling 
\item[-] Genomförande 
\end{itemize}

\paragraph{Bidrag till problemlösning, syntes och analys}

\begin{itemize}
\item[-] Problemlösning 
\item[-] Kreativitet, idérikedom
\item[-] Skapande av modell
\item[-] Analys av projektrelaterat material 
\item[-] Diskussionsbidrag
\item[-] Slutsatser 
\end{itemize}

\paragraph{Huvudansvarig författare av avsnitt}

\begin{itemize}
\item[-] Avsnitten anges
\item[-] Eventuell redaktionell ansvarsfördelning bör anges"
\end{itemize}

\subsection*{Erik Ahlqvist}

\paragraph{Ansvarsområden}

\begin{itemize}
\item[-] Lagt mycket tid på inläsning av tidigare kandidatarbeten samt patent. Även en del artiklar, avhandlingar samt viss kurslitteratur. 
\item[-] Deltagit i planeringen, samt stor del skrivande i planeringsrapporten.
\item[-] Metoden för att göra ekonomiska uppskattningar.
\item[-] Hämtat all information kring andra energibesparande åtgärder. 
\end{itemize}


\paragraph{Bidrag till problemlösning, syntes och analys}

\begin{itemize}
\item[-] Deltagit i arbetet med att skriva om konduktion.
\item[-] Aktivt bidragit till att bygga upp en vettig diskussion, påverka innehållet och summera ihop det.
\item[-] Genom arbetet med diskussionen tagit en naturligt stor del även vid summeringen i slutsatserna.
\end{itemize}


\paragraph{Huvudansvarig författare av avsnitt}

\begin{itemize}
\item[-] 
\item[-] Huvudansvarig för hela jämförelsen med andra energibesparande åtgärder.
\item[-] Skrivit många utkast till resten av diskussion samt slutsats.
\end{itemize}




\subsection*{Ylva Dahl}

\paragraph{Ansvarsområden}

\begin{itemize}
\item[-] Läst på om fastigheten och dess väderstation.
\item[-] Studerat in begreppen svartkroppsstrålning, free-running temperature, ekvivalent temperatur samt sökt information kring naturkonstanter för luft.
\item[-] Samlat in och sammanställt väderdata främst från SMHI:s databas för våra exempel.
\end{itemize}

\paragraph{Bidrag till problemlösning, syntes och analys}
\begin{itemize}
\item[-] Summerat energiflöden.
\end{itemize}

\paragraph{Huvudansvarig författare av avsnitt}

\begin{itemize}
\item[-] Skrivit om fastigheten och dess väderstation.
\item[-] Presenterat naturkonstanter för luft och begreppen svartkroppsstrålning, free-running temperature och ekvivalent temperatur.
\item[-] Skrivit text till bilderna i resultatet – väggar, burspråk, tak och grund – och sammanfattat i bild och text.
\item[-] Skrivit dispositionen och introduktioner till avsnitten teori och resultat.
\end{itemize}


\subsection*{Mats Lindström}

\paragraph{Ansvarsområden}

\begin{itemize}
\item[-] Flöden genom fönster respektive tak.
\item[-] Teori om konvektion och värmeledning.
\item[-] Utformning av rapport i \LaTeX, inklusive BibTex.
\end{itemize}

\paragraph{Bidrag till problemlösning, syntes och analys}

\begin{itemize}
\item[-] Program för beräkning av solens position
\item[-] Skapande av modell för solinstrålning genom fönster.
\end{itemize}

\paragraph{Huvudansvarig författare av avsnitt}

\begin{itemize}
\item[-] Har skrivit härledningar till konvektionsekvationerna, värmeledningsekvationerna och reflektioner i fönster, samt förklaring av vindens konvektionskoefficient.
\item[-] Presenterat teori, metod och resultat av flöden genom fönster via solinstrålning, långvågsstrålning, värmeledning och konvektion.
\end{itemize}



\subsection*{Dan Ståby}

\paragraph{Ansvarsområden}
\begin{itemize}
\item[-] Huvudansvar för metodutveckling.
\item[-] Förstuderat och läst in på Finita elementmetoden.
\end{itemize}

\paragraph{Bidrag till problemlösning, syntes och analys}
\begin{itemize}
\item[-] Satt upp randvillkor för alla modeller
\item[-] Implementerat finita-elementlösningarna i \emph{Matlab} och \emph{Comsol Multiphysics}.
\item[-] Byggt matlabkod för fyllda contour-grafer för triangulära element.
\item[-] Hjälpt till att beräkna hur mycket energi vi kan spara på att ta hänsyn till väder.
\item[-] Producerat grafer för värmeledning genom väggar och grund. Även producerat grafer för temperaturfördelning utanför en vägg.
\item[-] Beräknat på långvågig strålning genom fönster
\end{itemize}

\paragraph{Huvudansvarig författare av avsnitt}
\begin{itemize}
\item[-] Huvudansvar för Metod-avnisttet
\item[-] Skrivit teori och metod om finita elementlösningarna
\end{itemize}


\end{document}

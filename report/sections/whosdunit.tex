\documentclass[12pt,a4paper]{article}
\usepackage[utf8]{inputenc}
\usepackage[T1]{fontenc}
\usepackage[swedish, english]{babel}
\usepackage{amsmath}
\usepackage{ae}
\usepackage{subfig}
\usepackage{units}
\usepackage{icomma}
\usepackage{color}
\usepackage{graphicx}
\usepackage{bbm}
\usepackage{textcomp}
\usepackage{float}
\usepackage{units}
\usepackage{amssymb}
\newcommand{\rd}{\ensuremath{\mathrm{d}}}
\usepackage{fullpage}

\newcommand{\id}{\ensuremath{\,\rd}}
\usepackage{hyperref}

% Ökar styckeavstånd
\setlength{\parskip}{2ex plus0.5ex minus0.2ex}
% Tar bort indrag i början av stycke
\addtolength{\parindent}{-0.6 cm}

\begin{document}
\selectlanguage{swedish}

\section*{Arbetsfördelning under projektet}

Arbetet i gruppen har fungerat väl och alla har engagerat sig i projektets olika delar. Hela gruppen har haft en prestigelös inställning och arbetat för projektets bästa. När olika åsikter har förekommit har en saklig diskussion förts för att nå konsensus. För att färdigställa arbetet har alla gruppens medlemmar deltagit i korrekturläsningen och bidragit till att få en sammanhållen rapport.


\subsection*{Erik Ahlqvist}

\paragraph{Ansvarsområden}

\begin{itemize}
\item[-] Lagt mycket tid på inläsning av tidigare kandidatarbeten samt patent. Även en del artiklar, avhandlingar samt viss kurslitteratur. 
\item[-] Deltagit i planeringen, samt stor del skrivande i planeringsrapporten.
\item[-] Metoden för att göra ekonomiska uppskattningar.
\item[-] Hämtat all information kring andra energibesparande åtgärder. 
\end{itemize}


\paragraph{Bidrag till problemlösning, syntes och analys}

\begin{itemize}
\item[-] Anvarig för möten och gruppens välbefinnande.
\item[-] Deltagit i arbetet med att skriva om konduktion.
\item[-] Aktivt bidragit till att bygga upp en vettig diskussion, påverka innehållet och summera ihop det.
\item[-] Genom arbetet med diskussionen tagit en naturligt stor del även vid summeringen i slutsatserna.
\end{itemize}


\paragraph{Huvudansvarig författare av avsnitt}

\begin{itemize}
\item[-] Har haft det redaktionella huvudansvaret för kapitlen Introduktion och Diskussion.
\item[-] Bidragit till sammanfattning/abstract
\item[-] Skrivit stora delar av inledningen.
\item[-] Huvudansvarig för hela jämförelsen med andra energibesparande åtgärder.
\item[-] Skrivit utkast till resten av diskussion samt slutsats.
\end{itemize}




\subsection*{Ylva Dahl}

\paragraph{Ansvarsområden}

\begin{itemize}
\item[-] Läst på om fastigheten och dess väderstation.
\item[-] Studerat in begreppen svartkroppsstrålning, free-running temperature, ekvivalent temperatur samt sökt information kring naturkonstanter för luft.
\item[-] Samlat in och sammanställt väderdata främst från SMHI:s databas för våra exempel.
\item[-] Anvarig för mötesprotokoll.
\end{itemize}

\paragraph{Bidrag till problemlösning, syntes och analys}
\begin{itemize}
\item[-] Summerat energiflöden.
\end{itemize}

\paragraph{Huvudansvarig författare av avsnitt}
\begin{itemize}
\item[-] Har haft det redaktionella huvudansvaret för kapitlet Resultat.
\item[-] Skrivit om fastigheten och dess väderstation.
\item[-] Presenterat begreppen svartkroppsstrålning, free-running temperature, ekvivalent temperatur och naturkonstanter för luft.
\item[-] Skrivit text till bilderna i resultatet – väggar, burspråk, tak och grund – och sammanfattat i bild och text.
\item[-] Skrivit dispositionen och introduktioner till avsnitten teori och resultat samt diskussionen av resultaten.
\end{itemize}


\subsection*{Mats Lindström}

\paragraph{Ansvarsområden}

\begin{itemize}
\item[-] Flöden genom fönster respektive tak.
\item[-] Teori om konvektion och värmeledning.
\item[-] Utformning av rapport i \LaTeX, inklusive BibTex.
\end{itemize}

\paragraph{Bidrag till problemlösning, syntes och analys}

\begin{itemize}
\item[-] Program för beräkning av solens position
\item[-] Skapande av modell för solinstrålning genom fönster.
\end{itemize}

\paragraph{Huvudansvarig författare av avsnitt}

\begin{itemize}
\item[-] Har haft det redaktionella huvudansvaret för kapitlet Teoretiskt ramverk.
\item[-] Har skrivit härledningar till konvektionsekvationerna, värmeledningsekvationerna och reflektioner i fönster, samt förklaring av vindens konvektionskoefficient.
\item[-] Presenterat teori, metod och resultat av flöden genom fönster via solinstrålning, långvågsstrålning, värmeledning och konvektion.
\end{itemize}


\subsection*{Dan Ståby}

\paragraph{Ansvarsområden}

\begin{itemize}
\item[-] Huvudansvar för metodutveckling och beräkningar.
\item[-] Ansvar för skötsel av online-lagring samt distribuering av planeringsfiler samt råfiler för raporten
\end{itemize}

 \paragraph{Bidrag till problemlösning, syntes och analys}

\begin{itemize}
\item[-] Förstuderat och läst in på Finita elemntmetoden.
\item[-] Satt upp randvillkor för alla modeller
\item[-] Implementerat finita-elementlösningarna i \emph{Matlab} och \emph{Comsol Multiphysics}.
\item[-] Byggt matlabkod för fyllda contour-grafer för triangulära element.
\item[-] Hjälpt till att beräkna hur mycket energi vi kan spara på att ta hänsyn till väder.
\item[-] Räknat på långvågig strålning genom fönster och reflektioner
\end{itemize}

\paragraph{Huvudansvarig författare av avsnitt}

\begin{itemize}
\item[-] Redaktionellt huvudansvar för Metod-avsnittet
\item[-] Skrivit teori och metod om finita elementlösningarna
\item[-] Skrivit teori om Newton-Raphsons metod.
\item[-] Skrivit rekommendationer till fortsatt arbete
\item[-] Skrivit diskussion om metod
\item[-] Producerat grafer för värmeledning genom väggar och grund. Även producerat
grafer för temperaturfördelning utanför en vägg.
\end{itemize}

\end{document}

\section{Tillämpningar}

De två primära tillämpningarna av det här arbetet är att identifiera energiläckor för att kunna göra sitt hus mer energieffektivt och att låta sitt energiförsörjningssystem vara väderberoende.

Att energieffektivisera sitt hus kan, om man väljer rätt metod, löna sig både ekonomiskt och miljömässigt. Det leder dessutom till mindre fluktuationer i inomhustemperaturen, speciellt om man har stor tröghet i sitt uppvärmningssystem.

Att låta energiförsörjningssystemet bero av väderdata från en vid fastigheten monterad väderstation kräver experimentella mätning på den aktuella fastigheten för att bli implementerbart.
Dessutom kommer direkta eneriflöden, så som solinstrålning och vind, att kompenseras för fördröjt, vilket kan bli ett problem om man har ett långsamt system för uppvärmning. Ett alternativ då är prognosstyrning men de färdiga system som finns på marknaden idag är så pass bra att det inte är av intresse att försöka bygga ett eget, inte ur ekonomisk synvinkel i alla fall.

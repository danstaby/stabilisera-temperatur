\chapter*{Sammanfattning}
% På svenska här

% Avsikten med arbetet
Avsikten var att undersöka vilka hur energi flödar genom en fastighet beroende på vilket väder det är. Detta syftar i sin tur till att kunna minska fastighetens löpande energikostnader samt att få ett jämnare inomhusklimat. Därför presenteras även några olika metoder samt kostnadsförslag till dessa för hur energiflödena kan minskas nämligen isolering av de ännu oisolerade väggarna, SMHI:s prognosstyrningstjänst samt termostater inne i lägenheterna.

% metoden som användes
Fastigheten har därför delats upp i byggnadsskalet med väggar, tak och fönster och grunden där vi undersökt energiflödet genom var och en av delarna för sig. Solinstrålning genom fönstren och ofrivillig ventilation, alltså drag på grund av vind, behandlas också separat.

Termisk energi överförs genom strålning, konvektion och ledning och i varje beräkning av fastighetens energiflöde är det viktigt att undersöka dem alla. För detta har vi använt både analytiska beräkningsmetoder och simuleringar i Matlab och i Comsol. En återkommande beräkningsmetod är finita elementmetoden och givetvis är även värmeledningsekvationen en central del.

% vilka resultat som erhölls samt 
Vi har kunnat konstatera att 

% vilka slutsatser som dragits

% INTE DISKUSSION! 

%\newpage

\chapter*{Abstract}
% På engelska här

\newpage

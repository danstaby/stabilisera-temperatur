\subsection{Finita element av inkompressibel fluid}

För att lösa Navier-Stokes ekvationer kan lämpligen en datormodell användas.
Här består denna modell av ett system uppsatt med Galerkins metod.
I denna lösning så begränsar vi dock oss till att enbart behandla statiska flöden
då detta kan ses som en okej approxiation.

Om ett system är statiskt kan vi sätta alla tidsderivator i ekvationerna
\eqref{eq:convection:continuity}-\eqref{eq:convection:energy} till noll.
För att hantera trycket används den tidigare nämnda Boussinesq approximationen
samt penalty metoden för att göra hastighetsvektorn källfri. Den bas det
senare kommer att integreras över är inte källfri. Detta då den är bitvis
slät och består av linjära plan. Penaltymetoden ger tillåtelse att snirkla
runt detta faktum.

Genom penaltymetoden beskrives här trycket som $p$ enligt ekvation
\eqref{eq:femconvection:penalty}. Här är $p_s$ någon form av idealt statiskt
tryck som är önskat. Detta tryck följer Boussinesq approximation. Med dessa
idealiseringar kan differentialekvationerna sättas upp igen. \cite{heinrich88}

\begin{equation}
\label{eq:femconvection:penalty}
p = p_s - \lambda\nabla\cdot\mathbf{v}
\end{equation}


\begin{equation}
\label{eq:femconvection:u}
\mathbf{v}\cdot\nabla u =
\frac{\lambda}{\rho_0}\nabla\cdot\mathbf{v} +
\nu\Delta u
\end{equation}

\begin{equation}
\label{eq:femconvection:w}
\mathbf{v}\cdot\nabla w =
\frac{\lambda}{\rho_0}\nabla\cdot\mathbf{v} + \nu\Delta w +g\beta(T-T_0)
\end{equation}

\begin{equation}
\label{eq:femconvection:T}
\mathbf{v}\cdot\nabla T = \alpha\Delta T
\end{equation}


\subsubsection{Triangulär bas}
För att använda Galerkins metod måste en bas väljas. En som är populär samt
enkel att räkna på är linjära triangulära element. 

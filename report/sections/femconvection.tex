\section{Finita element av inkompressibel fluid}
\label{sec:femconvection}
För att lösa Navier-Stokes ekvationer som återfinns i avsnitt
\ref{section:convection} kan lämpligen en datormodell användas.
Här består denna modell av ett system uppsatt med Galerkins metod.
Denna lösning begränsas dock till att enbart behandla statiska flöden
vilket genomförs genom att sätta alla tidsderivator till noll.

För att hantera trycket i Navier-Stokes ekvationer
används den tidigare nämnda Boussinesqs approximation
och penalty-metoden. De gör hastighetsvektorn källfri och att
kontinuitetsekvationen uppfylls. Det finns således inget direkt behov av att räkna ut trycket.
Vid användning av många olika sorters elementtyper som inte uppfyller Babuska-Brezzikriteriet
är detta dessutom nödvändigt då ekvationssystemet som skall lösas blir singulärt. 
Detta beror på att trycket inte existerar i lika många ekvationer som hastighetskomponenterna.
Finns det då fler okända trycknoder än hastighetsnoder finns ingen entydigt definierad lösning
till ekvationssystemet. En annan möjlighet för att lösa problemet är att välja element som
uppfyller Babuska-Brezzikriteriet och det uppstår fler okända hastighetsnoder än trycknoder.
\cite{babuska1973}\cite{segal2011}

Genom penalty-metoden beskrivs här trycket som $p$ enligt 

\begin{equation}
\label{eq:femconvection:penalty}
p = p_s - \lambda\nabla\cdot\mathbf{v}.
\end{equation}

Här är $p_s$ någon form av önskat idealt statiskt
tryck. Detta tryck följer Boussinesqs approximation.
\cite{heinrich88}\cite{taylor79}
Som kan ses leder den godtyckliga penaltyparametern $\lambda$ till en justering trycket
om hastighetsfältets divergens inte är identiskt med noll. I en del litteratur anges 
det att penaltyparametern skall vara i storleksordningen $\lambda = \unit[10^7]{Pa~s}$ men att den
är väldigt applikationsberoende. En för liten vald penaltyparameter leder till att
trycket inte elimineras. Andra problem uppstår vid en för stor parameter såsom att ekvationssystemet
kan bli svårlöst och få stabilitetsproblem när parametern blir
för stor i jämförelse med de andra delarna i differentialekvationen.\cite{reddy93}\cite{roy05}\cite{basak04}\cite{segal2011}

Fortsatt skall trycket deriveras med avseende på de rumsliga variablerna vilket möjliggör
att eliminera trycket från differentialekvationerna, vilket resulterar i

\begin{align}
\label{eq:femconvection:partx}
\frac{\partial p}{\partial x} &= \frac{\partial p_s}{\partial x} -
\frac{\partial}{\partial x} \lambda\nabla\cdot\mathbf{v} = -
\frac{\partial}{\partial x} \lambda\nabla\cdot\mathbf{v}
\\
\label{eq:femconvection:partz}
\frac{\partial p}{\partial z} &= \frac{\partial p_s}{\partial z} -
\frac{\partial}{\partial z} \lambda\nabla\cdot\mathbf{v} =
-g\rho_0 - \frac{\partial}{\partial z} \lambda\nabla\cdot\mathbf{v}.
\end{align}

Notera att det statiska trycket
$p_s$ ej beror på $x$ vilket resulterar i att derivatan är noll.

Detta förs in i momentekvationerna vilket ger ekvationerna

\begin{align}
\label{eq:femconvection:u}
\mathbf{v}\cdot\nabla u -
\frac{\lambda}{\rho_0}\frac{\partial}{\partial x}\nabla\cdot\mathbf{v} -\nu\Delta u &= 0
\\
\label{eq:femconvection:w}
\mathbf{v}\cdot\nabla w -
\frac{\lambda}{\rho_0}\frac{\partial}{\partial z}\nabla\cdot\mathbf{v}
- \nu\Delta w - g\beta(T-T_0) &= 0 \\ 
\label{eq:femconvection:T}
\mathbf{v}\cdot\nabla T - \alpha\Delta T &= 0
\end{align}

vilket är det ekvationssystem $\mathbf{L}(u,w,T) = 0$
som ska lösas.

%%%%
Nu skapas en svag formulering som är ett variationsproblem där $(T,u,w)$ söks
så att de satisfierar

\begin{equation}
\label{eq:femconvection:variation}
\int_\Omega \mathbf{L}(T,u,w) \phi \mathrm{d}\Omega = 0\mbox{,  } \forall \phi \in \Phi.
\end{equation}

De okända funktionerna ska här vara en linjärkombination av testfunktionerna.

Då differentialekvationerna innehåller andra ordningens deriveringsoperatorer i form av
partialderiveringsoperatorn och divergensoperatorn behöver de analyseras ytterligare.
Här används divergensteoremet för att eliminera en derivering. Efter detta
förs differentialekvationerna in i den svaga formuleringen i ekvation
\eqref{eq:femconvection:variation}.
Det ger

\begin{align}
\int_\Omega \left(\phi\mathbf{v}\cdot\nabla u +
\frac{\partial \phi}{\partial x}\frac{\lambda}{\rho_0}\nabla\cdot\mathbf{v}
+\nu\nabla\phi\nabla u\right)\mathrm{d}\Omega = \nonumber \\
\int_\Gamma\left( \nu\phi\nabla u\cdot\mathbf{n} +
n_x\frac{\lambda}{\rho_0}\phi\nabla\cdot\mathbf{v}\right)\mathrm{d}\Gamma
\label{eq:femconvection:weaku}
\end{align}
\begin{align}
\int_\Omega\left(\phi\mathbf{v}\cdot\nabla w
+ \frac{\partial \phi}{\partial z} \frac{\lambda}{\rho_0}\nabla\cdot\mathbf{v}
+ \nu\nabla\phi\cdot\nabla w + \phi g\beta(T_0-T)\right)\mathrm{d}\Omega
= \nonumber \\
\int_\Gamma\left(\nu\phi\nabla w\cdot\mathbf{n} +
n_z\phi\frac{\lambda}{\rho_0}\nabla\cdot\mathbf{v}\right)\mathrm{d}\Gamma
\label{eq:femconvection:weakw}
\end{align}
\begin{equation}
\int_\Omega\left(\phi\mathbf{v}\cdot\nabla T + \alpha\nabla\phi\nabla T\right)\mathrm{d}\Omega
= \int_\Gamma \alpha\phi\nabla T\cdot\mathbf{n}\mathrm{d}\Gamma.
\label{eq:femconvection:weakT}
\end{equation}


Kvar är då att faktiskt lösa problemet.
Ansätt en lösning på formen

\begin{align}
u(x,z) &= \sum^N_{k=1}u_k\phi_k(x,z)
\nonumber \\
w(x,z) &= \sum^N_{k=1}w_k\phi_k(x,z)
\nonumber \\
T(x,z) &= \sum^N_{k=1}T_k\phi_k(x,z).
\label{eq:femconvection:ansatz}
\end{align}

Då $\phi_k \in \Phi$, $k=1,~2,~..,~N$ så uppfylls även att $u,~w$ och $T$ tillhör
$\Phi$ då dessa är en linjärkombination av basen $\phi_k$. Om konstanterna
är dimensionerade så att dessa uppfyller $\mathbf{L}(u,w,T) = 0$ är detta
en lösning till variationsproblemet. \cite{johnson2009}\cite{heath2002}\cite{lewis04}

Ansatsen kan nu sättas in i den svaga formuleringen.
Alla förekomster av $\phi$ byts ut mot $\phi_i$ med $i=1,~2,~..,~N$. Då
kontinuitetskriteriet~\eqref{eq:convection:continuity} säger att
$\nabla\cdot\mathbf{v} = 0$ sätts detta på randen vilket ger att den
ena termen integralerna på höger sida i ekvationerna~\eqref{eq:femconvection:weaku} och
\eqref{eq:femconvection:weakw}
försvinner. Slutligen ger
detta galerkinformuleringen i ekvationerna

\begin{align}
\label{eq:femconvection:galerkinu}
\sum^N_{j=1}\sum^N_{k=1}u_ju_k\int_\Omega \phi_i\phi_j\nabla\phi_k \mathrm{d}\Omega &+
\sum^N_{j=1}\sum^N_{k=1}w_ju_k\int_\Omega \phi_i\phi_j\nabla\phi_k \mathrm{d}\Omega  \\ +
\nonumber
\sum^N_{j=1} u_j\int_\Omega\frac{\lambda}{\rho_0}
\frac{\partial \phi_i}{\partial x}\frac{\partial \phi_j}{\partial x} \mathrm{d}\Omega &+
\sum^N_{j=1} w_j\int_\Omega\frac{\lambda}{\rho_0}\frac{\partial \phi_i}{\partial x}
\frac{\partial \phi_j}{\partial z} \mathrm{d}\Omega  \\ +
\nonumber
\sum^N_{j=1} u_j \int_\Omega \nu\nabla\phi_i\nabla\phi_j \mathrm{d}\Omega &=
\int_\Gamma \nu\phi_i\nabla u \cdot \mathbf{n} \mathrm{d}\Gamma
\end{align}

\begin{align}
\label{eq:femconvection:galerkinw}
\sum^N_{j=1}\sum^N_{k=1}u_jw_k\int_\Omega \phi_i\phi_j\nabla\phi_k \mathrm{d}\Omega &+
\sum^N_{j=1}\sum^N_{k=1}w_jw_k\int_\Omega \phi_i\phi_j\nabla\phi_k \mathrm{d}\Omega \\ +
\nonumber
\sum^N_{j=1} u_j\int_\Omega \frac{\lambda}{\rho_0}\frac{\partial \phi_i}{\partial z}
\frac{\partial \phi_j}{\partial x} \mathrm{d}\Omega &+
\sum^N_{j=1} w_j\int_\Omega \frac{\lambda}{\rho_0}
\frac{\partial \phi_i}{\partial z}\frac{\partial \phi_j}{\partial z} \mathrm{d}\Omega \\ +
\nonumber
\sum^N_{j=1} w_j \int_\Omega \nu\nabla\phi_i\nabla\phi_j \mathrm{d}\Omega &+
\sum^N_{j=1}T_j \int_\Omega - g\beta\phi_i\phi_j \mathrm{d}\Omega \\
\nonumber
= \int_\Gamma \nu\phi_i\nabla w \cdot \mathbf{n} \mathrm{d}\Gamma &-
\int_\Omega T_0g\beta\phi_i\mathrm{d}\Omega
\end{align}

\begin{align}
\label{eq:femconvection:galerkinT}
\sum^N_{j=1}\sum^N_{k=1}u_jT_k\int_\Omega \phi_i\phi_j\nabla\phi_k \mathrm{d}\Omega &+
\sum^N_{j=1}\sum^N_{k=1}w_jT_k\int_\Omega \phi_i\phi_j\nabla\phi_k \mathrm{d}\Omega \\ +
\nonumber
\sum^N_{j=1} T_j \int_\Omega \alpha\nabla\phi_i\nabla\phi_j \mathrm{d}\Omega &=
\int_\Gamma \alpha\phi_i\nabla w \cdot \mathbf{n} \mathrm{d}\Gamma.
\end{align}


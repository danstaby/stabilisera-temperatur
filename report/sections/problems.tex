\section{Problem som vi stött på}

I början av arbetet hoppades vi att fastighetens väderstation skulle kunna tillhandahålla väderdata som vi kunde använda för att sätta upp statistiska modeller. 
Ganska snart insåg vi dock att detta inte skulle bli verklighet och vi planerade istället att sätta upp analytiska modeller. 
Senare visade sig dock att statistiken hade varit bra för att närmare kunna undersöka hur stor inverkan olika väderparametrar har vid just fastigheten på Walleriusgatan. 
Detta fick istället lösas med statistik från SMHIs hemsida, som dock visade sig vara något otillräcklig.

För att sätta upp de analytiska modellerna behövdes också värden för ett antal materialkonstanter för luft och för tegel. 
Dessa beror dock av en eller flera väderparametrar så som tryck, temperatur och fukt och således är det tveksamt om de överhuvudtaget bör kallas konstanter. 
Det visade sig dock vara mycket svårt att få fram tillförlitlig och tillräckligt högupplöst data som visade detta än mindre uttryck av ekvationskaraktär. 
Vid diskussion med sakkunniga fick vi uppfattningen att det finns en tydlig konsensus inom branchen vad som är relevant och inte, men ingen tydlig dokumentation på området.

Konvergens. Svårt att få våra datamodeller att konvergera. 

Svårt att få exakta värden på kostnader efter som vi är tredje part.

\section{Värmeledning}\label{sec:heatconduction}
Konduktion. Det pågår ständigt värmetransport från kalla till varma sidor. Värmetransporten är proportionell mot temperaturskillnaden över konstruktionen. Konduktion, eller ledning genom ett material, här värmeledning innebär att värme flyttar sig inom ett material, utan att materialet i sig rör sig eller flyttar sig. Konduktiviteten, eller värmeledningsförmågan, betecknad kappa inom fysiken eller lambda inom byggsektorn är en materialegenskap. Eftersom överföring av värme i form av ledning är den enklaste överföringen är det den som lambda värdet bygger på. För att bestämma lambdavärdet för ett material utsätter man det för en temperaturskillnad och mäter den värmemängd som passerar genom materialet per tidsenhet. Då värmeöverföring via konvektion samt strålning är så pass svår att behandla matematiskt så ingår dessa i lambdavärdet som är väldigt enkelt att jobba med matematiskt. $q=k \nabla \cdot T$ vilket ger oss $Q=U \Delta \cdot T$ och därifrån kan vi lösa ut U-värdet som $U = \frac{Q}{\Delta\cdot T}$. Värmeledningsförmågan påverkas av materialets densitet, porositet, temperatur samt fuktighet. Fuktkorrigering görs, enligt vissa framställda värden. Värmemotstånd, R beräknas utifrån lambdavärdet.

\begin{equation}
R_i=\frac{d}{\lambda}
\end{equation}

Överföringsmotstånd. Man påför sedan Rsi samt Rse, utsida samt insida. Beror på både strålnings samt konvektion. U-värdet beräknas som inversen av R-värdet. Har man olika material handlar det om att summera ihop 1/r.

\section{Diskussion kring resultatet}

% \emph{\color{red} Vad resultatet faktiskt säger. Vad betyder olika storlekar på energiflödet vid olika tidpunkter.}

På grund av approximationerna som beskrivs i avsnitt~\ref{sec:discmethod} är resultaten en uppskattning av energiflödenas storlek. Dessa ska ses som en fingervisning av vad man kan förvänta sig för energiflöden genom en byggnad i samma stil som den på Walleriusgatan vid olika årstider och väderlekar. Det intressanta är framför allt hur stora flödena är relativt varandra och hur de olika flödena förändras med de olika väderparametrarna.

En stor del av energiflödet in, men framför allt ut, genom byggnaden gick genom fönstren. Det var inte något som uppmärksammades förrän under senare delen av arbetet. Det beroende mycket på att fastigheten har relativt moderna fönster och detta inte framstod som något som behövde åtgärdas. Det finns ett flertal olika metoder för att åtgärda energiflödet genom fönster som lämpar sig för olika byggnader och klimat och det finns inget enkelt sätt att säga vilken som är bäst för just den här fastigheten.

Det transienta flödet genom väggen beskriver hur snabbt väggarna reagerar på ett väderomslag. Det varierar väldigt mellan husets olika delar på grund av väggarnas olika egenskaper. Den välisolerade norrsidan har en större tröghet och en avsevärt längre reaktionstid än sydsidan och framför allt jämfört med burspråket, som också ligger på sydsidan. Sker reglering utan hänsyn till detta riskeras stor ojämnhet mellan rummen på vardera sidan av byggnaden.

Resultatet tar inte hänsyn till eventuell fukt eller regn. Troligen tar väggarna åt sig fukt vid blött väder vilket kan ändra väggens värmeledande och värmelagrande egenskaper. Regn kan också tänkas störa luftflödena kring byggnaden och även direkt kyla den genom att vara kallt och ha hög värmekapacitet. Den störta förlusten tros dock vara avdunstning. I det här projektet tas dock ingen hänsyn till det. Vid uppehåll och torrt väder borde resultaten vara fullt tillämpbara.


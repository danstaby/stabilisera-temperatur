\section{Diskussion kring resultatet}

\emph{\color{red} Vad resultatet faktiskt säger. Vad betyder olika storlekar på energiflödet vid olika tidpunkter.}

Resultatet tar ingen hänsyn till eventuell fukt eller regn. Troligen tar väggarna åt sig fukt vid blött väder vilket kan ändra väggens värmeledande och värmelagrande egenskaper. Regn kan också tänkas störa luftflödena kring byggnaden och även direkt kyla den genom att vara kallt och ha hög värmekapacitet. Den störta förlusten tros dock vara avdunstning. I det här projektet tas dock ingen hänsyn till det. Vid uppehåll och torrt väder borde våra resultat vara fullt tillämpbara.
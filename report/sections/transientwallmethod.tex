\section{Transient värmeflöde genom en vägg}

När utomhustemperaturen förändras förändras också väggens temperatur vilket påverkar energiflödena genom väggen. Om vill man hålla konstant inomhustemperatur påverkar det även fastighetens energianvändning. Här gör vi en jämförelse mellan en oisolerad, 50 cm tjock tegelvägg, motsvarande den som finns på fastighetens södersida, och en 50 cm tjock tegelvägg med 10 cm isolering, motsvarande den som finns på fastighetens norrsida. Simuleringen har gjorts med finitna elementmetoden med utgångspunkt i ekvationen

\begin{equation}
c\rho \frac{\partial T}{\partial t} = \nabla \cdot (k \nabla T)
\end{equation}

där $T=T(x,t)$ är väggens temperatur och både $\rho$ och $k$ kan vara $x$-beroende.


\section{Sammanfattning av energiflöden: free-running temperature}

\emph{\color{red} Jag är medveten om att mycket saknas och att det absolut inte är ett lämpligt sätt att sammanställa det på. Detta är ett försök till ett första utkast för att se vad det blir och vad som saknas. Eventuellt är det är helt fel tillvägagångssätt.}

De olika energiflödena är alltså från fasta energikällor, flöde genom väggarna, burspråk och tak, genom grunden, på grund av solinstrålning genom fönster samt på grund av ofrivillig ventilation.

Från fasta energikällor fås alltid ett bidrag på totalt $\unit[10,5]{kW}$, se avsnitt \ref{sec:constsources}.

Area västerväggen 61 och area söderväggen 151 => 212\\
Area norrväggen 253\\
Area burspråk 47\\
Area fönster: 206 

\paragraph{I december:}
UTFLÖDET\\

Från burspråket fås –
\textbf{kl 5:} 19 W/m2
\textbf{kl 15:} 15 W/m2

Från fönstren fås –
\textbf{kl 5:} 
\textbf{kl 15:} 

Från flöde genom väggarna fås (oisolerat/isolerat) –
\textbf{\textbf{kl 5:}} 32,2 / 7,88 W/m2
\textbf{kl 15:} 33,3 / 7,91 W/m2

%%
Fönster –
\textbf{kl 5:} 206*
\textbf{kl 15:} 206*

Burspråket – 
\textbf{kl 5:} 47*19=893 W
\textbf{kl 15:} 47*15=795 W

Norrväggen (alltid isolerad) –
\textbf{kl 5:} 253*7,88=1994 W
\textbf{kl 15:} 253*7,91=2001 W

Söder- och västerväggarna (utan isolering) –
\textbf{kl 5:} 212*32,2=6826 W
\textbf{kl 15:} 212*33,3=7060 W

Söder- och västerväggarna (isolerade) –
\textbf{kl 5:} 212*7,88=1671 W
\textbf{kl 15:} 212*7,91=1680 W

Summa vägg, utan isolering –
\textbf{kl 5:} 1994+893+6826+ ?? =9713 W
\textbf{kl 15:} 2001+795+7060 ?? =9856 W

Summa vägg, med isolering –
\textbf{kl 5:} 1994+893+1671+ ?? =4558 W
\textbf{kl 15:} 2001+795+1680+ ?? =4476 W

I december är energiutflödet genom grunden ungefär 12,1 W/m2. Grunden är 286 m2.\\
Energiutflödet genom grunden är 3461 W.

En molning dag sker ingen solinstrålning genom rutan.

Detta ger ett totalt energiflöde ut ur fastigheten en molnig, vindstilla, dag i december \textbf{kl 5:}\\
10500 + VÄGG (inkl fönster)+3489 = ?? utan isolering.\\
10500 + VÄGG (inkl fönster)+3489 = ?? med isolering.\\

Detta ger ett totalt energiflöde ut ur fastigheten en molnig, vindstilla, dag i december \textbf{kl 15:}\\
10500 + VÄGG (inkl fönster)+3489 = ?? utan isolering.\\
10500 + VÄGG (inkl fönster)+3489 = ?? med isolering.\\

%%%%%%%%%%%%%%%%%%%%%%%%%%%%%%%%%%%%%%%

\paragraph{I april, soligt:}
UTFLÖDET\\

Från fönstren fås –
\textbf{kl 5:} 
\textbf{kl 15:} 

Från burspråket fås –
\textbf{kl 5:} 10 W/m2
\textbf{kl 15:} -15 W/m2

Från flöde genom väggarna fås (oisolerat/isolerat) –
\textbf{kl 5:} -5 W/m2 / -0,9 W/m2
\textbf{kl 15:} 1 W/m2 / -0,4 W/m2

%%

Fönstren –
\textbf{kl 5:} 206*
\textbf{kl 15:} 206*

Burspråket – 
\textbf{kl 5:} 47*10= 470 W
\textbf{kl 15:} 47*-15= -705 W

Norrväggen (alltid isolerad) –
\textbf{kl 5:} 253*-0,9 = -228 W
\textbf{kl 15:} 253*-0,4 = -101 W

Söder- och västerväggarna (utan isolering) –
\textbf{kl 5:} 212*-5 = -1060 W
\textbf{kl 15:} 212*1 = 212 W

Söder- och västerväggarna (isolerade) –
\textbf{kl 5:} 212*-0,9 = -191 W
\textbf{kl 15:} 212*-0,4 = -85 W

Summa vägg, utan isolering –
\textbf{kl 5:} 
\textbf{kl 15:} 

Summa vägg, med isolering –
\textbf{kl 5:} 
\textbf{kl 15:} 

I april är energiutflödet genom grunden ungefär 12,5 W/m2. Grunden är 286 m2.\\
Energiutflödet genom grunden är 3575 W.

En solig dag i april sker ingen solinstrålning kl 5, men 540 W/m2 kl 15. Det motsvarar 0 W kl 5 respektive 81540 W in genom fönstren på södersidan.

Detta ger ett totalt energiflöde ut ur fastigheten en solig, vindstilla, dag i april \textbf{kl 5:}\\
10500 + VÄGG (inkl fönster)+3489 = ?? utan isolering.\\
10500 + VÄGG (inkl fönster)+3489 = ?? med isolering.\\

Detta ger ett totalt energiflöde ut ur fastigheten en molnig, solig, dag i april \textbf{kl 15:}\\
10500 + VÄGG (inkl fönster)+3489 - 81540 = ?? utan isolering.\\
10500 + VÄGG (inkl fönster)+3489 - 81540 = ?? med isolering.\\


%%%%%%%%%%%%%%%%%%%%%%%%%%%%%%%%%%%%%%%

\paragraph{I april, molnigt:}
UTFLÖDET\\

Från fönstren fås –
\textbf{kl 5:} 
\textbf{kl 15:} 

Från burspråket fås –
\textbf{kl 5:} 9,1 W/m2
\textbf{kl 15:} 1,5 W/m2

Från flöde genom väggarna fås (oisolerat/isolerat) –
\textbf{kl 5:} 7,3 W/m2 / 2,11 W/m2
\textbf{kl 15:} 9,8 W/m2 / 2,26 W/m2

%%

Fönstren –
\textbf{kl 5:} 206*
\textbf{kl 15:} 206*

Burspråket – 
\textbf{kl 5:} 47*9,1 = 428 W
\textbf{kl 15:} 47*1,5 = 71 W

Norrväggen (alltid isolerad) –
\textbf{kl 5:} 253*2,11 = 534 W
\textbf{kl 15:} 253*2,26 = 572 W

Söder- och västerväggarna (utan isolering) –
\textbf{kl 5:} 212*7,3 = 
\textbf{kl 15:} 212*9,8 = 

Söder- och västerväggarna (isolerade) –
\textbf{kl 5:} 212*2,11 = 1548 W
\textbf{kl 15:} 212*2,26 = 2078 W

Summa vägg, utan isolering –
\textbf{kl 5:} 
\textbf{kl 15:} 

Summa vägg, med isolering –
\textbf{kl 5:} 
\textbf{kl 15:} 

I april är energiutflödet genom grunden ungefär 12,5 W/m2. Grunden är 286 m2.\\
Energiutflödet genom grunden är 3575 W.

En molning dag sker ingen solinstrålning genom rutan.

Detta ger ett totalt energiflöde ut ur fastigheten en molnig, vindstilla, dag i april \textbf{kl 5:}\\
10500 + VÄGG (inkl fönster)+3489 = ?? utan isolering.\\
10500 + VÄGG (inkl fönster)+3489 = ?? med isolering.\\

Detta ger ett totalt energiflöde ut ur fastigheten en molnig, vindstilla, dag i april \textbf{kl 5:}\\
10500 + VÄGG (inkl fönster)+3489 = ?? utan isolering.\\
10500 + VÄGG (inkl fönster)+3489 = ?? med isolering.\\

%%%%%%%%%%

Husets volym: 35000 m3 = 35000000000 cm3\\
volymetrisk värmekapacitet luft = 0.00121J / cm3K

Luften i huset innehåller: 1.2409e+10 J


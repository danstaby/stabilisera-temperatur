\section{Sammanfattning av energiflöden: energibalanser}

De olika energiflödena är alltså från fasta energikällor, flöde genom väggarna, burspråk och tak, genom grunden, på grund av solinstrålning genom fönster samt på grund av ofrivillig ventilation. Den sista punkten, som orsakas av vind, utelämnas i ett första steg.

Från fasta energikällor fås alltid ett bidrag på totalt $\unit[10,5]{kW}$, se avsnitt \ref{sec:constsources}.

Ur figurerna i avsnitt \ref{sec:steadystatewall} får vi energiflödena per kvadratmeter genom de olika avsnitten av klimatskalet och med hjälp av areorna i tabell \ref{tbl:uvalue} fås det totala energiutflödet genom husets hela klimatskal.

Flödet genom grunden fås ur figur \ref{fig:cooling_ground} och solinstrålningen genom fönster en solig dag fås ur figur \ref{fig:effekt0415and1231}.


\paragraph{I december:}
UTFLÖDET\\
Summa vägg, utan isolering –\\
\textbf{kl 5:} 6386+893+1994+6826=16099 W\\
\textbf{kl 15:} 5150+795+2001+7060=15006 W\\

Summa vägg, med isolering –\\
\textbf{kl 5:} 6386+893+1994+1671=10944 W\\
\textbf{kl 15:} 5150+795+2001+1680=9626 W\\

I december är energiutflödet genom grunden ungefär 12,1 W/m2. Grunden är 286 m2.\\
Energiutflödet genom grunden är 3461 W.

En molning dag i december sätts till 20\% av maximal instrålning, vilket ger $\unit[60]{Wm^{-2}}$. Utstrålning ur fönster?

Detta ger ett totalt energiflöde ut ur fastigheten en molnig, vindstilla, dag i december \textbf{kl 5:}\\
10500 - 16099 - 3461 = -9060 utan isolering.\\
10500 - 10944 - 3461 = -3905 med isolering.\\

Detta ger ett totalt energiflöde ut ur fastigheten en molnig, vindstilla, dag i december \textbf{kl 15:}\\
10500 -15006 - 3461 = -7967 utan isolering.\\
10500 - 9626 - 3461 = -2587 med isolering.\\

%%%%%%%%%%%%%%%%%%%%%%%%%%%%%%%%%%%%%%%

\paragraph{I april, soligt:}
UTFLÖDET\\
Summa vägg, utan isolering –\\
\textbf{kl 5:} 2884+470-228-1060=2066 W\\
\textbf{kl 15:} 2266-705-101+212=1672 W\\

Summa vägg, med isolering –\\
\textbf{kl 5:} 2884+470-228-191=2935 W\\
\textbf{kl 15:} 2266-705-101-85=1375 W\\

I april är energiutflödet genom grunden ungefär 12,5 W/m2. Grunden är 286 m2.\\
Energiutflödet genom grundxen är 3575 W.

En solig dag i april sker ingen solinstrålning kl 5, men 540 W/m2 kl 15. Det motsvarar 0 W kl 5 respektive 81540 W in genom fönstren på södersidan.

Detta ger ett totalt energiflöde ut ur fastigheten en (solig) vindstilla, dag i april \textbf{kl 5:}\\
10500 - 2066 - 3575 =  4859 W, utan isolering.\\
10500 - 2935 - 3575 =  3990 W, med isolering.\\

Detta ger ett totalt energiflöde ut ur fastigheten en solig, dag i april \textbf{kl 15:}\\
10500 - 1672 - 3575 + 81540 = 86793 W utan isolering.\\
10500 - 1375 - 3575 + 81540  =  87090 W med isolering.\\


%%%%%%%%%%%%%%%%%%%%%%%%%%%%%%%%%%%%%%%

\paragraph{I april, molnigt:}
UTFLÖDET\\
Summa vägg, utan isolering –\\
\textbf{kl 5:} 2884+470-228-1060=2066 W\\
\textbf{kl 15:} 2266-705-101+212=1672 W\\

Summa vägg, med isolering –\\
\textbf{kl 5:} 2884+470-228-191=2935 W\\
\textbf{kl 15:} 2266-705-101-85=1375 W\\

I april är energiutflödet genom grunden ungefär 12,5 W/m2. Grunden är 286 m2.\\
Energiutflödet genom grunden är 3575 W.

En molning dag sker ingen solinstrålning genom rutan.

Detta ger ett totalt energiflöde ut ur fastigheten en solig, vindstilla, dag i april \textbf{kl 5:}\\
10500 - 2066 - 3575 =  4859 W, utan isolering.\\
10500 - 2935 - 3575 =  3990 W, med isolering.\\

Detta ger ett totalt energiflöde ut ur fastigheten en solig, dag i april \textbf{kl 15:}\\
10500 - 1672 - 3575 = 5253 W utan isolering.\\
10500 - 1375 - 3575  =  5550 W med isolering.\\

%%%%%%%%%%%%%%%%%%%%%%%%%%%%%%%%%%%%%%%%

\section{Diskussion kring metoden}\label{sec:discmethod}

I avsikt att nå projektets mål har en rad olika metoder använts. För att utföra beräkningar på strålning från både solen och
omgivningen har principen kring svartkroppsstrålning använts. För beräkningar på ledning utnyttjas termodynamikens nollte huvudsats
och dessa beräkningar genomförs med Fouriers värmelag och värmeledningsekvationen. Konvektion sker främst på grund av cirkulation i luften och för att beräkna hur
detta sker har ett flertal strömningsmekaniska metoder använts.

I början av projektet fanns det förhoppningar om att fastighetens väderstation skulle kunna tillhandahålla väderdata att använda för att sätta upp statistiska modeller. 
Ganska snart insågs dock att detta inte skulle bli verklighet och modeller blev istället vår metod. Senare visade sig att statistiken hade
varit bra för att närmare kunna undersöka hur stor inverkan olika väderparametrar har vid just fastigheten på Walleriusgatan. 
Detta fick istället lösas med statistik från SMHI:s hemsida. Statistiken var tyvärr något otillräcklig.

För att sätta upp de fysikaliska modellerna behövdes också värden för ett antal materialkonstanter för luft och för tegel. 
Dessa beror dock av en eller flera väderparametrar såsom tryck, temperatur och fukt och således är det tveksamt om de överhuvudtaget borde kallas konstanter. 
Det visade sig dock vara mycket svårt att få fram tillförlitlig och tillräckligt högupplöst data som visade detta, än mindre att hitta uttryck på ekvationsform.
Vid diskussion med sakkunniga fick vi uppfattningen att det finns en tydlig konsensus inom branschen vad som är relevant och inte, men ingen tydlig dokumentation på området.
Det har även varit ett stort problem att finna goda källor på vissa naturkonstanter. Värmeledningsförmåga och specifik värmekapacitet kan till exempel
variera mycket mellan olika tillverkare av ett visst material och ålder på materialet. Av denna anledning finns det även stora felkällor i val av dessa naturkonstanter. Det har även
varit svårt att hitta mätningar på precis rätt material eller mätningar på liknande material över huvud taget. Av denna anledning finns det ibland
källor av tveksam karaktär vilket även introducerar fel. För att få exakta värmden måste således mätningar genomföras på materialen i fastigheten. Valet av
naturkonstanter är ändock det bästa som gick att genomföra utan att ha tillgång till mätningar från fastigheten.

Vid lösning av de relevanta differentialekvationerna behövdes problemets geometri och randvillkor sättas upp. För
att förenkla beräkningarna har ofta antalet dimensioner i problemen reducerats. Bland annat så har väggarna antagits vara oändligt
långa vilket är ekvivalent med att de är perfekt isolerade på korsidorna, alltså alla andra sidor än utsidan och insidan. Detta stämmer inte
i praktiken och kan således vara en källa till fel. En liknande approximation genomfördes för grunden, där fastigheten
antagits vara oändligt lång. I det här fallet är den undersökta fastigheten mittendelen av en lång fastighet vilket
gör att approximationen är giltig men så pass grov att den ändå är en källa till fel. Bergets material antas vara homogen granit vilket
heller inte är helt korrekt. 

I alla lösningar har differentialekvationernas beteende utanför definitionsmängderna approximerats. Beteendet på randerna
är väldigt beroende på omgivningen och detta kommer även vara en stor källa till fel. Här har bland annat svartkroppsstrålning,
solinstrålning och konvektionsparametern approximerats. Dessa approximationer är en nödvändighet för att man inte ska behöva
simulera hela stadsdelen eller till och med staden vilket kommer bli väldigt beräkningsintensivt. Vi hoppas ändå att våra
värden ger en idé om hur det kan se ut i verkligheten och att det ger en god idé om hur stora energiflödena är i jämförelse med varandra.

Som ett extra steg har förslag på ytterligare energibesparande åtgärder tagits fram och dessa har jämförts med att ta hänsyn till väderparametrar på det sätt som projektet fått fram. För att sätta dessa i proportion till varandra har även konstnader för de olika åtgärderna tagits fram. Det har dock varit svårt att få exakta värden på kostnader då författarna för projektet är tredje part och inte direkt företräder fastighetens ansvariga.

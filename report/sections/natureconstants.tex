\section{Naturkonstanter för luft}

Vid olika tryck, fuktighet, temperatur och annat väder som påverkar. Luft består av 78,1\% kväve och 21,0\% syre (mer exakt http://en.wikipedia.org/wiki/Air\#Composition)

\subsection{Termisk diffusivitet}
In heat transfer analysis, thermal diffusivity is the thermal conductivity divided by density and specific heat capacity at constant pressure. It has the SI unit of $m^2/s$. The formula is:

\begin{equation}
\alpha = {k \over {\rho c_p}}
\end{equation}

where
\begin{itemize}
   \item[] k is thermal conductivity, $\unit{W/(m·K)}$
   \item[] $\rho$ is density, $\unit{kg/m^3}$
   \item[] $c_p$ is specific heat capacity, $\unit{J/(kg·K)}$
\end{itemize}

Air: $\unit[1.9\cdot10^{-5}]{m/s^2}$

Källor:\\
http://www.electronics-cooling.com/2007/08/thermal-diffusivity/\\
http://en.wikipedia.org/wiki/Thermal\_diffusivity\\

\subsubsection{Termisk konduktivitet}
The property of a material's ability to conduct heat. Unit: W/(m·K).

\subsubsection{Desitet}

https://www.brisbanehotairballooning.com.au/faqs/education/116-calculate-air-density.html

Gäller så länge ideala gaslagen gäller vilket den gör vid lufttryck. Källa?




\subsubsection{Specifik värmekapacitivitet}
kan approximeras med konstanten 1.005 kJ/(kg·K (källa: http://www.engineeringtoolbox.com/air-properties-d\_156.html) (för torr luft).

\subsection{Volymetriska expantionskoefficienten}

\subsection{Kinematiska viskositeten}

\subsection{Andra funderingare}
Fråga någon på Geofysik, Klimat, t.ex. någon här: http://www.gvc.gu.se/Forskning/klimat/stadsklimat/gucg/people/


\end{document}



\subsection{Naturkonstanter för luft}

Vid olika tryck, fuktighet, temperatur och annat väder som påverkar. Luft består av 78,1\% kväve och 21,0\% syre (mer exakt http://en.wikipedia.org/wiki/Air\#Composition)


\subsubsection{Termisk diffusivitet} % Viktig
Inom värmeöverförningsanalys är termisk diffusitet den termiska konduktivitetetn dividerat med densiteten och den specifika värmekapaciteten vid konstant tryck. Det har SI-enheten $m^2/s$. Formeln är:

\begin{equation}
\alpha=\frac{k}{\rho c_p}
\end{equation}

där 
\begin{itemize}
   \item[] k är termisk konduktivitetet, $\unit{W/(m·K)}$
   \item[] $\rho$ är densiteten, $\unit{kg/m^3}$
   \item[] $c_p$ specifika värmekapaciteten, $\unit{J/(kg·K)}$
\end{itemize}

Luft: $\unit[1.9\cdot10^{-5}]{m/s^2}$

Källor:\\
http://www.electronics-cooling.com/2007/08/thermal-diffusivity/\\
http://en.wikipedia.org/wiki/Thermal\_diffusivity\\

%%%%%

\subsubsection{Termisk konduktivitet (Värmeledningsförmåga)} % Viktigt
Ett materials förmåga att leda värme. Enhet: W/(m·K).\\
Vid rumstemperatur: $\unit[0,024]{W/(m·K)}$, uppmätt. ($\unit[0,031]{W/(m·K)}$ beräknat enligt Schroeder).

Värmeledningsförmågan ändras med temperaturen. För de flesta ämnen minskar den något med stigande temperatur. Vid låga temperaturer kan det även bero på trycket.

Ur Schroeder (s. 41ff) fås:
\begin{equation}
k_t=\frac{1}{2}\frac{C_V}{V} l \bar{v}
\end{equation}

där $\bar{v}$ är molekylernas genomsnittliga hastighet.

\begin{equation}
\frac{C_V}{V}=\frac{\tfrac{1}{2}Nk}{V}=\frac{f}{2}\frac{P}{T}
\end{equation}

där $f$ är antal frihetsgrader för en molekyl och $l\approx\frac{1}{4\pi r^2}\frac{V}{N}$. För luft i rumstemperatur är $f=5$ och $l=\unit[1,5\cdot 10^{-7}]{m}$.


%%%%%

\subsubsection{Densitet} % inte så viktigt
\label{sec:densitet}

https://www.brisbanehotairballooning.com.au/faqs/education/116-calculate-air-density.html

Gäller så länge ideala gaslagen gäller vilket den gör vid lufttryck. Källa?

%%%%%

\subsubsection{Specifik värmekapacitivitet}
kan approximeras med konstanten 1.005 kJ/(kg·K (källa: http://www.engineeringtoolbox.com/air-properties-d\_156.html) (för torr luft).

%%%%%

\subsubsection{Volumetriska expansionskoefficienten} % Viktig
Vattnets expansionskoefficient är $\unit[1.8\cdot 10^{-4}]{/K}$ (vid 'normalt' tryck). \emph{Källa: nationellt resurscentrum för fysik vid Lunds Universitet.} Även vid hög luftfuktighet är det relativt lite vatten i luften. Då vattnet utvidgas så lite borde det inte spela någon roll.

För torr luft borde det gå att använda allmänna gaslagen som ger ett förhållande mellan tryck, volym och temperatur.

%%%%%

\subsubsection{Kinematiska viskositeten} % Inte så viktig
Dynamisk viskositet luft:\\
$\unit[1,67\cdot10^{-5}]{Pa\cdot s}$, vid $\unit[0]{^\circ C}$.\\
$\unit[1,78\cdot10^{-5}]{Pa\cdot sh}$, vid $\unit[15]{^\circ C}$.


\subsubsection{Volymetriska expantionskoefficienten}

\subsubsection{Kinematiska viskositeten}

\subsubsection{Andra funderingare}
Fråga någon på Geofysik, Klimat, t.ex. någon här:\\
http://www.gvc.gu.se/Forskning/klimat/stadsklimat/gucg/people/


Den kinematiska viskositeten $\nu$ definieras som: $\nu = \frac {\mu} {\rho}.$


där $\mu$ är den dynamiska viskositeten och $\rho=\unit[1,2041]{kg/m^3}$ är lufts densitet, se avsnitt \ref{sec:densitet}.\\
\emph{(Källa Wikipedia, som hänvisar till Henrik Alvarez (1997). Energiteknik. Lund: Marinlitteratur.)}

Graf över beroende av tryck och temperatur:\\
http://en.wikipedia.org/wiki/File:Air\_dry\_dynamic\_visocity\_on\_pressure\_temperature.svg

Den dynamiska viskositeten för gasen kan anses vara oberoende av tryck i de flesta tekniska tillämpningar. \\
\emph{(källa: http://mac6.ma.psu.edu/stirling/simulations/DHT/ViscosityTemperatureSutherland.html)}

\subsubsection{Sutherland's formula}
Sutherlands formel kan användas för att bestämma den dynamiska viskositeten för en ideal gas som funktion av temperatur.\\
\emph{(källa: wikipeida som hänvisar till Alexander J. Smits, Jean-Paul Dussauge Turbulent shear layers in supersonic flow, Birkhäuser, 2006, ISBN 0387261400 p. 46)}

\begin{equation}
{\mu} = {\mu}_0 \frac {T_0+C} {T + C} \left (\frac {T} {T_0} \right )^{3/2}.
\end{equation}

Luft: C=120K, $T_0=\unit[291.15]{K}$, $\mu_0=\unit[18.27]{\mu Pa s}$

Riktig för temperaturer 0 < T < 555 K med ett fel på grund av tryck på mindre än 10\% under 3.45 MPa.


%%%%%

\subsubsection{Andra funderingar}
\begin{itemize}
\item[-] Fråga någon på Geofysik, Klimat, t.ex. någon här: \\
http://www.gvc.gu.se/Forskning/klimat/stadsklimat/gucg/people/
\item[-] Materialedata for torr luft (101325 Pa): http://www.formel.dk/materialedata/luft.htm
\item[-] Mycket tydligt om R-värden och U-värden: \\
http://sv.wikipedia.org/wiki/V\%C3\%A4rmemotst\%C3\%A5nd
\end{itemize}


%%%%%

\subsubsection{Övergångsmotstånd}
Hur det beror av fuktighet, vindhastighet, temperatur, solintensitet
\begin{equation}
U=\frac{1}{\Sigma R}
\end{equation}


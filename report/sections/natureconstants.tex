\section{Naturkonstanter för luft}

Vid olika tryck, fuktighet, temperatur och annat väder som påverkar. Luft består av 78,1\% kväve och 21,0\% syre (mer exakt http://en.wikipedia.org/wiki/Air\#Composition)

\subsection{Termisk diffusivitet} % Viktig
In heat transfer analysis, thermal diffusivity is the thermal conductivity divided by density and specific heat capacity at constant pressure. It has the SI unit of $m^2/s$. The formula is:

\begin{equation}
\alpha = {k \over {\rho c_p}}
\end{equation}

where
\begin{itemize}
   \item[] k is thermal conductivity, $\unit{W/(m·K)}$
   \item[] $\rho$ is density, $\unit{kg/m^3}$
   \item[] $c_p$ is specific heat capacity, $\unit{J/(kg·K)}$
\end{itemize}

Air: $\unit[1.9\cdot10^{-5}]{m/s^2}$

Källor:\\
http://www.electronics-cooling.com/2007/08/thermal-diffusivity/\\
http://en.wikipedia.org/wiki/Thermal\_diffusivity\\

\subsubsection{Termisk konduktivitet} % Viktigt
The property of a material's ability to conduct heat. Unit: W/(m·K).

\subsubsection{Densitet} % inte så viktigt

https://www.brisbanehotairballooning.com.au/faqs/education/116-calculate-air-density.html

Gäller så länge ideala gaslagen gäller vilket den gör vid lufttryck. Källa?


\subsubsection{Specifik värmekapacitivitet}
kan approximeras med konstanten 1.005 kJ/(kg·K (källa: http://www.engineeringtoolbox.com/air-properties-d\_156.html) (för torr luft).

\subsection{Volumetriska expansionskoefficienten} % Viktig
Vattnets expansionskoefficient är 1.8e-4 per grad (vid 'normalt' tryck). Källa: nationellt resurscentrum för fysik vid Lunds Universitet. Även vid hög luftfuktighet är det relativt lite vatten i luften. Då vattnet utvidgas så lite borde det inte spela någon roll.

För torr luft borde det gå att använda allmänna gaslagen.

\subsection{Kinematiska viskositeten} % Inte så viktig
Dynamisk viskositet luft: $\unit[16,7\cdot10^{-6}]{Pa\cdot sh}$, vid $0^\circ C$.
$\unit[1.78\cdot10^{-5}]{Pa\cdot sh}$, vid $15^\circ C$

Den kinematiska viskositeten $\nu$ definieras som: $\nu = \frac {\mu} {\rho}.$

där $\mu$ är den dynamiska viskositeten och $\rho=\unit[1,2041]{kg/m^3}$ är lufts densitet. \emph{(Källa Wikipedia, som hänvisar till Henrik Alvarez (1997). Energiteknik. Lund: Marinlitteratur.)}

Beroende på tryck och temperatur: http://en.wikipedia.org/wiki/File:Air\_dry\_dynamic\_visocity\_on\_pressure\_temperature.svg

\subsubsection{Sutherland's formula}
can be used to derive the dynamic viscosity of an ideal gas as a function of the temperature \emph{(källa: wikipeida som hänvisar till Alexander J. Smits, Jean-Paul Dussauge Turbulent shear layers in supersonic flow, Birkhäuser, 2006, ISBN 0387261400 p. 46)}

\begin{equation}
{\mu} = {\mu}_0 \frac {T_0+C} {T + C} \left (\frac {T} {T_0} \right )^{3/2}.
\end{equation}

Luft: C=120K, T0=291.15K, $\mu$0=18.27$\mu$Pa s

Valid for temperatures between 0 < T < 555 K with an error due to pressure less than 10\% below 3.45 MPa.

\subsection{Andra funderingar}
\begin{itemize}
\item[] Fråga någon på Geofysik, Klimat, t.ex. någon här: \\
http://www.gvc.gu.se/Forskning/klimat/stadsklimat/gucg/people/
\item[] Materialedata for tør luft (101325 Pa): http://www.formel.dk/materialedata/luft.htm
\end{itemize}

\subsection{Övergångsmotstånd}
Hur det beror av fuktighet, vindhastighet, temperatur, solintensitet
\begin{equation}
U=\frac{1}{\Sigma R}
\end{equation}
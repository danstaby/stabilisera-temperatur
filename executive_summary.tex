\documentclass[11pt,a4paper]{article}
\usepackage[utf8]{inputenc}
\usepackage[T1]{fontenc}
\usepackage[swedish]{babel}
\usepackage{amsmath}
\usepackage{ae}
\usepackage{units}
\usepackage{icomma}
\usepackage{color}
\usepackage{graphicx}
\usepackage{bbm}
\usepackage{textcomp}
\usepackage{url}
\usepackage{verbatim}
\usepackage{subfig}
\usepackage{marvosym}
\usepackage{eso-pic}
\usepackage{fancyhdr}
\usepackage{amssymb}
\usepackage[margin=1in]{geometry}
\usepackage{cite}
\usepackage{multicol}
\usepackage{wrapfig}
\usepackage{calc}
\usepackage{minibox}
\usepackage[table,dvipsnames]{xcolor}
\usepackage{multirow}

\begin{document}

\addtolength{\parindent}{-0.6 cm}
\pagenumbering{gobble}
\pagestyle{fancy}
\lhead{\sc\footnotesize Executive Summary till CTK Bachelor's Challenge}
\rhead{\sc\footnotesize \today}
\mbox{}
%\vspace{4mm}

\begin{center}
\textcolor{ForestGreen}{\textbf{\Huge För ett behagligt och grönt boende}}
\end{center}

\mbox{}
%\vspace{4mm}

\setlength{\columnsep}{5mm}
%\setlength{\columnseprule}{0.5mm}
\begin{multicols}{2}
\addtolength{\parskip}{1.3ex}
%\linespread{1.1}
\normalsize

% Ingress
\textbf{Vi har undersökt hur vädret påverkar energiflödena genom en fastighet. Genom att ta hänsyn till fler väderparametrar än utomhusflödet kan energikonsumtionen minskas vilket i förlängningen bevarar miljön. Dessutom fås ett jämnare, och där med trivsammare, inomhusklimat vilket bara är en positiv bieffekt i vår förhoppning att kunna minska energiförbrukingen.}

\emph{\color{red} Vilket är vårt viktigaste resultat.}
Under loppet av vårt kandidatarbete har det visat sig att både sol och vind kan
ha markant effekt på inomhusklimatet och således energiförbrukningen. Det 
förstnämnda leder till ett netto inflöde av energi vilket gör att en ganska
betydande del kan sparas (\emph{\color{red} Hur mycket?}). Vinden
ger däremot ett netto utflöde av energi och där går det inte att spara energi
om inte fastigheten har för höga temperaturer inomhus då det inte blåser utomhus.

Genom att ta hänsyn till nyss nämnda väderparametrar kan ett approximativt
värde för energiåtgång beräknas. Då kännedom om de momentana energiförlusterna
existerar så är det en lätt sak att reglera värmeanläggningen efter detta.
För att implementera vår modell behövs ingen ny värmeanläggning och väderdata är relativt enkelt att samla in och distribuera. 

\begin{comment}
Är man en liten bostadsrättsförening eller bo i villa kanske man tycker att väderstationen är en stor utgift – då kan man gå ihop några grannar och samla in väderdata gemensamt.\end{comment}

I och med att systemet är billigt att implementera och kan ge stora energivinster ligger det nära till hands att snart ha en färdig produkt ute på marknaden.

En lättillgängligt åtgärd kan nyttjas i många fastigheter och fler kan vara med och spara in på våra gemensamma resurser.


% visa stor förståelse för framtidens problem
% och teknologiska färdigheter och förståelse för marknaden.

Genom att beräkna de olika energiflödena – ledning, konvektion och strålning – genom en fastighets olika gränsytor – väggar, fönster, tak och grund – har det totala energiflödet vid olika väder kunnat beräknats. Detta har sedan jämfört med det energiflöde man tror sig få vid reglering efter enbart utomhustemperaturen. Då har vi sett att solinstrålning är en stor parameter och genom att inte värma huset när solen kan göra det minskas energiåtgången i fastigheten.


\renewcommand{\arraystretch}{1.2}
\noindent
\resizebox{8cm}{!} {
\begin{tabular}{l r}
\hline
\multicolumn{2}{|c|}{\cellcolor{YellowGreen} \textbf{Projektnamn:}}\\[3pt]
\multicolumn{2}{|c|}{\cellcolor{YellowGreen} \textit{Väder\!parametrars inverkan på}}\\
\multicolumn{2}{|c|}{\cellcolor{YellowGreen} \textit{energi\!f\,örluster i en fastighe\!t -}}\\
\multicolumn{2}{|c|}{\cellcolor{YellowGreen} \textit{en s\!tu\!die av värme\!fl\,öden}}\\
\multicolumn{2}{|c|}{\cellcolor{YellowGreen} \textbf{Gruppmedlemmar:}} \\[3pt]
\multicolumn{1}{|l}{\cellcolor{YellowGreen} Erik Ahlqvist (E)} & \multicolumn{1}{r|}{\cellcolor{YellowGreen} Ylva Dahl (F)}\\
\multicolumn{1}{|l}{\cellcolor{YellowGreen} Mats Lindström (F)} & \multicolumn{1}{r|}{\cellcolor{YellowGreen} Dan Ståby (F)}\\
\multicolumn{2}{|c|}{\cellcolor{YellowGreen} Institutionen för Teknisk Fysik} \\
\multicolumn{2}{|c|}{\cellcolor{YellowGreen} \textbf{Kategori:}} \\[3pt]
\multicolumn{2}{|c|}{\cellcolor{YellowGreen} Green Initiative \&}\\
\multicolumn{2}{|c|}{\cellcolor{YellowGreen} Sustainable Development}\\
\hline
& \\
\end{tabular}
}

Moderna passivhus är utnyttjar redan vädret till sin fördel, men många hus är redan byggda utan den tekniken. Med våra modeller kan man till viss del utnyttja vädrets gratis energitillförsel även i äldre byggnader. Genom att ta hänsyn till solen kan man spara upp till \emph{\color{red} ??\%} av sina energikostnader och genom att ta hänsyn till vind kan inomhusklimatet bli betydligt jämnare.


\emph{\color{red}Därför ska vi vinna!}
En essentiell del av dagens och morgondagens teknologiska utveckling måste vara
att minska energiförbrukningen genom effektivare processer. Då mycket energi
används för uppvärmning av fastigheter, speciellt i våra nordliga breddgrader
är det ypperligt att effektivisera uppvärmningsprocesser och
minimera energiflörluster. Just den typen av förbättring som vårt kandidatarbete
har studerat är väldigt bra på att uppfylla detta då det dels sparar mycket energi
och dels är billigt att implementera. Detta innebär att en produkt baserad på
denna princip även blir kommersiellt gångbar utan subventioner. Ett problem
som många andra miljöförbättringar har. 


\end{multicols}
\end{document}

\documentclass[11pt,a4paper]{article}
\usepackage[utf8]{inputenc}
\usepackage[T1]{fontenc}
\usepackage[swedish]{babel}
\usepackage{amsmath}
\usepackage{ae}
\usepackage{units}
\usepackage{icomma}
\usepackage{color}
\usepackage{graphicx}
\usepackage{bbm}
\usepackage{textcomp}
\usepackage{url}
\usepackage{verbatim}
\usepackage{subfig}
\usepackage{marvosym}
\usepackage{eso-pic}
\usepackage{fancyhdr}
\usepackage{amssymb}
\usepackage[margin=1in]{geometry}
\usepackage{cite}
\usepackage{multicol}
\usepackage{wrapfig}
\usepackage{calc}
\usepackage{minibox}
\usepackage[table,dvipsnames]{xcolor}
\usepackage{multirow}

\begin{document}

\addtolength{\parindent}{-0.6 cm}
\pagenumbering{gobble}
\pagestyle{fancy}
\lhead{\sc\footnotesize Executive Summary till CTK Bachelor's Challenge}
\rhead{\sc\footnotesize \today}
\mbox{}
\vspace{4mm}

\begin{center}
\textcolor{ForestGreen}{\textbf{\Huge För ett behagligt och grönt boende}}
\end{center}

\mbox{}
\vspace{4mm}

\setlength{\columnsep}{5mm}
%\setlength{\columnseprule}{0.5mm}
\begin{multicols}{2}
\addtolength{\parskip}{1.5ex}
\linespread{1.1}
\normalsize

% Ingress
\textbf{Vi har undersökt hur vädret påverkar energiflödena genom en fastighet. Genom att ta hänsyn till fler väderparametrar än utomhusflödet kan energikonsumtionen minskas vilket i förlängningen bevarar miljön. Dessutom fås ett jämnare, och där med trivsammare, inomhusklimat vilket bara är en positiv bieffekt i vår förhoppning att kunna minska energiförbrukingen.}

Vår kan snart bli en kommersiell produkt, vid kontakt med ett aktuellt företag. Väderdata är enkelt att samla och och detta kan bli en lättillgänglig energibesparande åtgärd att implementera i många fastigheter. I och med att den kan komma till nytta i många fastigheter kan fler vara med och hushålla med våra ändliga resurser.


% visa stor förståelse för framtidens problem
% och teknologiska färdigheter och förståelse för marknaden.

Genom att beräkna de olika energiflödena – ledning, konvektion och strålning – genom en fastighets olika gränsytor – väggar, fönster, tak och grund – har det totala energiflödet vid olika väder kunnat beräknats. Detta har sedan jämfört med det energiflöde man tror sig få vid reglering efter enbart utomhustemperaturen. Då har vi sett att solinstrålning är en stor parameter och genom att inte värma huset när solen kan göra det minskas energiåtgången i fastigheten.

Moderna passivhus är utnyttjar redan vädret till sin fördel, men många hus är redan byggda utan den tekniken. Med våra modeller kan man till viss del utnyttja vädrets gratis energitillförsel även i äldre byggnader. Genom att ta hänsyn till solen kan man spara upp till 20~\% av sina energikostnader och genom att ta hänsyn till vind kan inomhusklimatet bli betydligt jämnare.

\renewcommand{\arraystretch}{1.2}
\noindent
\resizebox{8cm}{!} {
\begin{tabular}{l r}
\hline
\multicolumn{2}{|c|}{\cellcolor{YellowGreen} \textbf{Projektnamn:}}\\[3pt]
\multicolumn{2}{|c|}{\cellcolor{YellowGreen} \textit{Väder\!parametrars inverkan på}}\\
\multicolumn{2}{|c|}{\cellcolor{YellowGreen} \textit{energi\!f\,örluster i en fastighe\!t -}}\\
\multicolumn{2}{|c|}{\cellcolor{YellowGreen} \textit{en s\!tu\!die av värme\!fl\,öden}}\\
\multicolumn{2}{|c|}{\cellcolor{YellowGreen} \textbf{Gruppmedlemmar:}} \\[3pt]
\multicolumn{1}{|l}{\cellcolor{YellowGreen} Erik Ahlqvist (E)} & \multicolumn{1}{r|}{\cellcolor{YellowGreen} Ylva Dahl (F)}\\
\multicolumn{1}{|l}{\cellcolor{YellowGreen} Mats Lindström (F)} & \multicolumn{1}{r|}{\cellcolor{YellowGreen} Dan Ståby (F)}\\
\multicolumn{2}{|c|}{\cellcolor{YellowGreen} Institutionen för Teknisk Fysik} \\
\multicolumn{2}{|c|}{\cellcolor{YellowGreen} \textbf{Kategori:}} \\[3pt]
\multicolumn{2}{|c|}{\cellcolor{YellowGreen} Green Initiative \&}\\
\multicolumn{2}{|c|}{\cellcolor{YellowGreen} Sustainable Development}\\
\hline
& \\
\end{tabular}
}

Varför ska vi vinna


\end{multicols}
\end{document}
